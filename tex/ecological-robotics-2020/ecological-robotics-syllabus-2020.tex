%!TEX TS-program = xelatex
%!TEX encoding = UTF-8 Unicode

%%%  Syllabus template for use with style files at http://kjhealy.github.com/latex-custom-kjh
%%%  Kieran Healy

\documentclass[11pt,article,oneside]{memoir}

% packages
\usepackage{org-preamble-xelatex}
\usepackage{wallpaper}
\usepackage{xcolor}
\usepackage{multicol}
\usepackage{enumitem}
\setlist[itemize]{leftmargin=*}

\AtBeginBibliography{\small}

% Definitions
\def\myauthor{Author}
\def\mytitle{Title}
\def\mycopyright{\myauthor}
\def\mykeywords{}
\def\mybibliostyle{plain}
\def\mybibliocommand{}
\def\mysubtitle{}
\def\myaffiliation{Louisiana State University}
\def\myaddress{Robotics Lab, Patrick Taylor Hall}
\def\myemail{mdeque1@lsu.edu, hbgilbert@lsu.edu, baharmon@lsu.edu, hyenam@lsu.edu, \& cbarbalata@lsu.edu}
\def\myweb{https://baharmon.github.io/}
\def\myphone{919.622.8414}
\def\myversion{}
\def\myrevision{}
\def\myaffiliation{\ \\Louisiana State University}
%\def\myauthor{\normalsize M.~de Queiroz, H.~Gilbert, B.~Harmon, H.Y.~Nam, \& C.~Barbalata}
\def\myauthor{\footnotesize Marcio de Queiroz, Hunter Gilbert, Brendan Harmon, Hye Yeon Nam, \& Corina Barbalata}
\def\mykeywords{Landscape Architecture, Syllabus, Graduate}
\def\mysubtitle{Syllabus}
\def\mytitle{
\includegraphics[height=0.95cm]{../logos/lsu_coe_logo.jpg}\\
\hspace*{2.79cm}+\\ \vspace*{0.275cm}
\includegraphics[height=1cm]{../logos/lsu_art_design_logo.pdf}
\vspace*{0.5cm}
\\[0.1cm] {\normalfont \normalsize ME 4933 \& LA 7504 \& ART 7255 \& DART 7020 |} \Large Ecological Robotics
} 

% color
\makeatletter
\newcommand{\globalcolor}[1]{%
  \color{#1}\global\let\default@color\current@color
}
\makeatother

% begin
\begin{document}

\setlength\bibitemsep{0.5em}

% fonts
\defaultfontfeatures{}
\defaultfontfeatures{Scale=MatchLowercase}         
\setmainfont{Lato Regular}
\setmonofont[Scale=0.8]{IBM Plex Mono}

\def\ind{\hangindent=1 true cm\hangafter=1 \noindent}
\def\labelitemi{$\cdot$}
\chapterstyle{article-4-sans}  

\title{\LARGE \mytitle}
\author{\Large\myauthor \newline \footnotesize\texttt{\noindent\myemail}}
\date{Robotics Lab, Patrick Taylor Hall 
\newline Tuesday \& Thursday 4:30pm--6:00pm
\newline Spring 2020}
\published{\,}

% -------------------------------- COVER PAGE -------------------------------- 

\pagenumbering{gobble}
\globalcolor{black}
\vspace*{-10em}
\maketitle
\vspace*{1cm}
\ThisCenterWallPaper{1}{../images/robotic-garden-01.png}
\clearpage


% -------------------------------- DESCRIPTION -------------------------------- 

\pagenumbering{arabic}
\globalcolor{black}

\vspace*{-10em}
\maketitle

\section{Course Description}
Ecological robotics is an introduction to 
environmental applications for robots. 
%
In this course you will learn how to 
build DIY robots, program industrial robots,
and design custom end effectors. 
%
Through a series of projects 
you will design, prototype, and program 
a robotic process for planting. 
%
This interdisciplinary course is open to students in the 
College of Art \& Design and the College of Engineering. 
\\

\noindent\textbf{Keywords}
\begin{multicols}{3}
\raggedright
\small
\begin{itemize}
\item Digital fabrication
\item Generative design
\item Visual programming
\item Soft robotics
\item DIY robotics
\item Robotic gardening
\item 3D modeling \& rendering
\item Machine vision
\item Image classification
\end{itemize}
\end{multicols}

% -------------------------------- SCHEDULE -------------------------------- 
\section{Topics}
%
\begin{table}[H]
\begin{tabular}{l @{\hskip 0.7cm} l @{\hskip 2.1cm} l}
\textbf{Robotics in Art,} & \textbf{Ecological Robots} & \textbf{Ecological Robots}\\
\textbf{Design, \& Engineering} & \textbf{in the Lab} & \textbf{in the Field}\\
\end{tabular}
\end{table}
%
\vspace*{-1em}
%
\begin{table}[H]
\begin{tabular}{l l l l l l}
\small
\textbf{1} & Robotics in A, D, \& E & \textbf{6} & Horticulture for robots & \textbf{11} & Environmental apps.\\
\textbf{2} & Our robots & \textbf{7} & Delivery system & \textbf{12} & Prototyping\\
\textbf{3} & Intro to robotics & \textbf{8} & Digital fabrication & \textbf{13} & Prototyping\\
\textbf{4} & Visual programming & \textbf{9} & Prototyping & \textbf{14} & Prototyping\\
\textbf{5} & Pick \& place project & \textbf{10} & Lab planting project & \textbf{15} & Applied project\\
\end{tabular}
\end{table}
%

%\begin{tabular}{l l @{\hskip 1cm}l}

\clearpage

% -------------------------------- SCHEDULE -------------------------------- 
\section{Course Schedule}

\begin{table}[H]
%
\normalsize \textbf{Robotics in Art, Design, \& Engineering}\\
%
\small{%
\begin{tabular}{l r @{\hskip 0.1cm} l @{\hskip 0.75cm} l}
\\
\textbf{Week 1} & \textbf{Lecture |} & Robotics in art, design, \& eng. \\
\textbf{} & \textbf{Workshop |} & DIY robotics\\
%
\textbf{Week 2} & \textbf{Tour |} & Robots \& fabrication facilities\\
\textbf{} & \textbf{Demo |} & Industrial robotic operations\\
%
\textbf{Week 3} & \textbf{Lecture |} & Intro to robotics I\\
\textbf{} & \textbf{Lecture |} & Intro to robotics II\\
%
\textbf{Week 4} & \textbf{Workshop |} & Installation party\\
\textbf{} & \textbf{Tutorial |} & Visual programming \\
%
\textbf{Week 5} & \textbf{Workshop |} & Prototyping \\
\textbf{} & \textbf{Review |} & Project demonstrations & \textbf{Project:} Pick \& place\\
\\
\end{tabular}}\\
%
\normalsize \textbf{Ecological Robotics in the Lab}\\
%
\small{%
\begin{tabular}{l r @{\hskip 0.1cm} l @{\hskip 0.75cm} l}
\\
\textbf{Week 6} & \textbf{Lecture |} & Horticulture for robots\\
\textbf{} & \textbf{Tutorial |} & Modeling plants in 3D\\
%
\textbf{Week 7} & \textbf{Lecture |} & Soft robotics\\
\textbf{} & \textbf{Workshop |} & Delivery system ideation\\ 
%
\textbf{Week 8} & \textbf{Tutorial |} & Digital fabrication\\
\textbf{} & \textbf{Workshop |} & Delivery system prototyping\\ 
%
\textbf{Week 9} & \textbf{Workshop |} & Prototyping I\\ 
\textbf{} & \textbf{Workshop |} & Prototyping II\\ 
%
\textbf{Week 10} & \textbf{Workshop |} & Documentation\\ 
\textbf{} & \textbf{Review |} & Project demonstrations & \textbf{Project:} Laboratory planting\\
\\
\end{tabular}}\\
%
\normalsize \textbf{Ecological Robotics in the Field}\\
%
\small{%
\begin{tabular}{l r @{\hskip 0.1cm} l @{\hskip 1cm} l}
\\
%greenwalls, wetland restoration, and/or precision agriculture
\textbf{Week 11} & \textbf{Lecture |} & Environmental applications \\
\textbf{} & \textbf{Workshop |} & Ideation\\
%
\textbf{Week 12} & \textbf{Tutorial |} & Generative design\\
\textbf{} & \textbf{Workshop |} & Prototyping I\\
%
\textbf{Week 13} & \textbf{Workshop |} & Prototyping II\\
\textbf{} & \textbf{Workshop |} & Prototyping III\\
%
\textbf{Week 14} & \textbf{Workshop |} & Prototyping IV\\
\textbf{} & \textbf{Workshop |} & Prototyping V\\
%
\textbf{Week 15} & \textbf{Workshop |} & Documentation\\
\textbf{} & \textbf{Review |} & Project presentations & \textbf{Project:} Applied planting\\ 
%
\end{tabular}}
\end{table}

\clearpage

% -------------------------------- Projects -------------------------------- 
\section{Projects}
As interdisciplinary teams you will develop 
novel environmental applications for robots.
\\

\noindent \textbf{Pick \& place}
Program an industrial robotic arm to pick and place
potted plants in a procedurally defined pattern.
Record a video of your robot in action.
\\

\noindent \textbf{Laboratory planting}
Design, fabricate, and test a robotic process for planting.
Use a collaborative, industrial robotic arm to
deploy and test your prototype.
Focus on an aspect of planting like 
seed delivery, sapling delivery,
water and nutrient delivery, or
grading topography. 
Record a video of your robotic system in action.
\\

\noindent \textbf{Applied planting}
Design, illustrate, and document
a robotic system for planting a green wall.
Deliverables include 
a poster illustrating your design,
3D models and CAD drawings of your design, 
a CNC milled greenwall module,
and a 3D printed scale model of your design.
\\

% -------------------------------- Software -------------------------------- 
\section{Software}
%\begin{multicols}{2}
%\raggedright
SolidWorks | \url{https://www.solidworks.com/}\\
Rhinoceros | \url{https://www.rhino3d.com/}\\
Grasshopper | \url{http://grasshopper3d.com/}\\
HAL Robotics Framework | \url{https://hal-robotics.com/}\\
Thea Render | \url{https://www.thearender.com}\\
Xfrog | \url{https://xfrog.com/}
%\end{multicols}

% -------------------------------- Resources -------------------------------- 
\section{Resources}
Grasshopper Primer | \url{http://grasshopperprimer.com}\\
Intro to HAL Playlist | \url{http://bit.ly/hal-playlist}

% -------------------------------- Grading -------------------------------- 
\section{Grading}
%
\begin{table}[H]
%\small
\begin{tabular}{l r @{\hskip 2cm} l @{\hskip 0.5cm} l}
%\begin{tabular}{l l}
%
Pick \& place & 20\% \\
Laboratory planting & 40\%  \\
Applied planting & 40\% \\
%
\end{tabular}
\end{table}

\clearpage

% -------------------------------- Readings -------------------------------- 
\section{Readings}
\vspace*{0.5cm}
\nocite{*}
\setlength\bibitemsep{0.65\baselineskip}
\printbibliography[heading=none]
\clearpage
% -------------------------------- Policies -------------------------------- 
\section{Policies}

\noindent \textbf{Time Commitment Expectations}
LSU's general policy states that for each credit hour, you (the student) should plan to
spend at least two hours working on course related activities outside of class. Since this course is for three credit hours, you should expect to spend a minimum of six hours outside of class each week working on assignments for this course. For more information see: 
\url{http://catalog.lsu.edu/content.php?catoid=12&navoid=822}.\\

\noindent \textbf{LSU student code of conduct}
The LSU student code of conduct explains student rights, excused absences, and what is expected of student behavior. Students are expected to understand this code:  \url{http://students.lsu.edu/saa/students/code}.\\ %Any violations of the LSU student code will be duly reported to the Dean of Students.\\

\noindent \textbf{Disability Code}
The University is committed to making reasonable efforts to assist individuals with disabilities in
their efforts to avail themselves of services and programs offered by the University. To this end,
Louisiana State University will provide reasonable accommodations for persons with
documented qualifying disabilities. If you have a disability and feel you need accommodations in
this course, you must present a letter to me from Disability Services in 115 Johnston Hall,
indicating the existence of a disability and the suggested accommodations.\\

\noindent \textbf{Academic Integrity}
According to section 10.1 of the LSU Code of Student Conduct, ``A student may be charged with Academic Misconduct'' for a variety of offenses, including the following: unauthorized copying, collusion, or collaboration; ``falsifying'' data or citations; ``assisting someone in the commission or attempted commission of an offense''; and plagiarism, which is defined in section 10.1.H as a ``lack of appropriate citation, or the unacknowledged inclusion of someone else's words, structure, ideas, or data; failure to identify a source, or the submission of essentially the same work for two assignments without permission of the instructor(s).''\\

\noindent \textbf{Plagiarism and Citation Method}
Plagiarism is the ``lack of appropriate citation, or the unacknowledged inclusion of someone else's words, structure, ideas, or data; failure to identify a source, or the submission of essentially the same work for two assignments without permission of the instructor(s)'' (Sec. 10.1.H of the LSU Code of Student Conduct). As a student at LSU, it is your responsibility to refrain from plagiarizing the academic property of another and to utilize appropriate citation method for all coursework. In this class, it is recommended that you use Chicago Style author-date citations. Ignorance of the citation method is not an excuse for academic misconduct.

\end{document}
