%!TEX TS-program = xelatex
%!TEX encoding = UTF-8 Unicode

% Class
\documentclass[11pt,article,oneside]{memoir}

% Packages
\usepackage{../style/xelatex-preamble}

% Settings
\AtBeginBibliography{\small}

% Definitions
\def\myauthor{Author}
\def\mytitle{Title}
\def\mycopyright{\myauthor}
\def\mykeywords{}
\def\mybibliostyle{plain}
\def\mybibliocommand{}
\def\mysubtitle{}
\def\myaffiliation{Louisiana State University}
\def\myemail{baharmon@lsu.edu} 
\def\myweb{https://baharmon.github.io/}
\def\myphone{919.622.8414}
\def\myversion{}
\def\myrevision{}
\def\myaffiliation{\ \\Louisiana State University}
\def\myauthor{Brendan Harmon}
\def\mykeywords{Landscape Architecture, Syllabus, Graduate}
\def\mysubtitle{Syllabus}
\def\mytitle{ \includegraphics[width=6cm]{../logos/lsu_art_design_logo.pdf} \\
[0.1cm] {\normalfont \normalsize LA 7504 |} \Large Computational Design} 

% Color
\makeatletter
\newcommand{\globalcolor}[1]{%
  \color{#1}\global\let\default@color\current@color
}
\makeatother

% Begin
\begin{document}

% Fonts
\defaultfontfeatures{}
\defaultfontfeatures{Scale=MatchLowercase}
\setmainfont{IBM Plex Sans}
\setmonofont[Scale=0.8]{IBM Plex Mono}

% Style
\setlength\bibitemsep{0.5em}
\def\ind{\hangindent=1 true cm\hangafter=1 \noindent}
\def\labelitemi{$\cdot$}
\chapterstyle{syllabus}

% Frontmatter
\title{\LARGE \mytitle}
\author{\Large\myauthor \newline \footnotesize\texttt{\noindent\myemail}}
\date{Spring 2026 \newline Tuesday \& Thursday \newline 14:30--15:50 \newline Design 308}
\published{\,}

% -------------------------------- Cover Page -------------------------------- 

\pagenumbering{gobble}
\globalcolor{black}
\vspace*{-10em}
\maketitle
\ThisCenterWallPaper{1.1}{../images/hilltop-lidar.png}
\clearpage

% -------------------------------- Description -------------------------------- 

\pagenumbering{arabic}
\globalcolor{black}

\vspace*{-10em}
\maketitle

\ThisCenterWallPaper{1}{../images/robotic-wall-2.png}

\vfill

\section{Course Description}

This course is an introduction to 
computational design for architects and landscape architects.
In this course you will learn visual programming with Grasshopper
and use algorithms to generate, analyze, and build designs.
Experiment with emerging technologies 
for the design and construction of the built environment.
Learn how to procedurally model landscape and urban form,
record and analyze your environment, 
numerically simulate physical processes 
such as the flow of wind and water,
digitally fabricate models,
and autonomously construct structures.

% -------------------------------- Schedule ----------------------------

\section{Schedule}

\begin{table}[H]
\begin{tabular}{@{} l @{\hskip 7.3em} l @{\hskip 10.2em} l}
\textbf{Fundamentals} & \textbf{Modeling} & \textbf{Fabrication}\\
\end{tabular}
\end{table}

\vspace*{-1em}

\begin{table}[H]
\small
\begin{tabular}{@{} l l l l l l}
\small
\textbf{1} & Introduction \hspace{6em}  & \textbf{6} & Physics  \hspace{8em} & \textbf{11} & Soundscapes\\
\textbf{2} & Stochasticity & \textbf{7} & Terrain & \textbf{12} & Machining\\
\textbf{3} & Randomness & \textbf{8} & Earthworks & \textbf{13} & Printing\\
\textbf{4} & Attractors & \textbf{9} & -- & \textbf{14} & Robotics\\
\textbf{5} & Point Clouds & \textbf{10} & Noise & \textbf{15} & Construction\\
\end{tabular}
\end{table}

\clearpage

% ---------------------------- Assignments -----------------------------

\section{Assignments}

\paragraph{1 Computational Design} Learn the fundamentals of Grasshopper \dotfill

\paragraph{2 Stochasticity} Find the cumulative sum of random numbers to model a random walk \dotfill

\paragraph{3 Randomness} Use randomness to generate procedural paving and planting \dotfill

\paragraph{4 Attractors} Use attractors to generate gradients of topography, paving, and planting \dotfill

\paragraph{5 Point Clouds} Model a landscape using a library of scanned point clouds \dotfill

\paragraph{6 Physics} Use a physics engine to simulate gravity and solve geometric packing problems \dotfill 

\paragraph{7 Terrain} Acquire topographic data, model the terrain, and then model its hydrology \dotfill

\paragraph{8 Earthworks} Use cut and fill operations to procedurally model landforms \dotfill

\paragraph{9 ---} \dotfill

\paragraph{10 Noise} Generate terrain from cellular noise and its biome from gradient noise \dotfill

\paragraph{11 Soundscapes} Plot the spectrogram of a bioacoustic recording \dotfill

\paragraph{12 Machining} CNC mill a terrain model generated from cellular noise \dotfill

\paragraph{13 Printing}  3D print a spectrogram in clay \dotfill

\paragraph{14 Robotics} Program pick-and-place operations with a 6-axis collaborative robotic arm \dotfill

\paragraph{15 Autonomous Construction} Program a robotic arm to assemble a procedural brick wall \dotfill

% -------------------------------- Projects -------------------------------- 

\section{Project}

\paragraph{Point Cloud Landscape}
Experiment with point clouds as a new design medium
by procedurally designing, modeling, and rendering a landscape.
First use algorithms to procedurally design a landscape.
Then model your design as a point cloud using assets 
from a library of laser scanned or neural rendered landscape elements. 
Upload your design to \href{https://sketchfab.com/}{Sketchfab}.
\textbf{Due:} Week 15

% -------------------------------- Grading -------------------------------- 

\section{Grading}
%
\begin{table}[H]
\begin{tabular}{ @{} l @{\hskip 1em} l @{\hskip 9em} l @{\hskip 1em} l @{\hskip 9em} l @{\hskip 1em} l}
\textbf{Assignments} & 60\% & \textbf{Project} & 35\% & \textbf{Portfolio} & 5\% \\
\end{tabular}
\end{table}

% -------------------------------- Logistics -------------------------------- 

\section{Logistics}

We will meet on Tuesdays and Thursdays 
from 14:30--15:50
in Design 308. 
All course content including tutorials, lectures, and datasets
will be published on the course
\href{https://baharmon.github.io/tutorials/design}{website}.
Our Discord \href{https://discord.gg/DcKmz6wZBe}{server}
will be used for posting announcements, 
student work in progress, reading responses, 
projects, and troubleshooting. 
During class please post each step of your work
to your channel on the 
\href{https://discord.gg/DcKmz6wZBe}{server}.
\\

\noindent
Website | \url{https://baharmon.github.io/tutorials/design}\\
Server | \url{https://discord.gg/DcKmz6wZBe}\\ 
Videos | \url{https://www.youtube.com/@baharmon}

% -------------------------------- Software -------------------------------- 
\section{Software}

Rhinoceros | \url{https://www.rhino3d.com/}\\
Python | \url{https://www.python.org/}

% ---------------------------------- Plugins ---------------------------------- 

\section{Plugins}
Snapping Gecko | \url{https://www.food4rhino.com/app/snappinggecko}\\
Bitmap+ | \url{https://www.food4rhino.com/en/app/bitmap}\\
Docofossor | \url{https://www.food4rhino.com/app/docofossor}\\
Dendro | \url{https://www.food4rhino.com/en/app/dendro}\\
RhinoCAM | \url{https://mecsoft.com/rhinocam-software/}%\\

% -------------------------------- Precedents ------------------------------- 

%\section{Precedents} 
%Philipp Urech, 2020, \href{https://skfb.ly/pztrn}{Parque Copan} \\
%ETHZ, Endless Wall
%ETHZ, Kitvus Vinery
%ETHZ, Circularity Park
% Snohetta, MAX Lab Landscape

% -------------------------------- Resources --------------------------------
\section{Dataset}
Computational Design Dataset | \url{https://doi.org/10.5281/zenodo.8191264} \\
Cloud Forest Library | \url{https://xyz.cct.lsu.edu/cloud-forest} \\
Cloud Forest Dataset | \url{https://doi.org/10.5281/zenodo.8194066}

\clearpage

% -------------------------------- Resources -------------------------------- 
\section{Resources}

Jump Start Grasshopper | \url{https://vimeo.com/showcase/9410915}\\
Grasshopper Guides | \url{https://developer.rhino3d.com/guides/grasshopper}\\
Grasshopper Primer | \url{http://grasshopperprimer.com}\\
TU Delft DigiPedia | \url{https://digipedia.tudelft.nl/software/grasshopper}\\
Parametric Camp | \url{https://www.youtube.com/parametriccamp}

% -------------------------------- Readings -------------------------------- 
\section{Required Readings}
\vspace*{0.5cm}
\nocite{*}
\setlength\bibitemsep{0.65\baselineskip}
\printbibliography[keyword=required, heading=none]

% -------------------------------- Readings -------------------------------- 
\section{Recommended Readings}
\vspace*{0.5cm}
\nocite{*}
\setlength\bibitemsep{0.65\baselineskip}
\printbibliography[keyword=recommended, heading=none]

% -------------------------------- Policies -------------------------------- 

\section{Policies}

\paragraph{Time Commitment Expectations}
LSU's general policy states that for each credit hour, 
you (the student) should plan to
spend at least two hours working 
on course related activities outside of class. 
Since this course is for three credit hours, 
you should expect to spend a minimum of six hours 
outside of class each week 
working on assignments for this course. 
For more information see: 
\url{https://catalog.lsu.edu}.

\paragraph{LSU student code of conduct}
The LSU student code of conduct explains 
student rights, excused absences, 
and what is expected of student behavior. 
Students are expected to understand this code:  
\url{https://www.lsu.edu/saa/students/codeofconduct.php}.

\paragraph{Disability Code}
The University is committed to making reasonable efforts 
to assist individuals with disabilities in
their efforts to avail themselves of 
services and programs offered by the University. 
To this end, Louisiana State University will provide 
reasonable accommodations for persons
with documented qualifying disabilities.
If you have a disability and 
feel you need accommodations in this course, 
you must present a letter to me 
from Disability Services in 115 Johnston Hall,
indicating the existence of a disability 
and the suggested accommodations.

\paragraph{Academic Integrity}
According to section 10.1 
of the LSU Code of Student Conduct, 
``A student may be charged with Academic Misconduct'' 
for a variety of offenses, including the following: 
unauthorized copying, collusion, or collaboration,
falsifying data or citations,
``assisting someone in the commission 
or attempted commission of an offense'',
and plagiarism.

\paragraph{Plagiarism and Citation Method}
Plagiarism is the 
``lack of appropriate citation, or the unacknowledged inclusion 
of someone else's words, structure, ideas, or data; 
failure to identify a source, 
or the submission of essentially the same work 
for two assignments without permission of the instructor(s)'' 
(Sec. 10.1.H of the LSU Code of Student Conduct). 
As a student at LSU, 
it is your responsibility to refrain from 
plagiarizing the academic property of another 
and to utilize appropriate citation method for all coursework. 
In this class, it is recommended that you use 
Chicago Style author-date citations. 
Ignorance of the citation method
is not an excuse for academic misconduct.

\paragraph{Generative Artificial Intelligence}
In this course the use of generative artificial intelligence (AI) 
is permitted for the purposes of enhancing your understanding 
of course materials, encouraging creative exploration, 
and supporting academic growth. 
The use of generative AI, however, 
is not permitted for writing assignments. 
These programs should not be used to produce work 
that misrepresents your abilities 
or deceives as to the conditions 
under which the work was completed. 
If you use AI to generate content 
you must clearly acknowledge the use of AI generated material. 
Proper attribution of AI program use
should include an explanation of how the program 
contributed to the assignment and your academic growth. 
Failing to give proper attribution 
to the use of AI programs in academic work
will be reported to Student Advocacy \& Accountability 
for review under the Code of Student Conduct 
and may result in impacts to your assignment and course grades.

\paragraph{Accreditation}
LSU's Robert Reich School of Landscape Architecture (RRSLA) 
must meet the accreditation requirements 
as stated by the Landscape Architectural Accreditation
Board (LAAB) to ensure the school
 is meeting the expectations of the field. 
The LAAB requires programs to provide digital copies 
of student work as part of this process.
Students in this course will be expected 
to comply with the following requirements
as 5\% of their course grade: 
(1) Students must provide a course portfolio
with work samples specified by the instructor 
before the end of the grading period. 
(2) Each student's course portfolio must be saved as 
a single, high resolution PDF file with multiple pages. 
(3) Files must follow the naming convention
established by the school: 
department-coursenumber-semesteryear-username.pdf.
Example: LA7504-S2026-baharmon.pdf.

\end{document}
