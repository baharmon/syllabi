%!TEX TS-program = xelatex
%!TEX encoding = UTF-8 Unicode

%%%  Syllabus template for use with style files at http://kjhealy.github.com/latex-custom-kjh
%%%  Kieran Healy

\documentclass[11pt,article,oneside]{memoir}

% packages
\usepackage{org-preamble-xelatex}
\usepackage{wallpaper}
\usepackage{xcolor}
\usepackage{multicol}
\usepackage{enumitem}
\setlist[itemize]{leftmargin=*}

\AtBeginBibliography{\small}

% Definitions
\def\myauthor{Author}
\def\mytitle{Title}
\def\mycopyright{\myauthor}
\def\mykeywords{}
\def\mybibliostyle{plain}
\def\mybibliocommand{}
\def\mysubtitle{}
\def\myaffiliation{Louisiana State University}
\def\myaddress{309 Design}
\def\myemail{baharmon@lsu.edu} 
\def\myweb{https://baharmon.github.io/}
\def\myphone{919.622.8414}
\def\myversion{}
\def\myrevision{}
\def\myaffiliation{\ \\Louisiana State University}
\def\myauthor{Brendan Harmon}
\def\mykeywords{Landscape Architecture, Syllabus, Graduate}
\def\mysubtitle{Syllabus}
\def\mytitle{ \includegraphics[width=6cm]{../logos/lsu_art_design_logo.pdf} \\[0.1cm] {\normalfont \normalsize LA 7032 |} \Large Geospatial Modeling} 

% color
\makeatletter
\newcommand{\globalcolor}[1]{%
  \color{#1}\global\let\default@color\current@color
}
\makeatother

% begin
\begin{document}

\setlength\bibitemsep{0.5em}

% fonts
\defaultfontfeatures{}
\defaultfontfeatures{Scale=MatchLowercase}         
\setmainfont[Scale=1, Path = ../fonts/lato/,BoldItalicFont=Lato-RegIta,BoldFont=Lato-Reg,ItalicFont=Lato-LigIta]{Lato-Lig}
\setsansfont[Scale=1, Path = ../fonts/lato/,BoldItalicFont=Lato-RegIta,BoldFont=Lato-Reg,ItalicFont=Lato-LigIta]{Lato-Lig}
\setmonofont[Mapping=tex-text,Scale=0.8,Path = ../fonts/inconsolata/]{i}

\def\ind{\hangindent=1 true cm\hangafter=1 \noindent}
\def\labelitemi{$\cdot$}
\chapterstyle{article-4-sans}  

\title{\LARGE \mytitle}
\author{\Large\myauthor \newline \footnotesize\texttt{\noindent\myemail}}
\date{Fall 2018. Design 217.\newline Tuesday \& Thursday 12:30am--3:20pm.}
\published{\,}

% -------------------------------- COVER PAGE -------------------------------- 

\pagenumbering{gobble}
\globalcolor{black}
\vspace*{-10em}
\maketitle
\ThisCenterWallPaper{1}{../images/yosemite.png}
\clearpage


% -------------------------------- DESCRIPTION -------------------------------- 

\pagenumbering{arabic}
\globalcolor{black}

\vspace*{-10em}
\maketitle

\section{Course Description}

This course is an introduction to 
geospatial modeling for landscape architects.
In this course you will develop a creative digital design process
seamlessly integrating research and design
using geographic information systems (GIS),
3D modeling and rendering, and
visual programming.
You will learn how to use geospatial data
to model and analyze landscapes
and visual programming to
parametrically model and transform new landforms.
You will learn how to model plants -- from trees to grasses -- in 3D,
automatically distribute them across your digital landscape,
and render photorealistic scenes.
Through a series of projects you will design the restoration 
of a highly eroded landscape with a deep gully.\\

%\noindent \textbf{Objectives}
%\begin{itemize}
%\item To model, analyze, and simulate physical patterns and processes
%\item To integrate geospatial modeling into your creative design process
%\item To design the restoration of a highly eroded landscape
%\end{itemize}

% -------------------------------- SCHEDULE -------------------------------- 
\section{Topics}
%
\begin{table}[H]
\begin{tabular}{l @{\hskip 1.25cm} l @{\hskip 1.8cm} l}
\textbf{Geospatial modeling} & \textbf{Generative design} & \textbf{Ecosystem modeling}\\
\end{tabular}
\end{table}
%
\vspace*{-1em}
%
\begin{table}[H]
\begin{tabular}{l l l l l l}
\small
\textbf{1} & Lidar & \textbf{6} & Parametric design & \textbf{11} & Particle systems\\
\textbf{2} & Terrain modeling & \textbf{7} & Parametric landforms & \textbf{12} & Image classification\\
\textbf{3} & Digital fabrication & \textbf{8} & Visual programming & \textbf{13} & 3D rendering\\
\textbf{4} & Hydrological modeling & \textbf{9} & Generative processes & \textbf{14} & Freeform modeling\\
\textbf{5} & Erosion modeling & \textbf{10} & Attractors & \textbf{15} & Virtual reality\\
\end{tabular}
\end{table}
%

%\begin{tabular}{l l @{\hskip 1cm}l}

\clearpage

% -------------------------------- SCHEDULE -------------------------------- 
\section{Course Schedule}

\begin{table}[H]
\begin{tabular}{l r @{\hskip 0.1cm} l @{\hskip 0.5cm} l}
\\
\normalsize
\textbf{Geospatial}\\
\small
\\
\textbf{08.21.2018} & \textbf{Tutorial |} & Lidar \\
\textbf{08.23.2018} & \textbf{Fieldwork |} & Surveying & \textbf{Project:} Geospatial monitoring \\
%
\textbf{08.28.2018} & \textbf{Tutorial |} & Terrain modeling I \\
\textbf{08.30.2018} & \textbf{Tutorial |} & Terran modeling  II \\
%
\textbf{09.04.2018} & \textbf{Tutorial |} & Digital fabrication \\
\textbf{09.06.2018} & \textbf{Lab |} & Digital fabrication & \textbf{Project:} CNC machining\\
%
\textbf{09.11.2018} & \textbf{Tutorial |} & Hydrological modeling I \\
\textbf{09.13.2018} & \textbf{Tutorial |} & Hydrological modeling II \\
\textbf{09.15.2018} & \textbf{Workshop |} & Drone photogrammetry \\
%
\textbf{09.18.2018} & \textbf{Tutorial |} & Erosion modeling \\
\textbf{09.20.2018} & \textbf{Tutorial |} & Landscape evolution & \textbf{Project:} Physical simulation \\
\\
\normalsize
\textbf{Generative}\\
\small
\\
\textbf{09.25.2018} & \textbf{Tutorial |} & Parametric design & \textbf{Project:} Laser-cut bench \\
\textbf{09.27.2018} & \textbf{Tutorial |} & Parametric landforms \\
%
\textbf{10.02.2018} & \textbf{Tutorial |} & Visual programming\\
%
\textbf{10.09.2018} & \textbf{Tutorial |} & Generative processes I \\
\textbf{10.11.2018} & \textbf{Tutorial |} & Generative processes II \\ 
%
\textbf{10.16.2018} & \textbf{Tutorial |} & Attractors \\
\textbf{10.18.2018} & \textbf{Studio |} & Attractors & \textbf{Project:} Families of form \\
\\
\normalsize
\textbf{Ecosystem}\\
\small
\\
\textbf{10.30.2018} & \textbf{Tutorial |} & 3D plants \\
\textbf{11.01.2018} & \textbf{Tutorial |} & Particle systems & \textbf{Project:} The Great Piece of Turf \\
%
\textbf{11.06.2018} & \textbf{Tutorial |} & Image classification \\
\textbf{11.08.2018} & \textbf{Tutorial |} & Landscape modeling \\
%
\textbf{11.13.2018} & \textbf{Tutorial |} & Landscape rendering I \\
\textbf{11.15.2018} & \textbf{Tutorial |} & Landscape rendering II \\
%
\textbf{11.20.2018} & \textbf{Tutorial |} & Freeform modeling I \\
\textbf{11.22.2018} & \textbf{Studio |} & Freeform modeling II \\
%
\textbf{11.27.2018} & \textbf{Tutorial |} & Virtual reality\\
\textbf{11.29.2018} & \textbf{Gallery |} & Final review & \textbf{Project:} Gully restoration \\ 
%
\end{tabular}
\end{table}

\clearpage

% -------------------------------- Paper -------------------------------- 
%\section{Paper}
%\noindent \textbf{The Alphabet and Algorithm}
%Read Mario Carpo's book \emph{The Alphabet and Algorithm}
%and write a 2000-word critical essay about
%the evolving nature of architectural authorship.
%Address how digital tools have transformed 
%the practice of landscape architecture
%and envision how they will shape 
%the future of the discipline. 
%
%%\noindent In preparation for this course please read:
%\nocite{*} \printbibliography[keyword=intro, heading=none]

% overleaf

% -------------------------------- Projects -------------------------------- 
\section{Projects}
The study landscape is a highly eroded watershed
feeding into Patterson Branch Creek in Fort Bragg, North Carolina.
This geomorphologically active watershed has deep gullies and
extensive areas of bare soils.
The creeks, streams, and rivers in this region are critical habitat
for endangered species of mussels.
Mussels, however, require stable stream habitat
and these actively eroding watersheds with sandy-loam soils
cause high sediment bed loads in streams and shifting streambeds.
The aim of this project is to restore the degraded watershed
and reduce sediment transport  
in order to enhance downstream mussel habitat.\\

\noindent \textbf{Gully monitoring}
Visit Clark Creek to survey active gullies.
Take photographs and 360 degree photospheres and
conduct a terrestrial lidar scan.
Back in the lab generate digital surface models (DSM)
from the lidar data.\\

\noindent \textbf{CNC machining}
Use the CNC router to machine a physical model
of the study landscape out of high density urethane foam.\\

\noindent \textbf{Physical simulation}
Using the CNC-machined model as a base
develop a physical simulation of sediment flow.
Experiment with casting, melting, and pouring. 
Try materials like hot wax or molten aluminum. 
Record your physical simulation as a video.\\

\noindent \textbf{Parametric bench}
With Rhino or Grasshopper
design a parametric bench and cut it into slices
for digital fabrication.
Build a laser-cut prototype.\\

\noindent \textbf{Families of form}
Use map algebra and visual programming to generatively design
erosion control features to restore your degraded study landscape.
Catalyze topographic changes 
with algorithmically generated interventions
to restore the landscape to a dynamic equilibrium.  
Digitally fabricate models of your designs
and augment these with projected water flow and sediment flux.\\

\noindent \textbf{The Great Piece of Turf}
Create a 3D model and 3D rendering of
Albrecht Dürer's Great Piece of Turf.
Use particle systems to distribute 3D flowers, grasses,
and other ground cover across a block of soil.\\

\noindent \textbf{Gully restoration}
Map the existing vegetation and landforms
using automated classification algorithms.
Design, model, and render in 3D
a bioswale to restore this degraded landscape.
Produce beautiful, photorealistic 3D renderings  
of the existing and restored landscape.\\

% -------------------------------- Workshop -------------------------------- 
\section{Drone Workshop}
Conduct a topographic survey with an unmanned aerial system (UAS)
at Hilltop Arboretum. 
After a morning theory session, 
survey the arboretum grounds with a drone,
and then use stereophotogrammetry to generate a digital surface model.

% -------------------------------- Software -------------------------------- 
\section{Software}
\begin{multicols}{2}
\raggedright
GRASS GIS | \url{https://grass.osgeo.org/} \\
ArcGIS | \url{https://www.esri.com/} \\
Rhinoceros | \url{https://www.rhino3d.com/}\\
%RhinoTerrain | \url{http://www.rhinoterrain.com/}\\
%RhinoCAM | \url{https://mecsoft.com/rhinocam-software/}\\
Grasshopper | \url{http://grasshopper3d.com/}\\
Blender | \url{https://www.blender.org/}
\end{multicols}

% -------------------------------- Resources -------------------------------- 
\section{Resources}
Intro to GRASS GIS | \url{https://ncsu-geoforall-lab.github.io/grass-intro-workshop/}\\
Hydrology in GRASS GIS | \url{https://grasswiki.osgeo.org/wiki/Hydrological_Sciences}\\
Grasshopper Primer | \url{http://grasshopperprimer.com}\\
BlenderGIS tutorial | \url{https://github.com/ptabriz/ICC_2017_Workshop}

% -------------------------------- Certificate -------------------------------- 
\section{Graduate Certificate in GIS}
This course counts as an applied topics course for the 
Graduate Certificate in Geographic Information Science.
The Graduate Certificate in Geographic Information Science at LSU 
is a 12 credit hour standalone certificate. 
%with courses offered 
%in the Department of Geography and Anthropology, 
%College of Art and Design, 
%Department of Economics, 
%School of the Coast and Environment, 
%Department of Civil and Environmental Engineering, 
%and Department of Computer Science. 
For more information about the Graduate Certificate in GIS visit: 
\url{http://ga.lsu.edu/gis-certificate/}.

% -------------------------------- Grading -------------------------------- 
\section{Grading}
%
\begin{table}[H]
%\small
\begin{tabular}{l r @{\hskip 2cm} l @{\hskip 0.5cm} l}
%\begin{tabular}{l l}
%
Gully monitoring & 10\% & Families of form & 25\% \\
CNC Machining & 10\% & The Great Piece of Turf & 10\% \\
Physical simulation & 10\% & Gully restoration & 25\% \\
Parametric bench & 10\% \\
%
\end{tabular}
\end{table}

\clearpage

% -------------------------------- Terminology -------------------------------- 
\section{Terminology}
\begin{multicols}{2}
\raggedright
\small
%
\textbf{Digital culture}
\begin{itemize}
\item Mass customization
\item Generative design
\item Parametric design
\item Performative design
\item Algorithm
%\item Scripting
\end{itemize}

\textbf{Spatial data}
\begin{itemize}
\item Raster \& Vector
\item Array
\item Point cloud
\item Mesh
\item Triangulated irregular network (TIN)
\item Discrete \& continuous data
\item Plain text
\item Comma separated values (CSV)
\item Integer \& floating point numbers
%\item Quadtree
\item Non-uniform rational basis spline (NURBS)
\end{itemize}

\textbf{Geospatial}
\begin{itemize}
\item Geographic information system (GIS)
\item Digital terrain model (DTM)
\item Digital elevation model (DEM)
\item Digital surface model (DSM)
\item Lidar
\item Delaunay triangulation
\item Interpolation
\item Bilinear interpolation
\item Nearest neighbors
\item Regularized spline with tension (RST)
\item Map algebra
\item Null value
\item Least cost path (LCP)
\item Resampling
\item Image classification
\end{itemize}

\textbf{3D rendering}
\begin{itemize}
\item Ray tracing
\item Diffuse shading
\item Texture map
\item Particle system
\item Head mounted display (HMD)
\item Cave automatic virtual environment (CAVE)
\end{itemize}

\textbf{Digital fabrication}
\begin{itemize}
\item 3D printing
\item Computer numeric control (CNC)
\item Collet \& Bit
\item High density urethane (HDU)
\item Medium density fiberboard (MDF)
\end{itemize}

\textbf{Geomorphology}
\begin{itemize}
\item Watershed
\item Single flow direction (SFD/D8)
\item Multiple flow direction (MFD)
\item Revised Universal Soil Loss Equation (RUSLE)
\item Unit Stream Power-based Erosion Deposition (USPED)
\item Simulated Water Erosion (SIMWE)
\item R-factor
\item Mannings
\item Sediment mass density
\item Gully
\item Knickpoint
\end{itemize}
%
\end{multicols}

\clearpage

% -------------------------------- Readings -------------------------------- 
\section{Readings}
\vspace*{0.5cm}
\nocite{*}
\setlength\bibitemsep{0.65\baselineskip}
\printbibliography[heading=none]
\clearpage
% -------------------------------- Policies -------------------------------- 
\section{Policies}

\noindent \textbf{Time Commitment Expectations}
LSU's general policy states that for each credit hour, you (the student) should plan to
spend at least two hours working on course related activities outside of class. Since this course is for three credit hours, you should expect to spend a minimum of six hours outside of class each week working on assignments for this course. For more information see: 
\url{http://catalog.lsu.edu/content.php?catoid=12&navoid=822}.\\

\noindent \textbf{LSU student code of conduct}
The LSU student code of conduct explains student rights, excused absences, and what is expected of student behavior. Students are expected to understand this code:  \url{http://students.lsu.edu/saa/students/code}.\\ %Any violations of the LSU student code will be duly reported to the Dean of Students.\\

\noindent \textbf{Disability Code}
The University is committed to making reasonable efforts to assist individuals with disabilities in
their efforts to avail themselves of services and programs offered by the University. To this end,
Louisiana State University will provide reasonable accommodations for persons with
documented qualifying disabilities. If you have a disability and feel you need accommodations in
this course, you must present a letter to me from Disability Services in 115 Johnston Hall,
indicating the existence of a disability and the suggested accommodations.\\

\noindent \textbf{Academic Integrity}
According to section 10.1 of the LSU Code of Student Conduct, ``A student may be charged with Academic Misconduct'' for a variety of offenses, including the following: unauthorized copying, collusion, or collaboration; ``falsifying'' data or citations; ``assisting someone in the commission or attempted commission of an offense''; and plagiarism, which is defined in section 10.1.H as a ``lack of appropriate citation, or the unacknowledged inclusion of someone else's words, structure, ideas, or data; failure to identify a source, or the submission of essentially the same work for two assignments without permission of the instructor(s).''\\

\noindent \textbf{Plagiarism and Citation Method}
Plagiarism is the ``lack of appropriate citation, or the unacknowledged inclusion of someone else's words, structure, ideas, or data; failure to identify a source, or the submission of essentially the same work for two assignments without permission of the instructor(s)'' (Sec. 10.1.H of the LSU Code of Student Conduct). As a student at LSU, it is your responsibility to refrain from plagiarizing the academic property of another and to utilize appropriate citation method for all coursework. In this class, it is recommended that you use Chicago Style author-date citations. Ignorance of the citation method is not an excuse for academic misconduct.

\end{document}
