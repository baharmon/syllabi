%!TEX TS-program = xelatex
%!TEX encoding = UTF-8 Unicode

%%%  Syllabus template for use with style files at http://kjhealy.github.com/latex-custom-kjh
%%%  Kieran Healy

\documentclass[11pt,article,oneside]{memoir}

% packages
\usepackage{org-preamble-xelatex}
\usepackage{wallpaper}
\usepackage{xcolor}
\usepackage{multicol}
\usepackage{enumitem}
\setlist[itemize]{leftmargin=*}
\usepackage{tikz}
\usepackage{tikzpagenodes} 

\AtBeginBibliography{\small}

% Definitions
\def\myauthor{Author}
\def\mytitle{Title}
\def\mycopyright{\myauthor}
\def\mykeywords{}
\def\mybibliostyle{plain}
\def\mybibliocommand{}
\def\mysubtitle{}
\def\myaffiliation{Louisiana State University}
\def\myaddress{309 Design}
\def\myemail{baharmon@lsu.edu} 
\def\myweb{https://baharmon.github.io/}
\def\myphone{919.622.8414}
\def\myversion{}
\def\myrevision{}
\def\myaffiliation{\ \\Louisiana State University}
\def\myauthor{Brendan Harmon}
\def\mykeywords{Landscape Architecture, Syllabus, Graduate}
\def\mysubtitle{Syllabus}
\def\mytitle{ \includegraphics[width=6cm]{../logos/lsu_art_design_logo.pdf} \\[0.1cm] {\normalfont \normalsize LA 7032 |} \Large Computational Design} 

% color
\makeatletter
\newcommand{\globalcolor}[1]{%
  \color{#1}\global\let\default@color\current@color
}
\makeatother

% begin
\begin{document}

\setlength\bibitemsep{0.5em}

% fonts
\defaultfontfeatures{}
\defaultfontfeatures{Scale=MatchLowercase}
\setmainfont{IBM Plex Sans}
\setmonofont[Scale=0.8]{IBM Plex Mono}

\def\ind{\hangindent=1 true cm\hangafter=1 \noindent}
\def\labelitemi{$\cdot$}
\chapterstyle{article-4-sans}  

\title{\LARGE \mytitle}
\author{\Large\myauthor \newline \footnotesize\texttt{\noindent\myemail}}
\date{Spring 2024. \newline Tuesday \& Thursday \newline 13:00--16:00 \newline Design 308}
\published{\,}

% -------------------------------- COVER PAGE -------------------------------- 

\pagenumbering{gobble}
\globalcolor{black}
\begin{tikzpicture}[remember picture,overlay]
\fill[white] ([xshift=-1cm,yshift=3.5cm]current page text area.west) rectangle (7.25,2);
\end{tikzpicture}
\vspace*{-10em}
\maketitle
\ThisCenterWallPaper{1.1}{../images/rosedown-landscape.jpg}
\clearpage


% -------------------------------- DESCRIPTION -------------------------------- 

\pagenumbering{arabic}
\globalcolor{black}

\vspace*{-10em}
\maketitle

\section{Course Description}

This course is an introduction to 
computational design for landscape architects.
In this course you will learn how to 
algorithmically model landscapes,
numerically simulate physical processes 
like the flow of water and sediment,
and digitally fabricate landforms.  
You will conduct surveys with drones and terrestrial lidar,
use point clouds to model and visualize terrain and planting,
parametrically model variations on landscapes
using visual programming, 
and computer numerical control (CNC) mill landforms. 
In this course you will 
learn creative approaches to computational design 
and apply emerging technologies to landscape architecture.
\\

\noindent
\textbf{Keywords}
\begin{multicols}{3}
\raggedright
\small
\begin{itemize}
\item Drones
\item Lidar
\item Laser scanning
\item Point clouds
\item Computational design
\item Visual programming
\item Digital fabrication
\item Robotics
\item Autonomous construction
\end{itemize}
\end{multicols}

% -------------------------------- SCHEDULE -------------------------------- 
\section{Schedule}

\begin{table}[H]
\begin{tabular}{l @{\hskip 2.36cm} l @{\hskip 2.6cm} l}
\textbf{Point Clouds} & \textbf{Modeling} & \textbf{Robotics}\\
\end{tabular}
\end{table}
%
\vspace*{-1em}
%
\begin{table}[H]
\small
\begin{tabular}{l l l l l l}
\small
\textbf{1} & Computational design \hspace{1em}  & \textbf{6} & Terrain  \hspace{6em}  & \textbf{11} & Robotic Operations\\
\textbf{2} & Drone lidar & \textbf{7} & Earthworks & \textbf{12} & Robotic Wall\\
\textbf{3} & Laser scanning & \textbf{8} & Hydrology & \textbf{13} & Robotic Construction\\
\textbf{4} & Stochasticity  & \textbf{9} & CNC Machining & \textbf{14} & Seminar\\
\textbf{5} & Noise & \textbf{10} & 3D Printing & \textbf{15} & Review\\

\end{tabular}
\end{table}

\clearpage

% -------------------------------- Logistics -------------------------------- 
\section{Logistics}

During our regularly scheduled class period
on Tuesdays and Thursdays from 1:00-4:00 pm,
we will meet in person, while also posting
on our Discord server at \url{https://discord.gg/zMcTnmP3KU}.
The discord server will be used for posting
announcements, student work in progress, 
reading responses, projects, and troubleshooting. 
All course content including tutorials, lectures, and datasets
will be published on the course website at:
\url{https://baharmon.github.io/generative-landscapes}.\\

\noindent
Course website | \url{https://baharmon.github.io/generative-landscapes}\\
Discord | \url{https://discord.gg/zMcTnmP3KU}\\
Youtube | \url{https://www.youtube.com/@baharmon}

% -------------------------------- Projects -------------------------------- 

\section{Projects}

\paragraph{Point Cloud Landscape}
Use point cloud modeling
to design and visualize 
a new garden at Hilltop Arboretum. 
Use a library of laser scanned plants 
to model your planting
and use volumetric modeling
to re-grade the terrain.
Experiment with 
algorithmically distributed plants,
procedurally generated terrain,
and parametric landscape elements. 
Render your landscape
as a point cloud. 
\textbf{Due:} Week 5

\paragraph{3D Printed Landscape}
Voxelize and 3D print 
your point cloud landscape.
Cut transects
through your model
to reveal the structure
of the landscape.
\textbf{Due:} Week 10

\paragraph{Robotic Wall}
Program a robotic arm to assemble 
a parametrically modeled wall
from bricks. 
Use an algorithm to procedurally model a brick wall,
program a robotic operation for assembling the wall,
and then use an industrial robotic arm
to assemble the wall
with high density foam blocks.
\textbf{Due:} Week 15

% -------------------------------- Seminar -------------------------------- 
\section{Seminar}

A discussion of 
how have digital tools and processes transformed 
the practice of landscape architecture
and how do you think they will shape 
the future of the discipline.

\nocite{*} \printbibliography[keyword=seminar, heading=none]


% -------------------------------- Grading -------------------------------- 
\section{Grading}
%
\begin{table}[H]
\begin{tabular}{l r @{\hskip 2cm} l @{\hskip 0.5cm} l}
Point Cloud Landscape & 20\% &  Assignments & 30\% \\
3D Printed Landscape & 20\% & Seminar & 5\%\\
Robotic Wall & 20\% & Course Portfolio & 5\% \\
\end{tabular}
\end{table}

% -------------------------------- Software -------------------------------- 
\section{Software}

Rhinoceros | \url{https://www.rhino3d.com/}\\
CloudCompare | \url{https://www.danielgm.net/cc/}\\
GRASS GIS | \url{https://grass.osgeo.org/}\\
QGIS | \url{https://www.qgis.org/}\\
%Python | \url{https://www.python.org/}\\

\section{Plugins}
Snapping Gecko | \url{https://www.food4rhino.com/app/snappinggecko}\\
Bitmap+ | \url{https://www.food4rhino.com/en/app/bitmap}\\
Docofossor | \url{https://www.food4rhino.com/app/docofossor}\\
Dendro | \url{https://www.food4rhino.com/en/app/dendro}\\
RhinoCAM | \url{https://mecsoft.com/rhinocam-software/}\\

% -------------------------------- Resources -------------------------------- 
\section{Resources}

Grasshopper Basics | \url{https://vimeo.com/channels/basicgh}\\
Grasshopper Primer | \url{http://grasshopperprimer.com}\\
Hydrology in GRASS GIS | \url{https://grasswiki.osgeo.org/wiki/Hydrological_Sciences}\\

% -------------------------------- Readings -------------------------------- 
\section{Required Readings}
\vspace*{0.5cm}
\nocite{*}
\setlength\bibitemsep{0.65\baselineskip}
\printbibliography[keyword=required, heading=none]

% -------------------------------- Readings -------------------------------- 
\section{Recommended Readings}
\vspace*{0.5cm}
\nocite{*}
\setlength\bibitemsep{0.65\baselineskip}
\printbibliography[keyword=recommended, heading=none]

% -------------------------------- Policies -------------------------------- 
\section{Policies}

\paragraph{Time Commitment Expectations}
LSU's general policy states that for each credit hour, you (the student) should plan to
spend at least two hours working on course related activities outside of class. Since this course is for three credit hours, you should expect to spend a minimum of six hours outside of class each week working on assignments for this course. For more information see: 
\url{http://catalog.lsu.edu/content.php?catoid=12&navoid=822}.

\paragraph{LSU student code of conduct}
The LSU student code of conduct explains student rights, excused absences, and what is expected of student behavior. Students are expected to understand this code:  \url{http://students.lsu.edu/saa/students/code}.
%Any violations of the LSU student code will be duly reported to the Dean of Students.

\paragraph{Disability Code}
The University is committed to making reasonable efforts to assist individuals with disabilities in
their efforts to avail themselves of services and programs offered by the University. To this end,
Louisiana State University will provide reasonable accommodations for persons with
documented qualifying disabilities. If you have a disability and feel you need accommodations in
this course, you must present a letter to me from Disability Services in 115 Johnston Hall,
indicating the existence of a disability and the suggested accommodations.

\paragraph{Academic Integrity}
According to section 10.1 of the LSU Code of Student Conduct, ``A student may be charged with Academic Misconduct'' for a variety of offenses, including the following: unauthorized copying, collusion, or collaboration; ``falsifying'' data or citations; ``assisting someone in the commission or attempted commission of an offense''; and plagiarism, which is defined in section 10.1.H as a ``lack of appropriate citation, or the unacknowledged inclusion of someone else's words, structure, ideas, or data; failure to identify a source, or the submission of essentially the same work for two assignments without permission of the instructor(s).''

\paragraph{Plagiarism and Citation Method}
Plagiarism is the ``lack of appropriate citation, or the unacknowledged inclusion of someone else's words, structure, ideas, or data; failure to identify a source, or the submission of essentially the same work for two assignments without permission of the instructor(s)'' (Sec. 10.1.H of the LSU Code of Student Conduct). As a student at LSU, it is your responsibility to refrain from plagiarizing the academic property of another and to utilize appropriate citation method for all coursework. In this class, it is recommended that you use Chicago Style author-date citations. Ignorance of the citation method is not an excuse for academic misconduct.

\paragraph{Accreditation Expectations}
As an accredited Landscape Architecture program
LSU's Robert Reich School of Landscape Architecture (RRSLA) 
must meet the accreditation requirements 
as stated by the Landscape Architectural Accreditation
Board (LAAB) to ensure RRSLA is meeting the expectations of the field. 
The LAAB requires programs to provide digital copies.
%of student work as part of this process.
%Students in this course will be expected 
%to comply with the following requirements
%as 5\% of their course grade: 
(1) Students must provide a course portfolio
with work samples specified by the instructor 
before the end of the grading period. 
(2) Each student's course portfolio must be saved as 
a single, high resolution PDF file with multiple pages. 
(3) Files must follow the naming convention
established by the school: department-coursenumber-semesteryear-username.pdf.
Example: LA7032-S2024 -baharmon.pdf.

\end{document}
