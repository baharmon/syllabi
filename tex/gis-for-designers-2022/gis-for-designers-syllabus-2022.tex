%!TEX TS-program = xelatex
%!TEX encoding = UTF-8 Unicode

\documentclass[11pt,article,oneside]{memoir}

% packages
\usepackage{org-preamble-xelatex}
\usepackage{wallpaper}
\usepackage{xcolor}
\usepackage{multicol}
\usepackage{enumitem}
\setlist[itemize]{leftmargin=*}

\AtBeginBibliography{\small}

% Definitions
\def\myauthor{Author}
\def\mytitle{Title}
\def\mycopyright{\myauthor}
\def\mykeywords{}
\def\mybibliostyle{plain}
\def\mybibliocommand{}
\def\mysubtitle{}
\def\myaffiliation{Louisiana State University}
\def\myaddress{309 Design}
\def\myemail{baharmon@lsu.edu} 
\def\myweb{https://baharmon.github.io/}
\def\myphone{919.622.8414}
\def\myversion{}
\def\myrevision{}
\def\myaffiliation{\ \\Louisiana State University}
\def\myauthor{Brendan Harmon}
\def\mykeywords{Landscape Architecture, Syllabus, Graduate, Undergraduate}
\def\mysubtitle{Syllabus}
\def\mytitle{ \includegraphics[width=6cm]{../logos/lsu_art_design_logo.pdf} \\[0.1cm] {\normalfont \normalsize LA 4201 \& 7075 |} \Large GIS for Designers} 

% color
\makeatletter
\newcommand{\globalcolor}[1]{%
  \color{#1}\global\let\default@color\current@color
}
\makeatother

% begin
\begin{document}

\setlength\bibitemsep{0.5em}

% fonts
\defaultfontfeatures{}
\defaultfontfeatures{Scale=MatchLowercase}         
\setmainfont[Scale=1, Path = ../fonts/lato/,BoldItalicFont=Lato-BolIta,BoldFont=Lato-Bol,ItalicFont=Lato-RegIta]{Lato-Reg}
\setmonofont[Mapping=tex-text,Scale=0.8,Path = ../fonts/inconsolata/]{i}

\def\ind{\hangindent=1 true cm\hangafter=1 \noindent}
\def\labelitemi{$\cdot$}
\chapterstyle{article-4-sans}  

\title{\LARGE \mytitle}
\author{\Large\myauthor \newline \footnotesize\texttt{\noindent\myemail}}
\date{Fall 2022 \newline Monday, Wednesday, \& Friday  \newline 9:30am--11:20pm \newline Design 301 }
\published{\,}

% -------------------------------- COVER PAGE -------------------------------- 

\pagenumbering{gobble}
\globalcolor{black}
\vspace*{-10em}
\maketitle
\ThisCenterWallPaper{1}{../images/new-orleans-portrait.png}
\clearpage


% -------------------------------- DESCRIPTION -------------------------------- 

\pagenumbering{arabic}
\globalcolor{black}
\vspace*{-10em}
\maketitle

\section{Course Description}

This course is an introduction to 
Geographic Information Systems (GIS) and Science (GISc) for designers. 
Learn about the history, theory, methods, and applications of GIS.
Acquire, map, model, and analyze spatial and temporal data.
Make beautiful maps and digitally fabricated models from spatiotemporal data.

\vspace*{0.5em}

% -------------------------------- SCHEDULE -------------------------------- 
\section{Topics}

\begin{table}[H]
\begin{tabular}{l l @{\hskip 0.5cm} l l @{\hskip 0.5cm} l l}
\small
\textbf{1} & Geodesy & \textbf{6} & Map Algebra & \textbf{11} & Lidar\\
\textbf{2} & Intro to GIS & \textbf{7} & Programming & \textbf{12} & Demographics\\
\textbf{3} & Global Data & \textbf{8} & Hydrology & \textbf{13} & Map Overlays\\
\textbf{4} & Urban Data & \textbf{9} & Visibility \& Solar & \textbf{14} & Critical Cartography\\
\textbf{5} & Terrain Analysis & \textbf{10} & Cartography & \textbf{15} & Map Exhibition\\
\end{tabular}
\end{table}

\vspace*{1em}

\includegraphics[width=\textwidth]{../images/new-orleans-landscape-3.png}

\clearpage

% -------------------------------- Online -------------------------------- 
\section{Logistics}

During our regularly scheduled class period
we will meet in person, while also posting
on our Discord server at \url{https://discord.gg/rUpX6jRBzs}. % update server invite 
Please bring your computer to class, follow along, 
and post your progress live on our Discord server. 
The discord server will be used for posting
announcements, classwork, homework, projects, and troubleshooting. 
Course content including tutorials, lectures, and datasets
will be published on the course website at:
\url{http://baharmon.github.io/gis-for-designers}.\\

\noindent
Course website | \url{http://baharmon.github.io/gis-for-designers}\\
Youtube | \url{https://www.youtube.com/channel/UCmGEF6Bf1SO92oLQoGCPDTw}\\
Discord | \url{https://discord.gg/rUpX6jRBzs}\\ % update server invite 

% -------------------------------- Projects -------------------------------- 

\section{Projects}
Map spatial and temporal data
at global, city, and site scales.
Make beautiful maps that 
clearly, legibly represent the data,
express your message, and
follow cartographic conventions.
Legends, scale bars, and north arrows are required.
Upload your work to the course drive.
Exhibit your collected work at the end of the semester.\\
%on \emph{12/04/2022}.\\

\noindent \textbf{City maps}
Create a thematic map for a city of your choice. 
Possible topics include the built environment, 
cultural events, cultural and historic places, 
socioeconomic conditions, public health, crime, education, 
hydrology, terrain, levees, flooding, etc.\\

\noindent \textbf{Suitability map}
Use map overlay analysis 
to develop a suitability map
about a topic of your choice.
Create a diagram illustrating 
the logic of your analysis.\\

\noindent \textbf{3D printed city}
3D print Manhattan in New York City from lidar data. 
As a class use a tiling scheme 
to divide the lidar data into smaller tiles for printing.
Each of you will 3D print a tile.\\
%\emph{Draft due: 11/23/2022}\\

\noindent \textbf{Portfolio}
Collect your work in a course portfolio 
for the school's accreditation archive.\\
%\emph{Due: 12/11/2022}\\

\clearpage

% -------------------------------- Software -------------------------------- 
\section{Software}
%\begin{multicols}{2}
%\raggedright
QGIS | \url{https://qgis.org/}\\
GRASS GIS | \url{https://grass.osgeo.org/}\\
ArcGIS Pro | \url{https://www.esri.com/}\\
Rhinoceros | \url{https://www.rhino3d.com/}\\
RhinoTerrain | \url{http://www.rhinoterrain.com/}\\
%Thea Render | \url{https://www.thearender.com/}\\
%\end{multicols}

% -------------------------------- Data -------------------------------- 
\section{Datasets}
Natural Earth Dataset for GRASS GIS | \url{https://zenodo.org/record/3762852}\\
Governor's Island Dataset for GRASS GIS | \url{https://zenodo.org/record/3940780}\\
Governor's Island Dataset for QGIS | \url{https://zenodo.org/record/4044664}\\
% Add raw dataset

% -------------------------------- Resources -------------------------------- 
\section{Resources}
Geospatial data sources | \url{http://baharmon.github.io/data}\\
Intro to GRASS GIS | \url{https://ncsu-geoforall-lab.github.io/grass-intro-workshop/}\\
GRASS GIS tutorials | \url{https://grass.osgeo.org/documentation/tutorials/}\\
QGIS training material | \url{https://www.qgis.org/en/site/forusers/trainingmaterial/}\\
ArcGIS training | \url{https://www.esri.com/training/}\\
Learn ArcGIS | \url{https://learn.arcgis.com/}\\

% -------------------------------- Grading -------------------------------- 
\section{Grading}
\vspace*{-0.4cm}
\begin{table}[H]
\begin{tabular}{@{}l r @{\hskip 2cm} l @{\hskip 0.5cm} l}
City map & 20\% & Homework & 35\% \\
Suitability map & 20\% & Portfolio & 5\% \\
3D printing & 20\% \\
\end{tabular}
\end{table}

\clearpage

% -------------------------------- Readings -------------------------------- 
\section{Readings}
\vspace*{0.5cm}
\nocite{*}
\setlength\bibitemsep{0.65\baselineskip}
\printbibliography[heading=none]

% -------------------------------- Graduate Certificate -------------------------------- 
\section{Graduate Certificate in GIS}
This course counts as an applied topics course for the 
Graduate Certificate in Geographic Information Science.
The Graduate Certificate in Geographic Information Science at LSU 
is a 12 credit hour standalone certificate
with courses offered 
in the Department of Geography and Anthropology, 
College of Art and Design, 
Department of Economics, 
School of the Coast and Environment, 
Department of Civil and Environmental Engineering, 
and Department of Computer Science. 
For more information about the Graduate Certificate in GIS visit: 
\url{http://ga.lsu.edu/gis-certificate/}.

% -------------------------------- CI Certificate -------------------------------- 

\section{Communication-Intensive Certification}
This is a certified Communication-Intensive (C-I) course which meets all of the requirements set forth by LSU's Communication across the Curriculum program, including
instruction and assignments emphasizing informal and formal modes;
teaching of discipline-specific communication techniques;
use of feedback loops for learning;
40\% of the course grade rooted in communication-based work; 
and practice of ethical and professional work standards.
Students interested in pursuing the LSU Communicator Certificate and/or the LSU Distinguished Communicator Medal may use this C-I course for credit. For more information about these student recognition programs, visit \url{www.cxc.lsu.edu}.\\

% -------------------------------- Policies -------------------------------- 
\section{Policies}

\noindent \textbf{Accreditation Expectations}
As an accredited Landscape Architecture program
LSU's Robert Reich School of Landscape Architecture (RRSLA) 
must meet the accreditation requirements 
as stated by the Landscape Architectural Accreditation
Board (LAAB) to ensure RRSLA is meeting the expectations of the field. 
The LAAB requires programs to provide digital copies 
of student work as part of this process.
Students in this course will be expected 
to comply with the following requirements
as 5\% of their course grade: 
(1) Students must provide a course portfolio
with work samples specified by the instructor 
before the end of the grading period. 
(2) Each student's course portfolio must be saved as 
a single, high resolution PDF file with multiple pages. 
(3) Files must follow the naming convention
established by the school: department-coursenumber-semesteryear-username.pdf.
Example: LA7075-F2022-baharmon.pdf.\\

\noindent \textbf{Late Work Policy}
There is a one week grace period for late class work. 
After the grace period, 
the grade for the assignment will be lowered by
a letter grade (i.e.~$10$ points) per week late.\\

\noindent \textbf{Attendance Policy}
When students have valid reasons for absence, they are responsible for providing reasonable advance notification and appropriate documentation of the reason for the absence and for making up examinations, obtaining lecture notes, and compensating for what may have been missed. Valid reasons that must be documented include: illness, serious family emergency, special curricular requirements such as field trips, court-imposed legal obligations such as subpoenas or jury duty, military obligations, serious weather conditions, religious observances, official participation in varsity athletic competitions, or university musical events. Absences without valid reasons are limited to three per term. Beyond these limits, each unexcused absence will lower the final course grade by one letter grade increment (i.e.~$3.\overline{3}$ points).\\

\noindent \textbf{Time Commitment Expectations}
LSU's general policy states that for each credit hour, you (the student) should plan to
spend at least two hours working on course related activities outside of class. Since this course is for three credit hours, you should expect to spend a minimum of six hours outside of class each week working on assignments for this course. For more information see: 
\url{http://catalog.lsu.edu/content.php?catoid=12&navoid=822}.\\

\noindent \textbf{LSU student code of conduct}
The LSU student code of conduct explains student rights, excused absences, and what is expected of student behavior. Students are expected to understand this code:  \url{http://students.lsu.edu/saa/students/code}.\\ %Any violations of the LSU student code will be duly reported to the Dean of Students.\\

\noindent \textbf{Disability Code}
The University is committed to making reasonable efforts to assist individuals with disabilities in
their efforts to avail themselves of services and programs offered by the University. To this end,
Louisiana State University will provide reasonable accommodations for persons with
documented qualifying disabilities. If you have a disability and feel you need accommodations in
this course, you must present a letter to me from Disability Services in 115 Johnston Hall,
indicating the existence of a disability and the suggested accommodations.\\

\noindent \textbf{Academic Integrity}
According to section 10.1 of the LSU Code of Student Conduct, ``A student may be charged with Academic Misconduct'' for a variety of offenses, including the following: unauthorized copying, collusion, or collaboration; ``falsifying'' data or citations; ``assisting someone in the commission or attempted commission of an offense''; and plagiarism, which is defined in section 10.1.H as a ``lack of appropriate citation, or the unacknowledged inclusion of someone else's words, structure, ideas, or data; failure to identify a source, or the submission of essentially the same work for two assignments without permission of the instructor(s).''\\

\noindent \textbf{Plagiarism and Citation Method}
Plagiarism is the ``lack of appropriate citation, or the unacknowledged inclusion of someone else's words, structure, ideas, or data; failure to identify a source, or the submission of essentially the same work for two assignments without permission of the instructor(s)'' (Sec. 10.1.H of the LSU Code of Student Conduct). As a student at LSU, it is your responsibility to refrain from plagiarizing the academic property of another and to utilize appropriate citation method for all coursework. In this class, it is recommended that you use Chicago Style author-date citations. Ignorance of the citation method is not an excuse for academic misconduct.\\

\end{document}
