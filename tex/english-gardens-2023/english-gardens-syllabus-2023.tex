%!TEX TS-program = xelatex
%!TEX encoding = UTF-8 Unicode

\documentclass[11pt,article,oneside]{memoir}

% packages
\usepackage{org-preamble-xelatex}
\usepackage{wallpaper}
\usepackage{xcolor}
\usepackage{multicol}
\usepackage{enumitem}
\setlist[itemize]{leftmargin=*}

\AtBeginBibliography{\small}

% Definitions
\def\myauthor{Author}
\def\mytitle{Title}
\def\mycopyright{\myauthor}
\def\mykeywords{}
\def\mybibliostyle{plain}
\def\mybibliocommand{}
\def\mysubtitle{}
\def\myaffiliation{Louisiana State University}
\def\myaddress{309 Design}
\def\myemail{baharmon@lsu.edu} 
\def\myweb{https://baharmon.github.io/}
\def\myphone{919.622.8414}
\def\myversion{}
\def\myrevision{}
\def\myaffiliation{\ \\Louisiana State University}
\def\myauthor{Brendan Harmon}
\def\mykeywords{Honors, Syllabus, Undergraduate}
\def\mysubtitle{Syllabus}
\def\mytitle{ \includegraphics[width=6cm]{../logos/lsu_art_design_logo.pdf} \\[0.1cm] {\normalfont \normalsize HNRS 2021 |} \Large English Gardens} 

% color
\makeatletter
\newcommand{\globalcolor}[1]{%
  \color{#1}\global\let\default@color\current@color
}
\makeatother

% begin
\begin{document}

\setlength\bibitemsep{0.5em}

% fonts
\defaultfontfeatures{}
\defaultfontfeatures{Scale=MatchLowercase}         
\setmainfont[Scale=1, Path = ../fonts/lato/,BoldItalicFont=Lato-BolIta,BoldFont=Lato-Bol,ItalicFont=Lato-RegIta]{Lato-Reg}
\setmonofont[Mapping=tex-text,Scale=0.8,Path = ../fonts/inconsolata/]{i}

\def\ind{\hangindent=1 true cm\hangafter=1 \noindent}
\def\labelitemi{$\cdot$}
\chapterstyle{article-4-sans}  

\title{\LARGE \mytitle}
\author{\Large\myauthor \newline \footnotesize\texttt{\noindent\myemail}}
\date{\textbf{Ogden Honors in Oxford} \newline Summer 2023}
\published{\,}

% -------------------------------- COVER PAGE -------------------------------- 

%\pagenumbering{gobble}
%\globalcolor{black}
%\vspace*{-10em}
%\maketitle
%\ThisCenterWallPaper{1}{../images/}
%\clearpage


% -------------------------------- DESCRIPTION -------------------------------- 

\pagenumbering{arabic}
\globalcolor{black}
\vspace*{-10em}
\maketitle

\section{Course Description}

This course will study the history and aesthetics of English gardens. Through art, literature, film, and landscape, we will examine how English conceptions of nature have evolved. By studying and visiting iconic gardens, we will explore themes such as the control of nature, the picturesque, the sublime, botanical imperialism, orientalism, and sustainability that have shaped the English landscape. We will study the artistic, cultural, historical, economic, and philosophical context for these gardens, with the aim of understanding the evolving aesthetics of landscape -- i.e.~why they were made, what they were meant to express, and how they are experienced today. This study abroad course  will be based in St. Hilda's College at the University of Oxford and will involve extensive travels throughout the Cotswolds region. We will learn about key figures in English landscape design, visit iconic gardens in the Cotswolds, and take a weekend trip to London.  For each topic covered in the course there will be a lecture, tour, and discussion. 

\vspace*{0.5em}

% -------------------------------- SCHEDULE -------------------------------- 
\section{Schedule}

\begin{table}[H]
\begin{tabular}{l l @{\hskip 0.5cm} l l @{\hskip 0.5cm} l l}
\small
& \textbf{Picturesque} & & \textbf{Arts \& Crafts} & & \textbf{Contemporary} \\
\textbf{1} & Oxford & \textbf{9} & Vernacular & \textbf{16} & Contemporary\\
\textbf{2} & William Kent & \textbf{9} & Botanical Imperialism & \textbf{16} & Charles Jencks\\
\textbf{3} & Capability Brown & \textbf{10} & Gertrude Jekyll & \textbf{17} & Piet Oudolf\\
\textbf{4} & Humphry Repton  & \textbf{11} & Sackville-West & \textbf{18} & Monty Don\\
\textbf{5} & Cotswolds & \textbf{12} & London  & \textbf{19} & Cotswolds\\
\textbf{6} & -- & \textbf{13} & London & \textbf{20} & --\\
\textbf{7} & -- & \textbf{14} & London  & \textbf{21} & --\\
\end{tabular}
\end{table}

\clearpage

% -------------------------------- Topics -------------------------------- 

\section{Picturesque}

\vspace*{0.em}

\noindent \textbf{1} 
\enspace
\textbf{Landscapes of Oxford}
\\
\noindent
Learn about the history of Oxford 
and how the tension between city and university
has shaped its unique spaces.
%
Take a walking tour of Oxford
to discover its landscapes by walking 
along its bustling city streets,
by its commons and canals, 
and through its college parks and quadrangles and parks
hidden behind walls of Cotswold's stone.
\\

\noindent \textbf{2} 
\enspace
\textbf{William Kent}
\\
\noindent
...
\\

% Readings

\noindent \textbf{3} 
\enspace
\textbf{Capability Brown}
\\
\noindent
...
\\

\noindent \textbf{4} 
\enspace
\textbf{Humphry Repton}
\\
\noindent
...
\\

\noindent \textbf{5} 
\enspace
\textbf{Cotswolds}
\\
\noindent
...
\\

\section{Arts \& Crafts}

\noindent \textbf{8} 
\enspace
\textbf{Vernacular Landscapes}
\\
\noindent
...
% punting on the Isis
\\

\noindent \textbf{9} 
\enspace
\textbf{Botanical Imperialism}
\\
\noindent
...
\\

\noindent \textbf{10} 
\enspace
\textbf{Gertrude Jekyll}
\\
\noindent
...
\\

\noindent \textbf{11} 
\enspace
\textbf{Vita Sackville-West}
\\
\noindent
...
\\

\noindent \textbf{12} 
\enspace
\textbf{London}
\\
\noindent
...
\\

\noindent \textbf{13} 
\enspace
\textbf{London}
\\
\noindent
...
\\

\noindent \textbf{14} 
\enspace
\textbf{London}
\\
\noindent
...
\\

\section{Contemporary}

\noindent \textbf{15} 
\enspace
\textbf{Contemporary}
\\
\noindent
...
\\

\noindent \textbf{16} 
\enspace
\textbf{Charles Jencks}
\\
\noindent
...
\\

\noindent \textbf{17} 
\enspace
\textbf{Piet Oudolf}
\\
\noindent
...
\\

\noindent \textbf{18} 
\enspace
\textbf{Monty Don}
\\
\noindent
...
\\

\noindent \textbf{19} 
\enspace
\textbf{Cotswolds}
\\
\noindent
...
\\

% -------------------------------- Projects -------------------------------- 

\section{Coursework}

\noindent \textbf{Participation}
\enspace
Participate in class activities 
and contribute meaningfully to class discussions.
%\textbf{Grade:} 33\%
\\

\noindent \textbf{Blog}
\enspace
Write a 250-500 word blog post about each garden you visit. 
Critically discuss concepts embodied in the landscape, 
reflect on your aesthetic experience,  
and include your own photographs.
%\textbf{Grade:} 33\%
\\

\noindent \textbf{Essay}
\enspace
On your final day in Oxford, 
write an essay on the meaning and aesthetics 
of English gardens.
%\textbf{Grade:} 34\%
\\

% -------------------------------- Grading -------------------------------- 

\section{Grading}
\vspace*{-0.4cm}
\begin{table}[H]
\begin{tabular}{@{}l r @{\hskip 2cm} l @{\hskip 0.5cm} l}
Participation & 33\%\\
Blog & 33\% \\
Essay & 34\% \\
\end{tabular}
\end{table}

\clearpage

% -------------------------------- Readings -------------------------------- 

\section{Readings}
\vspace*{0.5cm}
\nocite{*}
\setlength\bibitemsep{0.65\baselineskip}
\printbibliography[heading=none]

% -------------------------------- Policies -------------------------------- 

\section{Policies}

\noindent \textbf{Late Work Policy}
There is a one week grace period for late class work. 
After the grace period, 
the grade for the assignment will be lowered by
a letter grade (i.e.~$10$ points) per week late.\\

\noindent \textbf{Attendance Policy}
When students have valid reasons for absence, they are responsible for providing reasonable advance notification and appropriate documentation of the reason for the absence and for making up examinations, obtaining lecture notes, and compensating for what may have been missed. Valid reasons that must be documented include: illness, serious family emergency, special curricular requirements such as field trips, court-imposed legal obligations such as subpoenas or jury duty, military obligations, serious weather conditions, religious observances, official participation in varsity athletic competitions, or university musical events. Absences without valid reasons are limited to three per term. Beyond these limits, each unexcused absence will lower the final course grade by one letter grade increment (i.e.~$3.\overline{3}$ points).\\

\noindent \textbf{Time Commitment Expectations}
LSU's general policy states that for each credit hour, you (the student) should plan to
spend at least two hours working on course related activities outside of class. Since this course is for three credit hours, you should expect to spend a minimum of six hours outside of class each week working on assignments for this course. For more information see: 
\url{http://catalog.lsu.edu/content.php?catoid=12&navoid=822}.\\

\noindent \textbf{LSU student code of conduct}
The LSU student code of conduct explains student rights, excused absences, and what is expected of student behavior. Students are expected to understand this code:  \url{http://students.lsu.edu/saa/students/code}.\\ %Any violations of the LSU student code will be duly reported to the Dean of Students.\\

\noindent \textbf{Disability Code}
The University is committed to making reasonable efforts to assist individuals with disabilities in
their efforts to avail themselves of services and programs offered by the University. To this end,
Louisiana State University will provide reasonable accommodations for persons with
documented qualifying disabilities. If you have a disability and feel you need accommodations in
this course, you must present a letter to me from Disability Services in 115 Johnston Hall,
indicating the existence of a disability and the suggested accommodations.\\

\noindent \textbf{Academic Integrity}
According to section 10.1 of the LSU Code of Student Conduct, ``A student may be charged with Academic Misconduct'' for a variety of offenses, including the following: unauthorized copying, collusion, or collaboration; ``falsifying'' data or citations; ``assisting someone in the commission or attempted commission of an offense''; and plagiarism, which is defined in section 10.1.H as a ``lack of appropriate citation, or the unacknowledged inclusion of someone else's words, structure, ideas, or data; failure to identify a source, or the submission of essentially the same work for two assignments without permission of the instructor(s).''\\

\noindent \textbf{Plagiarism and Citation Method}
Plagiarism is the ``lack of appropriate citation, or the unacknowledged inclusion of someone else's words, structure, ideas, or data; failure to identify a source, or the submission of essentially the same work for two assignments without permission of the instructor(s)'' (Sec. 10.1.H of the LSU Code of Student Conduct). As a student at LSU, it is your responsibility to refrain from plagiarizing the academic property of another and to utilize appropriate citation method for all coursework. In this class, it is recommended that you use Chicago Style author-date citations. Ignorance of the citation method is not an excuse for academic misconduct.\\

\end{document}
