%!TEX TS-program = xelatex
%!TEX encoding = UTF-8 Unicode

\documentclass[11pt,article,oneside]{memoir}

% packages
\usepackage{org-preamble-xelatex}
\usepackage{wallpaper}
\usepackage{xcolor}
\usepackage{multicol}
\usepackage{enumitem}
\setlist[itemize]{leftmargin=*}
\usepackage{tikz}
\usepackage{tikzpagenodes} 

\AtBeginBibliography{\small}

% Definitions
\def\myauthor{Author}
\def\mytitle{Title}
\def\mycopyright{\myauthor}
\def\mykeywords{}
\def\mybibliostyle{plain}
\def\mybibliocommand{}
\def\mysubtitle{}
\def\myaffiliation{Louisiana State University}
\def\myaddress{309 Design}
\def\myemail{baharmon@lsu.edu \& fabianat@lsu.edu} 
\def\myweb{https://baharmon.github.io/}
\def\myphone{919.622.8414}
\def\myversion{}
\def\myrevision{}
\def\myaffiliation{\ \\Louisiana State University}
\def\myauthor{Brendan Harmon \& Fabiana Trindade da Silva}
\def\mykeywords{Landscape Architecture, Syllabus, Graduate}
\def\mysubtitle{Syllabus}
\def\mytitle{ \includegraphics[width=6cm]{../logos/lsu_art_design_logo.pdf} \\[0.1cm] {\normalfont \normalsize LA 5001 |} \Large The More Than Human City} 

% color
\makeatletter
\newcommand{\globalcolor}[1]{%
  \color{#1}\global\let\default@color\current@color
}
\makeatother

% begin
\begin{document}

\setlength\bibitemsep{0.5em}
\defaultfontfeatures{}
\defaultfontfeatures{Scale=MatchLowercase}
\setmainfont{IBM Plex Sans}
\setmonofont[Scale=0.8]{IBM Plex Mono}

\def\ind{\hangindent=1 true cm\hangafter=1 \noindent}
\def\labelitemi{$\cdot$}
\chapterstyle{article-4-sans}  

\title{\LARGE \mytitle}
\author{\Large\myauthor \newline {\footnotesize\texttt{\noindent\myemail}}}
\date{Fall 2024 | MWF | 13:30-17:30} 
\published{\,}

% -------------------------------- DESCRIPTION -------------------------------- 

\pagenumbering{arabic}
\globalcolor{black}

\vspace*{-10em}
\maketitle

\vspace*{-2em}
\section{Description}

How can a city be designed for more than just humans?  This studio explores how a city block can be designed to support the diversity of life and the agency of humans and non-humans alike. Our starting premise is that the city is an assemblage -- an entanglement of humans and non-humans, of biota and even inanimate things -- in which agency is distributed throughout. Our site will be Hudson Yards West atop the MTA West Side Yard. While the site is slated for development in phase two of Kohn Pedersen Fox's masterplan, we will design an alternative vision for the site, a vision that celebrates diversity, making places for humans and non-humans of all walks of life. This studio is a challenge to cast off binary thinking, be creative, think conceptually, and design differently. Announcements, readings, assignments, and student work will be posted on the course's Discord server at 
\url{https://discord.gg/r4m7bdKk3w}.

% -------------------------------- SCHEDULE -------------------------------- 

\section{Schedule}

\begin{table}[H]
\begin{tabular}{l l l l l l}
% 
\textbf{I} & \textbf{Research}  \hspace{2em}
	& \textbf{II} & \textbf{Planning} \hspace{2em} 
	& \textbf{III} & \textbf{Design} \\
\textbf{1} & Seminar \hspace{2em}
	& \textbf{6} & Urban Form I \hspace{2em} 
	& \textbf{11} & Site Design I \\
\textbf{2} & Kinship \hspace{2em}
	& \textbf{7} & Urban Form II \hspace{2em} 
	& \textbf{12} & Site Design II \\
\textbf{3} & Urban Systems \hspace{2em}
	& \textbf{8} & Landscape Plan I \hspace{2em} 
	& \textbf{13} & Site Design III \\
\textbf{4} & Habitat \hspace{2em}
	& \textbf{9} & Landscape Plan II \hspace{2em} 
	& \textbf{14} & Site Design IV \\
\textbf{5} & Fieldwork \hspace{2em}
	& \textbf{10} & Charrette \hspace{2em} 
	& \textbf{15} & Site Design V\\
\textbf{\ } & \ \hspace{2em}
	& \textbf{\ } & \  \hspace{2em} 
	& \textbf{\ } & \  \\
\textbf{\ } & Research Review \hspace{2em}
	& \textbf{\ } & Planning Review \hspace{2em} 
	& \textbf{\ } & Final Review \\
\end{tabular}
\end{table}

% -------------------------------- Grading -------------------------------- 

\section{Grading}

\begin{table}[H]
\begin{tabular}{l @{\hskip 1cm} l @{\hskip 1cm} l @{\hskip 1cm} l @{\hskip 1cm} l}
\textbf{32\%} \enspace Research
& \textbf{32\%} \enspace Planning
& \textbf{32\%} \enspace Design
& \textbf{4\%} \quad Portfolio \\
\end{tabular}
\end{table}

% -------------------------------- Weeks -------------------------------- 

\section{1 Seminar}
What would landscape architecture look like if non-human agency were acknowledged and the designer de-centered? The studio will begin with a seminar on more than human theories and their implications for design practice. For the seminar you should prepare at least three talking points -- e.g.~arguments, anecdotes, examples, images, questions, or provocations -- based on the readings. Post your talking points on your channel on the Discord server before the seminar and post your impressions afterwards. 
\\

\noindent
\textbf{Readings}
\nocite{*}
\setlength\bibitemsep{0.65\baselineskip}
\printbibliography[keyword=seminar, heading=none] 

% ---------------------------------------------------------------------------- 

\section{2 Kinship}

What do the lives of a city's non-humans look like? Before we can design for or better yet with urban ecologies, we should strive understand and even empathize with our city's co-inhabitants. 

\paragraph{Project:} A Day in the Life of\ldots \\

\noindent
Observe, map, and illustrate a day in the life of a non-human in New York City. Potential subjects include foxes, rats, squirrels, ants, cockroaches, bats, owls, bees, butterflies, dachshunds, hot dogs, subway cars, trees, weeds, joints, or rain drops. Post your illustration on your Discord channel and present it to the studio. 
%\\
%
%\noindent
%\textbf{Precedents}
%...

% ---------------------------------------------------------------------------- 

\section{3 Urban Systems}

Learn about your site by mapping the entangled systems that compose Manhattan. Research the pre-colonial history of Manahatta, the human and non-human inhabitants of the contemporary city, and the ongoing development of Hudson Yards.

\paragraph{Project:} Mapping City Systems \\

\noindent
In small teams, make maps of Manhattan's urban systems. Each team should address a theme such as history, ecology, geology, culture, socio-economics, infrastructure, and urban form. Maps may include images, timelines, plots, sections, and diagrammatic elements. Post your maps on your Discord channel and present them to the studio. 

% ---------------------------------------------------------------------------- 

\section{4 Habitat}

Who are Manhattan's non-human inhabitants and how can we design for them? What do other species need to inhabit the city? Who used to live in Manahatta, but no longer remain? How could the city be re-wilded? 

\paragraph{Project:} Habitat Research \\

\noindent
Choose a species -- one that once lived in city, that currently lives in the city, or that might come.  
Study their habit requirements and explore strategies for creating new habitat for them in the city. Potential non-extant species include beaver, black bear, etc. Potential extant species include fox, bat, chipmunk, kestrel, egret, owl, frog, bee, dragonfly, butterfly, etc. Draw a diagram showing the needs of your species and your design for them. Post your diagram on Discord and present it the studio. 

% ---------------------------------------------------------------------------- 

\section{5 Fieldwork}

We will visit New York City for fieldwork from September 23-26. On September 24, we will visit the site in the morning and take a boat tour in the afternoon. The site visit will include Little Island, the Whitney, the High Line, Hudson Yard, Penn Station, and the Hudson River Greenway. The cruise will be AIANY's Climate Change, Architecture, and the Future of NYC Tour. On September 25, you will document the site and conduct fieldwork in small groups. Documentation methods include photography, video, sketching, botanical inventory, and phenomenological mapping. For fieldwork, pick any three of the following projects.

\paragraph{Fieldwork 1:} A Day in the Life of | Streets of Manhattan | Streets of Manahatta | Scenes of Manhattan | Sounds of Manhattan | Smells of Manhattan | City Safari | City Block Transects | Urban Assemblages | Cognitive Mapping

\paragraph{Precedents}
Scape, \href{https://www.scapestudio.com/projects/981/}{Safari 7} |
Kate McLean, \href{https://sensorymaps.com/?projects=summer-streets-smellmapping-astor-place-nyc}{Smellmapping}
% Heidi Neilson, \href{https://heidineilson.com/urban-forest-on-14th-street/}{Urban Forest on 14th Street}

\paragraph{Resources}
McLean, Kate, \href{https://sensorymaps.com/wp-content/uploads/2024/07/Smellfie_Kit_2020_compressed.pdf}{Smellfie Kit}, 2020.

% ---------------------------------------------------------------------------- 

\section{6-7 Urban Form}

Working in small teams, develop conceptual designs for the urban form of Hudson Yards West. In phase one of the Hudson Yards development, a platform was built over the eastern railyard to maintain rail operations, while creating a blank slate for construction above. In phase two, the platform will be extended over the western railyard. Design alternative proposals for building over the railyard. Use massing models to design multiple proposals for urban form over the western railyard. How could new development honor the infrastructural heritage of the site, remember its geological and ecological history, build connections to the surrounding city, foster an inclusive community, and create space for multiple species?

\paragraph{Project:} Massing Model \\

\noindent
Design massing models for the site. Explore different strategies for building over the railyard. Deliverables include a laser-cut context model, physical massing models at the same scale, and floor area ratio and green area ratio calculations. 

% ---------------------------------------------------------------------------- 

\section{8-9 Landscape Plan}

Working in small teams, design a masterplan for Hudson Yards West based on your massing model. The design should be more-than-human -- it should be inclusive, welcoming Manhattan's human and non-human inhabitants.  The masterplan should be based on a big concept such as rewilding, vertical forests, or biophilia. How could the site be rewilded to a pre-Columbian or paleoecological baseline, while housing more New Yorkers? How could a forest grow in a city of skyscrapers where every square foot counts? What agency do trees and spontaneous vegetation, i.e.~weeds, have in the heart of the city? Consider the provisioning of ecosystem services, thermal comfort, climate resilience, and biodiversity conservation. You should think 3-dimensionally and design not only horizontal, but also vertical spaces. A significant proportion of the development should be dedicated to public arts and cultural programming, fair, affordable housing, green space, and non-human habitat. 

\paragraph{Project:} Masterplan \\

\noindent
Design a masterplan for the site. Deliverables include an illustrative masterplan, sections, diagrams, and a laser-cut model. Diagrams should include massing, programming, circulation, plant communities, and habitat.

% ---------------------------------------------------------------------------- 

\section{10 Charrette}

In this week-long charrette, you will collaborate with architecture students to design habitat facades for your masterplan. These facades should create habitat for specific species and express an ecological aesthetic. 

\paragraph{Project:} Habitat Facades \\

\noindent
Work in interdisciplinary teams to design a habitat facade. Deliverables include perspective renderings, illustrative sections-elevations, detail design drawings, diagrams, material specifications, and a plant list. Renderings by generative artificial intelligence are encouraged.  
\\

\noindent
\textbf{Precedents}
Harrison Atelier, \href{https://www.harrisonatelier.com/ontheground/}{On the Ground} \& \href{https://www.harrisonatelier.com/barcelonareusingrooftop/}{Feral Surfaces} |
Stefano Boeri Architetti, \href{https://www.stefanoboeriarchitetti.net/en/project/vertical-forest/}{Vertical Forest Milan} | Heatherwick Studio, \href{https://heatherwick.com/project/1000trees/}{1000 Trees} | op.AL \href{https://www.op-al.com/aeroponicaggregates}{Aeroponic Aggregates}

% ---------------------------------------------------------------------------- 

\section{11-15 Design}

As an individual project, create a schematic design for part of the masterplan. Select an area or system from the masterplan to develop in more detail. For example, you may choose to focus on a key public open space, on a wildlife friendly planting plan, on habitat facades, or on habitat islands. Your design should include  The design package should include select details and a custom design element. Design elements could include habitat structures, landscape furniture, signage, sculpture, etc.  

\paragraph{Project:} Site Design \\

\noindent
Deliverables include a plan, a laser-cut model, several section-elevations, several perspective renderings, a planting plan, a paving plan, a habitat plan, site system diagrams, details, and a custom design element. One of the section-elevations should be deep, revealing the interior of the buildings above, the railyard below, and the soil strata and infrastructure deeper still.

\section{Final Review}
For the final review, present a summary of the semester's work -- your research, masterplan, and site designs -- as a team. Each team will present on the screen in the auditorium and exhibit their models on the stage. 
% Poster Format
% Date

% -------------------------------- Policies -------------------------------- 

\section{Policies}

\paragraph{Time Commitment Expectations}
LSU's general policy states that for each credit hour, 
you (the student) should plan to
spend at least two hours working 
on course related activities outside of class. 
Since this course is for three credit hours, 
you should expect to spend a minimum of six hours 
outside of class each week 
working on assignments for this course. 
For more information see: 
\url{http://catalog.lsu.edu/content.php?catoid=12&navoid=822}.

\paragraph{LSU student code of conduct}
The LSU student code of conduct explains 
student rights, excused absences, 
and what is expected of student behavior. 
Students are expected to understand this code:  
\url{http://students.lsu.edu/saa/students/code}.

\paragraph{Disability Code}
The University is committed to making reasonable efforts 
to assist individuals with disabilities in
their efforts to avail themselves of 
services and programs offered by the University. 
To this end, Louisiana State University will provide 
reasonable accommodations for persons
with documented qualifying disabilities.
 If you have a disability and 
 feel you need accommodations in this course, 
 you must present a letter to me 
 from Disability Services in 115 Johnston Hall,
indicating the existence of a disability 
and the suggested accommodations.

\paragraph{Academic Integrity}
According to section 10.1 
of the LSU Code of Student Conduct, 
``A student may be charged with Academic Misconduct'' 
for a variety of offenses, including the following: 
unauthorized copying, collusion, or collaboration; 
``falsifying'' data or citations; 
``assisting someone in the commission 
or attempted commission of an offense''; 
and plagiarism, which is defined in section 10.1.H as a 
``lack of appropriate citation, 
or the unacknowledged inclusion 
of someone else's words, structure, ideas, or data; 
failure to identify a source, 
or the submission of essentially the same work 
for two assignments 
without permission of the instructor(s).''

\paragraph{Plagiarism and Citation Method}
Plagiarism is the 
``lack of appropriate citation, or the unacknowledged inclusion 
of someone else's words, structure, ideas, or data; 
failure to identify a source, 
or the submission of essentially the same work 
for two assignments without permission of the instructor(s)'' 
(Sec. 10.1.H of the LSU Code of Student Conduct). 
As a student at LSU, 
it is your responsibility to refrain from 
plagiarizing the academic property of another 
and to utilize appropriate citation method for all coursework. 
In this class, it is recommended that you use 
Chicago Style author-date citations. 
Ignorance of the citation method
 is not an excuse for academic misconduct.

\paragraph{Generative Artificial Intelligence}
In this course the use of generative artificial intelligence (AI) 
is permitted for the purposes of enhancing your understanding 
of course materials, encouraging creative exploration, 
and supporting academic growth. 
The use of generative AI, however, 
is not permitted for writing assignments. 
These programs should not be used to produce work 
that misrepresents your abilities 
or deceives as to the conditions 
under which the work was completed. 
If you use AI to generate content 
you must clearly acknowledge the use of AI generated material. 
Proper attribution of AI program use
should include an explanation of how the program 
contributed to the assignment and your academic growth. 
Failing to give proper attribution 
to the use of AI programs in academic work
 will be reported to Student Advocacy \& Accountability 
 for review under the Code of Student Conduct 
 and may result in impacts to your assignment and course grades.

\paragraph{Expectations}
As an accredited Landscape Architecture program
LSU's Robert Reich School of Landscape Architecture (RRSLA) 
must meet the accreditation requirements 
as stated by the Landscape Architectural Accreditation
Board (LAAB) to ensure RRSLA
 is meeting the expectations of the field. 
The LAAB requires programs to provide digital copies 
of student work as part of this process.
Students in this course will be expected 
to comply with the following requirements
as 4\% of their course grade: 
(1) Students must provide a course portfolio
with work samples specified by the instructor 
before the end of the grading period. 
(2) Each student's course portfolio must be saved as 
a single, high resolution PDF file with multiple pages. 
(3) Files must follow the naming convention
established by the school: 
department-coursenumber-semesteryear-username.pdf.
Example: LA5001-F2024-baharmon.pdf.

\end{document}
