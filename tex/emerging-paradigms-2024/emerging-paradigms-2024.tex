%!TEX TS-program = xelatex
%!TEX encoding = UTF-8 Unicode

\documentclass[11pt,article,oneside]{memoir}

% packages
\usepackage{org-preamble-xelatex}
\usepackage{wallpaper}
\usepackage{xcolor}
\usepackage{multicol}
\usepackage{enumitem}
\setlist[itemize]{leftmargin=*}
\usepackage{tikz}
\usepackage{tikzpagenodes} 

\AtBeginBibliography{\small}

% Definitions
\def\myauthor{Author}
\def\mytitle{Title}
\def\mycopyright{\myauthor}
\def\mykeywords{}
\def\mybibliostyle{plain}
\def\mybibliocommand{}
\def\mysubtitle{}
\def\myaffiliation{Louisiana State University}
\def\myaddress{309 Design}
\def\myemail{baharmon@lsu.edu} 
\def\myweb{https://baharmon.github.io/}
\def\myphone{919.622.8414}
\def\myversion{}
\def\myrevision{}
\def\myaffiliation{\ \\Louisiana State University}
\def\myauthor{Brendan Harmon}
\def\mykeywords{Landscape Architecture, Syllabus, Graduate}
\def\mysubtitle{Syllabus}
\def\mytitle{ \includegraphics[width=6cm]{../logos/lsu_art_design_logo.pdf} \\[0.1cm] {\normalfont \normalsize LA 7080 |} \Large Emerging Paradigms} 

% color
\makeatletter
\newcommand{\globalcolor}[1]{%
  \color{#1}\global\let\default@color\current@color
}
\makeatother

% begin
\begin{document}

\setlength\bibitemsep{0.5em}
\defaultfontfeatures{}
\defaultfontfeatures{Scale=MatchLowercase}
\setmainfont{IBM Plex Sans}
\setmonofont[Scale=0.8]{IBM Plex Mono}

\def\ind{\hangindent=1 true cm\hangafter=1 \noindent}
\def\labelitemi{$\cdot$}
\chapterstyle{article-4-sans}  

\title{\LARGE \mytitle}
\author{\Large\myauthor \newline \footnotesize\texttt{\noindent\myemail}}
\date{Fall 2024 | Fridays | 9:30-12:30 | Design 308}
\published{\,}

% -------------------------------- DESCRIPTION -------------------------------- 

\pagenumbering{arabic}
\globalcolor{black}

\vspace*{-10em}
\maketitle

\vspace*{-2em}
\section{Description}
This seminar is an introduction to 
emerging paradigms of landscape architecture.
Advances in technology are transforming 
the practice of landscape architecture,
while new philosophical paradigms 
are shifting the theoretical foundations of the discipline. 
While advances in artificial intelligence 
have begun to distribute creative agency
more broadly between humans and machines,
new theoretical currents are 
challenging anthropocentrism,
advocating for the agency of non-humans, 
and rethinking our relationship to the planet. 
Where do these emerging directions
in technology and theory converge
and where do they diverge? 
How are these advances reshaping
the ethical, political, creative, and aesthetic dimensions 
of the discipline? 
This seminar will meet biweekly 
on Friday mornings to 
discuss and critically reflect on emerging paradigms
in landscape architecture. 
Every other Friday
the class will convene for 
research-creation sessions to explore
how these emerging paradigms 
can be put into practice. 
Announcements, readings, student work, and support
will be posted on the course's Discord server at 
\url{https://discord.gg/aas6dzJxEc}.

% -------------------------------- SCHEDULE -------------------------------- 

\section{Schedule}

\begin{table}[H]
\begin{tabular}{l l l l}
\textbf{1} & Artificial Intelligence I  \hspace{4em}  & \textbf{8} & Xenofeminism II \\
\textbf{2} & Artificial Intelligence II \hspace{4em}  & \textbf{9} & Vibrant Matter I \\
\textbf{3} & Radiance Fields I \hspace{4em}  & \textbf{10} & Vibrant Matter II \\
\textbf{4} & Radiance Fields II \hspace{4em}  & \textbf{11} & More Than Human I \\
\textbf{5} & Cloudism I \hspace{4em}  & \textbf{12} & More Than Human II \\
\textbf{6} & Cloudism II \hspace{4em}  & \textbf{13} & Cyborgs I \\
\textbf{7} & Xenofeminism I  \hspace{4em}  & \textbf{14} & Cyborgs II\\
\end{tabular}
\end{table}

% ------------------------------- Software -------------------------------- 

% Adobe Firefly, Photoshop, Luma AI, Luma Genie, etc

% -------------------------------- Grading -------------------------------- 

\section{Grading}

\begin{table}[H]
\begin{tabular}{l @{\hskip 1cm} l @{\hskip 1cm} l @{\hskip 1cm} l}
\textbf{12\%} \enspace Artificial Intelligence & \textbf{12\%} \enspace Radiance Fields & \textbf{12\%} \enspace Cloudism \\
\textbf{12\%} \enspace Xenofeminism & \textbf{12\%} \enspace Vibrant Matter & \textbf{12\%} \enspace More Than Human \\
\textbf{12\%} \enspace Cyborgs & \textbf{12\%} \enspace Exhibition & \textbf{4\%} \quad Portfolio \\
\end{tabular}
\end{table}

% -------------------------------- Readings -------------------------------- 

\section{1 Artificial Intelligence}

Generative artificial intelligence is being used by landscape architects 
to conduct research,
allocate program, 
draw plans, 
to render scenes,
and generate asset libraries.
How else can artificial intelligence be used in the discipline?
How is artificial intelligence transforming 
the practice of landscape architecture?

\paragraph{Project:} Automatic Case Studies \\

\noindent
Use a custom large language model 
to automatically conduct a case study 
of a famous landscape architecture project. 
Critique the results.
Was the project correctly identified?
Was the designer correctly attributed?
How accurate was the stylistic analysis?
\\

\noindent
\textbf{Readings}
\nocite{*}
\setlength\bibitemsep{0.65\baselineskip}
\printbibliography[keyword=ai, heading=none]

\section{2 Radiance Fields}

Neural radiance fields 
-- an artificial intelligence based reconstruction technique -- 
can be used to capture landscapes in immersive detail. 
How can this emerging technology be used to 
record, design, and imagine landscape? 

\paragraph{Project:} Scanning Plants\\

\noindent
Capture a plant specimen as a neural radiance field,
export a point cloud, and publish it online in the 
\href{https://xyz.cct.lsu.edu/cloud-forest/}{Cloud Forest} collection.
\\

\noindent
\textbf{Readings}
\nocite{*}
\setlength\bibitemsep{0.65\baselineskip}
\printbibliography[keyword=radiance-fields, heading=none]

\section{3 Cloudism}

Point clouds captured using remote sensing
can be used to precisely model landscapes,
capturing both the structure
and the phenomenological detail
of the scene.
Landscape architects are experimenting
with point cloud modeling
as new creative medium.

\paragraph{Project:} Sensing Landscapes \\

\noindent
Capture a tree specimen using terrestrial lidar
and publish the point cloud online in the 
\href{https://xyz.cct.lsu.edu/cloud-forest/}{Cloud Forest} collection.

\paragraph{Precedents:} 
Philipp Urech, 2020, \href{https://skfb.ly/oRCrE}{Dublin Green Retrofit}.
\\

\noindent
\textbf{Readings}
\nocite{*}
\setlength\bibitemsep{0.65\baselineskip}
\printbibliography[keyword=cloudism, heading=none]

\section{4 Xenofeminism}

Xenofeminism is `a politics for alienation'.
It seeks freedom through alterity; 
it seeks to escape from
essentialism, naturalism, universalization, and mass virtuality.
It seeks to subvert technology for the sake of freedom. 
So then what would an ecology without nature look like?
How would a xenofeminist practice landscape architecture?

\paragraph{Project:} Chimera \\

\noindent
Use a text-to-3D model to create a hybrid of human and a non-human.
3D print your model. 

\paragraph{Precedents:} 
Špela Petrič, 2015--2017, \href{https://www.spelapetric.org/}{Confronting Vegetal Otherness}.
\\

\noindent
\textbf{Readings}
\nocite{*}
\setlength\bibitemsep{0.65\baselineskip}
\printbibliography[keyword=xenofeminism, heading=none]

\section{5 Vibrant Matter}

In vital materialism agency is distributed. 
Vital materialism 
posits that matter matters,
that together things 
have the capacity enact change.
How can inanimate matter have agency? 
When have you noticed the agency of a non-human assemblage before?
How could the agency of design be distributed? 

\paragraph{Project:} Assemblages \\

\noindent
Capture a non-human or more-than-human assemblage 
as a neural radiance field. 
Export and 3D print your model. 
\\

\noindent
\textbf{Readings}
\nocite{*}
\setlength\bibitemsep{0.65\baselineskip}
\printbibliography[keyword=vibrant-matter, heading=none]

\section{6 More Than Human}  

More-than-human theory
challenges anthropocentrism and
binary classifications
of us and them,
of humans and nature.
It posits an entangled state
in which humans and non-humans
are inextricably tied.
How could more-than-human design be practiced?
What would landscape architecture look like 
if non-human agency were acknowledged
and the designer de-centered? 

\paragraph{Project:} A Day in the Life of... \\

\noindent
Observe, map, and illustrate a day in the life of a non-human.

\paragraph{Precedents:} 
Marshmallow Laser Feast, 2015, \href{http://intheeyesoftheanimal.com/}{In the Eyes of the Animal}.
\\
%Marshmallow Laser Feast, 2023, \href{https://marshmallowlaserfeast.com/}{Breathing with the Forest}.\\

\noindent
\textbf{Readings}
\nocite{*}
\setlength\bibitemsep{0.65\baselineskip}
\printbibliography[keyword=more-than-human, heading=none]

\clearpage

\section{7 Cyborgs}

In cyborg landscapes 
humans, non-humans, and infrastructure evolve together, 
adapting to their changing environment. 
How can landscapes be designed for change?

\paragraph{Project:} Speculative Landscapes \\

\noindent
Use a text-to-image model to generate a rendering of a speculative cyborg landscape.

\paragraph{Precedents:} 
Lab for Environmental Design Strategies, 2022, \href{https://lab-eds.org}{Foresting Architecture}.
\\

\noindent
\textbf{Readings}
\nocite{*}
\setlength\bibitemsep{0.65\baselineskip}
\printbibliography[keyword=cyborgs, heading=none]

%\clearpage

% -------------------------------- Policies -------------------------------- 

\section{Policies}

\paragraph{Attendance}

To receive credit for the discussion seminar 
you must be an active participant 
and demonstrate an understanding of the assigned readings.
If you have an excused absence from a seminar session,
then you must write and submit a two page response to the topic
to make up for the missed discussion.
Your response should include references with citations
beyond the required readings. 
If you have an excused absence from a research-creation session,
then you must complete the project outside of class to get credit. 

\paragraph{Time Commitment Expectations}
LSU's general policy states that for each credit hour, 
you (the student) should plan to
spend at least two hours working 
on course related activities outside of class. 
Since this course is for three credit hours, 
you should expect to spend a minimum of six hours 
outside of class each week 
working on assignments for this course. 
For more information see: 
\url{http://catalog.lsu.edu/content.php?catoid=12&navoid=822}.

\paragraph{LSU student code of conduct}
The LSU student code of conduct explains 
student rights, excused absences, 
and what is expected of student behavior. 
Students are expected to understand this code:  
\url{http://students.lsu.edu/saa/students/code}.
%Any violations of the LSU student code will be duly reported to the Dean of Students.

\paragraph{Disability Code}
The University is committed to making reasonable efforts 
to assist individuals with disabilities in
their efforts to avail themselves of 
services and programs offered by the University. 
To this end, Louisiana State University will provide 
reasonable accommodations for persons
with documented qualifying disabilities.
 If you have a disability and 
 feel you need accommodations in this course, 
 you must present a letter to me 
 from Disability Services in 115 Johnston Hall,
indicating the existence of a disability 
and the suggested accommodations.

\paragraph{Academic Integrity}
According to section 10.1 
of the LSU Code of Student Conduct, 
``A student may be charged with Academic Misconduct'' 
for a variety of offenses, including the following: 
unauthorized copying, collusion, or collaboration; 
``falsifying'' data or citations; 
``assisting someone in the commission 
or attempted commission of an offense''; 
and plagiarism, which is defined in section 10.1.H as a 
``lack of appropriate citation, 
or the unacknowledged inclusion 
of someone else's words, structure, ideas, or data; 
failure to identify a source, 
or the submission of essentially the same work 
for two assignments 
without permission of the instructor(s).''

\paragraph{Plagiarism and Citation Method}
Plagiarism is the 
``lack of appropriate citation, or the unacknowledged inclusion 
of someone else's words, structure, ideas, or data; 
failure to identify a source, 
or the submission of essentially the same work 
for two assignments without permission of the instructor(s)'' 
(Sec. 10.1.H of the LSU Code of Student Conduct). 
As a student at LSU, 
it is your responsibility to refrain from 
plagiarizing the academic property of another 
and to utilize appropriate citation method for all coursework. 
In this class, it is recommended that you use 
Chicago Style author-date citations. 
Ignorance of the citation method
 is not an excuse for academic misconduct.

\paragraph{Generative Artificial Intelligence}
In this course the use of generative artificial intelligence (AI) 
is permitted for the purposes of enhancing your understanding 
of course materials, encouraging creative exploration, 
and supporting academic growth. 
The use of generative AI, however, 
is not permitted for writing assignments. 
These programs should not be used to produce work 
that misrepresents your abilities 
or deceives as to the conditions 
under which the work was completed. 
If you use AI to generate content 
you must clearly acknowledge the use of AI generated material. 
Proper attribution of AI program use
should include an explanation of how the program 
contributed to the assignment and your academic growth. 
Failing to give proper attribution 
to the use of AI programs in academic work
 will be reported to Student Advocacy \& Accountability 
 for review under the Code of Student Conduct 
 and may result in impacts to your assignment and course grades.

\paragraph{Expectations}
As an accredited Landscape Architecture program
LSU's Robert Reich School of Landscape Architecture (RRSLA) 
must meet the accreditation requirements 
as stated by the Landscape Architectural Accreditation
Board (LAAB) to ensure RRSLA
 is meeting the expectations of the field. 
The LAAB requires programs to provide digital copies 
of student work as part of this process.
Students in this course will be expected 
to comply with the following requirements
as 4\% of their course grade: 
(1) Students must provide a course portfolio
with work samples specified by the instructor 
before the end of the grading period. 
(2) Each student's course portfolio must be saved as 
a single, high resolution PDF file with multiple pages. 
(3) Files must follow the naming convention
established by the school: 
department-coursenumber-semesteryear-username.pdf.
Example: LA7080-F2024-baharmon.pdf.

\end{document}
