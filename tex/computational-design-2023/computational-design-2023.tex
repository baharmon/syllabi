%!TEX TS-program = xelatex
%!TEX encoding = UTF-8 Unicode

%%%  Syllabus template for use with style files at http://kjhealy.github.com/latex-custom-kjh
%%%  Kieran Healy

\documentclass[11pt,article,oneside]{memoir}

% packages
\usepackage{org-preamble-xelatex}
\usepackage{wallpaper}
\usepackage{xcolor}
\usepackage{multicol}
\usepackage{enumitem}
\setlist[itemize]{leftmargin=*}
\usepackage{tikz}
\usepackage{tikzpagenodes} 

\AtBeginBibliography{\small}

% Definitions
\def\myauthor{Author}
\def\mytitle{Title}
\def\mycopyright{\myauthor}
\def\mykeywords{}
\def\mybibliostyle{plain}
\def\mybibliocommand{}
\def\mysubtitle{}
\def\myaffiliation{Louisiana State University}
\def\myaddress{309 Design}
\def\myemail{baharmon@lsu.edu} 
\def\myweb{https://baharmon.github.io/}
\def\myphone{919.622.8414}
\def\myversion{}
\def\myrevision{}
\def\myaffiliation{\ \\Louisiana State University}
\def\myauthor{Brendan Harmon}
\def\mykeywords{Landscape Architecture, Syllabus, Graduate}
\def\mysubtitle{Syllabus}
\def\mytitle{ \includegraphics[width=6cm]{../logos/lsu_art_design_logo.pdf} \\[0.1cm] {\normalfont \normalsize LA 7032 |} \Large Computational Design} 

% color
\makeatletter
\newcommand{\globalcolor}[1]{%
  \color{#1}\global\let\default@color\current@color
}
\makeatother

% begin
\begin{document}

\setlength\bibitemsep{0.5em}

% fonts
\defaultfontfeatures{}
\defaultfontfeatures{Scale=MatchLowercase}
\setmainfont{IBM Plex Sans}
\setmonofont[Scale=0.8]{IBM Plex Mono}

\def\ind{\hangindent=1 true cm\hangafter=1 \noindent}
\def\labelitemi{$\cdot$}
\chapterstyle{article-4-sans}  

\title{\LARGE \mytitle}
\author{\Large\myauthor \newline \footnotesize\texttt{\noindent\myemail}}
\date{Spring 2023. \newline Tuesday \& Thursday \newline 1:00--3:30pm \newline Design 308}
\published{\,}

% -------------------------------- COVER PAGE -------------------------------- 

\pagenumbering{gobble}
\globalcolor{black}
\begin{tikzpicture}[remember picture,overlay]
\fill[white] ([xshift=-1cm,yshift=3.5cm]current page text area.west) rectangle (7.25,2);
\end{tikzpicture}
\vspace*{-10em}
\maketitle
\ThisCenterWallPaper{1.1}{../images/rosedown-landscape.jpg}
\clearpage


% -------------------------------- DESCRIPTION -------------------------------- 

\pagenumbering{arabic}
\globalcolor{black}

\vspace*{-10em}
\maketitle

\section{Course Description}

This course is an introduction to 
computational design for landscape architects.
In this course you will learn how to 
algorithmically model landscapes,
numerically simulate physical processes 
like the flow of water and sediment,
and digitally fabricate landforms.  
You will conduct surveys with drones and terrestrial lidar,
use point clouds to model and visualize terrain and planting,
parametrically model variations on landscapes
using visual programming, 
and computer numerical control (CNC) mill landforms. 
In this course you will 
learn creative approaches to computational design 
and apply emerging technologies to landscape architecture.
\\

\noindent\textbf{Keywords}
\begin{multicols}{3}
\raggedright
\small
\begin{itemize}
\item Algorithmic architecture
\item Generative design
\item Parametric modeling
\item Laser scanning
\item Lidar analytics
\item Drone photogrammetry
\item Visual programming
\item Digital fabrication
\item Robotics
\end{itemize}
\end{multicols}

% -------------------------------- SCHEDULE -------------------------------- 
\section{Schedule}

\begin{table}[H]
\begin{tabular}{l @{\hskip 2.1cm} l @{\hskip 2cm} l}
\textbf{Sensing} & \textbf{Computation} & \textbf{Fabrication}\\
\end{tabular}
\end{table}
%
\vspace*{-1em}
%
\begin{table}[H]
\small
\begin{tabular}{l l l l l l}
\small
\textbf{1} & Laser scanning  I  & \textbf{6} & Visual programming & \textbf{11} & Grading\\
\textbf{2} & Laser scanning  II & \textbf{7} & Randomness & \textbf{12} & Digital Fabrication I\\
\textbf{3} & Drones I & \textbf{8} & Noise & \textbf{13} & Digital Fabrication II\\
\textbf{4} & Drones II  & \textbf{9} & Attractors & \textbf{14} & Robotics\\
\textbf{5} & Terrain Analysis & \textbf{10} & Physics & \textbf{15} & Robotic Wall\\

\end{tabular}
\end{table}

\clearpage

% -------------------------------- Logistics -------------------------------- 
\section{Logistics}

During our regularly scheduled class period
on Tuesdays and Thursdays from 1:00-3:50 pm,
we will meet in person, while also posting
on our Discord server at \url{https://discord.gg/6kNkp2PSsu}.
The discord server will be used for posting
announcements, student work in progress, 
reading responses, projects, and troubleshooting. 
All course content including tutorials, lectures, and datasets
will be published on the course website at:
\url{https://baharmon.github.io/generative-landscapes}.\\

\noindent
Course website | \url{https://baharmon.github.io/generative-landscapes}\\
Discord | \url{https://discord.gg/6kNkp2PSsu}\\
Youtube | \url{https://www.youtube.com/@baharmon}\\

% -------------------------------- Essays -------------------------------- 
\section{Essays}

\noindent \textbf{Essay: Cloudism}
Read Christophe Girot's essay \emph{Cloudism}
and write a 500-word critical response. 
What potential do you see in point clouds
as a medium for landscape architecture?
What makes point clouds different
from other modes of representation?
\nocite{*} \printbibliography[keyword=a, heading=none]

\noindent \textbf{Essay: The Alphabet and Algorithm}
Read Mario Carpo's \emph{The Alphabet and Algorithm}
and then in response write a 500-word critical essay.
How have digital tools and processes transformed 
the practice of landscape architecture
and how do you think they will shape 
the future of the discipline?
How do you envision using
digital design tools and processes in your work?

\nocite{*} \printbibliography[keyword=b, heading=none]

% -------------------------------- Projects -------------------------------- 
\section{Projects}

\noindent \textbf{Point Cloud Library}
Use laser scanning to capture a library of plants as point clouds. 
Publish the point cloud to the web in an interactive viewer.
\\

\noindent \textbf{Drone Survey}
Conduct a topographic survey 
of the landform at Hilltop Arboretum
with a drone.
Use real-time kinematic GNSS
to establish ground control points. 
Use structure-from-motion photogrammetry 
to generate a point cloud, 
digital surface model, 
and orthophoto.
\\

\noindent \textbf{Parametric Landscape}
Use visual programming to generate
variations of landforms, plantings, and paving.
CNC mill, thermoform, and then cast 
an algorithmically generated terrain model.
\\

\noindent \textbf{Parametric Wall}
Program a robotic arm to assemble 
a parametrically modeled, curvilinear wall
from bricks.
\\

% -------------------------------- Grading -------------------------------- 
\section{Grading}
%
\begin{table}[H]
%\small
\begin{tabular}{l r @{\hskip 2cm} l @{\hskip 0.5cm} l}
%\begin{tabular}{l l}
%
Essays & 15\% &  Parametric Landscape & 20\% \\
Point Cloud Library & 20\% & Parametric Wall & 20\%\\
Drone Survey & 20\% & Course Portfolio & 5\% \\
%
\end{tabular}
\end{table}

% -------------------------------- Software -------------------------------- 
\section{Software}

Rhinoceros | \url{https://www.rhino3d.com/}\\
%Grasshopper | \url{http://grasshopper3d.com/}\\
%RhinoTerrain | \url{http://www.rhinoterrain.com/}\\
%Thea Render for Rhino | \url{https://www.thearender.com/}\\
Metashape | \url{https://www.agisoft.com/}\\
Faro Scene | \url{https://www.faro.com/}\\
CloudCompare | \url{https://www.danielgm.net/cc/}\\
GRASS GIS | \url{https://grass.osgeo.org/}\\
QGIS | \url{https://www.qgis.org/}\\
%ArcGIS | \url{https://www.esri.com/}\\
%Python | \url{https://www.python.org/}\\

\section{Plugins}
Snapping Gecko | \url{https://www.food4rhino.com/app/snappinggecko}\\
Bitmap+ | \url{https://www.food4rhino.com/en/app/bitmap}\\
Docofossor | \url{https://www.food4rhino.com/app/docofossor}\\
RhinoCAM | \url{https://mecsoft.com/rhinocam-software/}\\
%Elefront | \url{https://www.food4rhino.com/app/elefront}\\
%Bison | \url{https://www.bison.la/}\\

% -------------------------------- Resources -------------------------------- 
\section{Resources}

Grasshopper Basics | \url{https://vimeo.com/channels/basicgh}\\
Grasshopper Primer | \url{http://grasshopperprimer.com}\\
Hydrology in GRASS GIS | \url{https://grasswiki.osgeo.org/wiki/Hydrological_Sciences}\\

% -------------------------------- Readings -------------------------------- 
\section{Required Readings}
\vspace*{0.5cm}
\nocite{*}
\setlength\bibitemsep{0.65\baselineskip}
\printbibliography[keyword=required, heading=none]

% -------------------------------- Readings -------------------------------- 
\section{Recommended Readings}
\vspace*{0.5cm}
\nocite{*}
\setlength\bibitemsep{0.65\baselineskip}
\printbibliography[keyword=recommended, heading=none]

%% -------------------------------- Terminology -------------------------------- 
%\section{Terminology}
%\begin{multicols}{2}
%\raggedright
%\small
%%
%\textbf{Digital design}
%\begin{itemize}
%\item Mass customization
%\item Generative design
%\item Parametric modeling
%\item Performative design
%\item Algorithm
%%\item Scripting
%\end{itemize}
%
%\textbf{Spatial data}
%\begin{itemize}
%\item Raster \& Vector
%\item Array
%\item Point cloud
%\item Mesh
%\item Triangulated irregular network (TIN)
%%\item Discrete \& continuous data
%\item Plain text
%\item Comma separated values (CSV)
%\item Integer \& floating point numbers
%\item Quadtree \& octree
%\item Non-uniform rational basis spline (NURBS)
%\end{itemize}
%
%\textbf{Geospatial}
%\begin{itemize}
%\item Geographic information system (GIS)
%\item Digital terrain model (DTM)
%\item Digital elevation model (DEM)
%\item Digital surface model (DSM)
%\item Lidar
%\item Unmanned aerial system (UAS)
%\item Structure from motion (SfM)
%\item Delaunay triangulation
%\item Interpolation
%%\item Bilinear interpolation
%%\item Nearest neighbors
%\item Regularized spline with tension (RST)
%\item Map algebra
%%\item Null value
%%\item Least cost path (LCP)
%%\item Resampling
%%\item Image classification
%\end{itemize}
%
%\textbf{3D rendering}
%\begin{itemize}
%\item Ray tracing
%\item Diffuse shading
%\item Texture map
%\item Particle system
%\item Head mounted display (HMD)
%\item Cave automatic virtual environment (CAVE)
%\end{itemize}
%
%\textbf{Digital fabrication}
%\begin{itemize}
%\item 3D printing
%\item Computer numeric control (CNC)
%\item Collet \& Bit
%\item High density urethane (HDU)
%\item Medium density fiberboard (MDF)
%\end{itemize}
%
%%\textbf{Geomorphology}
%%\begin{itemize}
%%\item Watershed
%%\item Single flow direction (SFD/D8)
%%\item Multiple flow direction (MFD)
%%\item Revised Universal Soil Loss Equation (RUSLE)
%%\item Unit Stream Power-based Erosion Deposition (USPED)
%%\item Simulated Water Erosion (SIMWE)
%%\item R-factor
%%\item Mannings
%%\item Sediment mass density
%%\item Gully
%%\item Knickpoint
%%\end{itemize}
%%
%\end{multicols}
%
%\clearpage

% -------------------------------- Policies -------------------------------- 
\section{Policies}

\noindent \textbf{Time Commitment Expectations}
LSU's general policy states that for each credit hour, you (the student) should plan to
spend at least two hours working on course related activities outside of class. Since this course is for three credit hours, you should expect to spend a minimum of six hours outside of class each week working on assignments for this course. For more information see: 
\url{http://catalog.lsu.edu/content.php?catoid=12&navoid=822}.\\

\noindent \textbf{LSU student code of conduct}
The LSU student code of conduct explains student rights, excused absences, and what is expected of student behavior. Students are expected to understand this code:  \url{http://students.lsu.edu/saa/students/code}.\\ 
%Any violations of the LSU student code will be duly reported to the Dean of Students.\\

\clearpage

\noindent \textbf{Disability Code}
The University is committed to making reasonable efforts to assist individuals with disabilities in
their efforts to avail themselves of services and programs offered by the University. To this end,
Louisiana State University will provide reasonable accommodations for persons with
documented qualifying disabilities. If you have a disability and feel you need accommodations in
this course, you must present a letter to me from Disability Services in 115 Johnston Hall,
indicating the existence of a disability and the suggested accommodations.\\

\noindent \textbf{Academic Integrity}
According to section 10.1 of the LSU Code of Student Conduct, ``A student may be charged with Academic Misconduct'' for a variety of offenses, including the following: unauthorized copying, collusion, or collaboration; ``falsifying'' data or citations; ``assisting someone in the commission or attempted commission of an offense''; and plagiarism, which is defined in section 10.1.H as a ``lack of appropriate citation, or the unacknowledged inclusion of someone else's words, structure, ideas, or data; failure to identify a source, or the submission of essentially the same work for two assignments without permission of the instructor(s).''\\

\noindent \textbf{Plagiarism and Citation Method}
Plagiarism is the ``lack of appropriate citation, or the unacknowledged inclusion of someone else's words, structure, ideas, or data; failure to identify a source, or the submission of essentially the same work for two assignments without permission of the instructor(s)'' (Sec. 10.1.H of the LSU Code of Student Conduct). As a student at LSU, it is your responsibility to refrain from plagiarizing the academic property of another and to utilize appropriate citation method for all coursework. In this class, it is recommended that you use Chicago Style author-date citations. Ignorance of the citation method is not an excuse for academic misconduct.
\\

\noindent \textbf{Accreditation Expectations}
As an accredited Landscape Architecture program
LSU's Robert Reich School of Landscape Architecture (RRSLA) 
must meet the accreditation requirements 
as stated by the Landscape Architectural Accreditation
Board (LAAB) to ensure RRSLA is meeting the expectations of the field. 
The LAAB requires programs to provide digital copies 
of student work as part of this process.
Students in this course will be expected 
to comply with the following requirements
as 5\% of their course grade: 
(1) Students must provide a course portfolio
with work samples specified by the instructor 
before the end of the grading period. 
(2) Each student's course portfolio must be saved as 
a single, high resolution PDF file with multiple pages. 
(3) Files must follow the naming convention
established by the school: department-coursenumber-semesteryear-username.pdf.
Example: LA7032-S2023 -baharmon.pdf.\\

\end{document}
