%!TEX TS-program = xelatex
%!TEX encoding = UTF-8 Unicode

%%%  Syllabus template for use with style files at http://kjhealy.github.com/latex-custom-kjh
%%%  Kieran Healy

\documentclass[11pt,article,oneside]{memoir}

% packages
\usepackage{org-preamble-xelatex}
\usepackage{wallpaper}
\usepackage{xcolor}
\usepackage{multicol}
\usepackage{enumitem}
\setlist[itemize]{leftmargin=*}

\AtBeginBibliography{\small}

% Definitions
\def\myauthor{Author}
\def\mytitle{Title}
\def\mycopyright{\myauthor}
\def\mykeywords{}
\def\mybibliostyle{plain}
\def\mybibliocommand{}
\def\mysubtitle{}
\def\myaffiliation{Louisiana State University}
\def\myaddress{Design 304}
\def\myemail{baharmon@lsu.edu}
\def\myweb{https://baharmon.github.io/}
\def\myphone{919.622.8414}
\def\myversion{}
\def\myrevision{}
\def\myaffiliation{\ \\Louisiana State University}
\def\myauthor{\normalsize Brendan Harmon}
\def\mykeywords{Landscape Architecture, Syllabus, Graduate}
\def\mysubtitle{Syllabus}
\def\mytitle{
\includegraphics[height=1cm]{../logos/lsu_art_design_logo.pdf}
\vspace*{0.5cm}
\\[0.1cm] {\normalfont \normalsize LA 4008 \& LA 7051 |} \Large Ecological Robotics
} 

% color
\makeatletter
\newcommand{\globalcolor}[1]{%
  \color{#1}\global\let\default@color\current@color
}
\makeatother

% begin
\begin{document}

\setlength\bibitemsep{0.5em}

% fonts
\defaultfontfeatures{}
\defaultfontfeatures{Scale=MatchLowercase}         
\setmainfont[Scale=1, Path = ../fonts/lato/,BoldItalicFont=Lato-RegIta,BoldFont=Lato-Reg,ItalicFont=Lato-LigIta]{Lato-Lig}
\setsansfont[Scale=1, Path = ../fonts/lato/,BoldItalicFont=Lato-RegIta,BoldFont=Lato-Reg,ItalicFont=Lato-LigIta]{Lato-Lig}
\setmonofont[Mapping=tex-text,Scale=0.8,Path = ../fonts/inconsolata/]{i}

\def\ind{\hangindent=1 true cm\hangafter=1 \noindent}
\def\labelitemi{$\cdot$}
\chapterstyle{article-4-sans}  

\title{\LARGE \mytitle}
\author{\Large\myauthor \newline \footnotesize\texttt{\noindent\myemail}}
\date{
\footnotesize
MWF 1:30pm--5:30pm
}
\published{\,}

% -------------------------------- DESCRIPTION -------------------------------- 

\pagenumbering{gobble}
\globalcolor{black}

\vspace*{-12em}

\maketitle

\vspace*{-2em}
\section{Course Description}
\footnotesize
This studio will explore ecological applications for robots. In this studio you will learn how to program industrial robots, design your own robotic tools, and robotically plant a landscape. Through a series of projects you will design a robotic process for algorithmic planting and test it both in the studio and in the field. The field site will be a set of plots at Burden where you will conduct a controlled planting experiment. As an introduction to robotics for designers, this studio will cover topics such as visual programming, generative design, robotic operations, 3D printing, 3D scanning, and drone photogrammetry. 

% -------------------------------- SCHEDULE ---------------------------------- 

\section{Topics}

\ThisCenterWallPaper{1}{../images/robotic-garden-02.png}

\vspace*{-1em}
\begin{table}[H]
\footnotesize
\begin{tabular}{l @{\hskip 1.6cm} l @{\hskip 2.1cm} l}
\textbf{Robotic Sandbox} & \textbf{Robotic Planting} & \textbf{Robotic Landscape}\\
\end{tabular}
\end{table}
%
\vspace*{-2em}
%
\begin{table}[H]
\footnotesize
\begin{tabular}{l l l l l l}
\small
\textbf{1} & Robotic Operations  & \textbf{6} & 3D Printing Seeds & \textbf{11} & Site Preparation\\
\textbf{2} & Robotic Programming & \textbf{7} & Laboratory Planting & \textbf{12} & Field Planting\\
\textbf{3} &  Robotic Tools & \textbf{8} & Algorithmic Seeding & \textbf{13} & Drone Mapping\\
\textbf{4} & Visual Programming & \textbf{9} & Algorithmic Landforms & \textbf{14} & Documentation\\
\textbf{5} & Robotic Sandbox & \textbf{10} & 3D Scanning & \textbf{15} & Exhibition\\
\end{tabular}
\end{table}

\end{document}
