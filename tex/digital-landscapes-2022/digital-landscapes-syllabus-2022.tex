%!TEX TS-program = xelatex
%!TEX encoding = UTF-8 Unicode

%%%  Syllabus template for use with style files at http://kjhealy.github.com/latex-custom-kjh
%%%  Kieran Healy

\documentclass[11pt,article,oneside]{memoir}

% packages
\usepackage{org-preamble-xelatex}
\usepackage{wallpaper}
\usepackage{xcolor}
\usepackage{multicol}
\usepackage{enumitem}
\setlist[itemize]{leftmargin=*}

\AtBeginBibliography{\small}

% Definitions
\def\myauthor{Author}
\def\mytitle{Title}
\def\mycopyright{\myauthor}
\def\mykeywords{}
\def\mybibliostyle{plain}
\def\mybibliocommand{}
\def\mysubtitle{}
\def\myaffiliation{Louisiana State University}
\def\myaddress{309 Design}
\def\myemail{baharmon@lsu.edu}
\def\myweb{https://baharmon.github.io/}
\def\myphone{919.622.8414}
\def\myversion{}
\def\myrevision{}
\def\myaffiliation{\ \\Louisiana State University}
\def\myauthor{Brendan Harmon}
\def\mykeywords{Landscape Architecture, Syllabus, Graduate, Undergraduate}
\def\mysubtitle{Syllabus}
\def\mytitle{ \includegraphics[width=6cm]{../logos/lsu_art_design_logo.pdf} \\[0.1cm] {\normalfont \normalsize LA 2101 \& 7012 |} \Large Digital Landscapes}

% color
\makeatletter
\newcommand{\globalcolor}[1]{%
  \color{#1}\global\let\default@color\current@color
}
\makeatother

% begin
\begin{document}

\setlength\bibitemsep{0.5em}

% fonts
\defaultfontfeatures{}
\defaultfontfeatures{Scale=MatchLowercase}
\setmainfont[Scale=1, Path = ../fonts/lato/,BoldItalicFont=Lato-RegIta,BoldFont=Lato-Reg,ItalicFont=Lato-LigIta]{Lato-Lig}
\setsansfont[Scale=1, Path = ../fonts/lato/,BoldItalicFont=Lato-RegIta,BoldFont=Lato-Reg,ItalicFont=Lato-LigIta]{Lato-Lig}
\setmonofont[Mapping=tex-text,Scale=0.8,Path = ../fonts/inconsolata/]{i}

\def\ind{\hangindent=1 true cm\hangafter=1 \noindent}
\def\labelitemi{$\cdot$}
\chapterstyle{article-4-sans}

\title{\LARGE \mytitle}
\author{\Large\myauthor \newline \footnotesize\texttt{\noindent\myemail}}
\date{Spring 2022 \newline Monday, Wednesday, \& Friday 9:30am--11:50am} % Design 217.
\published{\,}

% -------------------------------- COVER PAGE --------------------------------

\pagenumbering{gobble}
\globalcolor{black}
\vspace*{-10em}
\maketitle
\ThisCenterWallPaper{1}{../images/yosemite.png}
\clearpage


% -------------------------------- DESCRIPTION --------------------------------

\pagenumbering{arabic}
\globalcolor{black}

\vspace*{-10em}
\maketitle

\section{Course Description}
%
This course is an introduction to the digital modeling of landscapes.
In this course you will learn the basics of 
photomontage, vector illustration, diagramming,
3D modeling, 3D rendering, and 3D printing. 
Topics will include how to model and render
topography, hardscape, landscape elements, and planting.

% -------------------------------- SCHEDULE --------------------------------
\section{Schedule}

\begin{table}[H]
\begin{tabular}{l l @{\hskip 1.5cm} l l @{\hskip 1.5cm} l l}
\small
\textbf{1} & Graphics & \textbf{6} & Planter & \textbf{11} & Terrain\\
\textbf{2} & Photomontage & \textbf{7} & Elements I & \textbf{12} & Hardscape\\
\textbf{3} & Streetscape I & \textbf{8} & Elements II & \textbf{13} & Streetscape III\\
\textbf{4} & Streetscape II & \textbf{9} & Spring Break & \textbf{14} & Layout Design\\
\textbf{5} & Bioswale & \textbf{10} & Elements III & \textbf{15} & Virtual Reality \\[0.15cm]
& \textbf{Sections} && \textbf{Elements} && \textbf{Renderings}
\end{tabular}
\end{table}

% -------------------------------- Online -------------------------------- 

\section{Server}

During our regularly scheduled class period
on Mondays, Wednesdays, and Fridays from 9:30-11:50 am,
we will meet in person, while also posting
on our Discord server at \url{https://discord.gg/pKcwymubbd}.
The discord server will be used for posting
announcements, homework, projects, and troubleshooting. 
Course content including tutorials, lectures, and datasets
will be published on the course website at:
\url{https://baharmon.github.io/digital-landscapes}.\\

\noindent
Course website | \url{https://baharmon.github.io/digital-landscapes}\\
Discord | \url{https://discord.gg/pKcwymubbd}\\
Youtube | \url{https://www.youtube.com/c/BrendanHarmon}\\

% -------------------------------- Projects --------------------------------
\section{Projects}

Over the course of the semester 
you will design, model, and render a streetscape.
Your project should focus on
a particular aspect or type of streetscape of your choosing. 
Explore concepts for streetscapes such as 
multimodal streets, traffic circles, chicanes,
colorful crosswalks,  green infrastructure, or public art.
Your graphics will include sections, 
custom designed landscape elements,
and perspective renderings.
Design a poster for your project. 
Present your work on 
\emph{/13/2022}.\\

\noindent \textbf{Streetscape Sections}
Use vector illustration and photomontage
to render section cuts through your streetscape. 
Use diagrams to show concepts such as 
circulation, program, and stormwater management.
\emph{Due:} 02/18/2022\\

\noindent \textbf{Streetscape Elements}
3D model, 3D render, and 3D print
custom landscape elements for your streetscape.
Elements may include 
walls, seating, sculpture, shade structures,
water features, planters, bioswales, and
other green infrastructure.
\emph{Due:} 03/25/2022\\

\noindent \textbf{Streetscape Renderings}
3D model and 3D render views your streetscape.
including perspectives and orthographic transects.
The renderings of the streetscape should include 
topography, planting, custom seating, 
and other landscape elements.
\emph{Due:} 04/29/2022\\

\noindent \textbf{Course Portfolio}
Collect your work in a course portfolio
for the school's accreditation archive.
\emph{Due:} 05/13/2022\\

% -------------------------------- Grading --------------------------------
\section{Grading}

\begin{table}[H]
\begin{tabular}{l r @{\hskip 2cm} l @{\hskip 0.5cm} l}
Streetscape Sections & 20\% & Homework & 35\%\\
Streetscape Elements & 20\% & Course Portfolio & 5\% \\
Streetscape Renderings & 20\%\\

\end{tabular}
\end{table}

% -------------------------------- Software --------------------------------
\section{Software}
\begin{multicols}{2}
\raggedright
Rhinoceros | \url{https://www.rhino3d.com/}\\
%Thea Render | \url{www.thearender.com/}\\
Lumion | \url{https://lumion.com}\\
%%Enscape | \url{https://enscape3d.com/}\\
Adobe CC | \url{https://www.adobe.com/}\\
\end{multicols}

%% --------------------------------Network --------------------------------
%\section{Network drives}
%
%%\noindent
%%Submit your work via the course network drive.\\
%
%\noindent
%Windows for undergrads: \verb|\\desn-knox.lsu.edu\Landscape-Classes\LA2101-S2020| \\
%Windows for grads: \verb|\\desn-knox.lsu.edu\Landscape-Classes\LA7102-S2020| \\
%
%\noindent
%Mac for undergrads: \verb|smb://desn-knox.lsu.edu/Landscape-Classes/LA2101-S2020| \\
%Mac for grads: \verb|smb://desn-knox.lsu.edu/Landscape-Classes/LA7102-S2020| \\

% -------------------------------- Readings --------------------------------
\section{Readings}
\vspace*{0.5cm}
\nocite{*}
\setlength\bibitemsep{0.65\baselineskip}
\printbibliography[heading=none]

%\clearpage

% -------------------------------- Policies --------------------------------
\section{Policies}

\noindent \textbf{Communication Intensive}
This is a certified Communication-Intensive (C-I) course,
which meets all of the requirements set
forth by LSU’s Communication across the Curriculum program, 
including instruction and assignments emphasizing
informal and formal visual and technological communication,
teaching of discipline-specific communication techniques,
use of feedback loops for learning,
40\% of the course grade rooted in communication-based work, 
and practice of ethical and professional work standards.
Students interested in pursuing the LSU Distinguished Communicators 
certification may use this C-I course for credit. 
For more information visit: 
\url{www.cxc.lsu.edu}.\\

\noindent \textbf{Accreditation Expectations}
As an accredited Landscape Architecture program
LSU's Robert Reich School of Landscape Architecture (RRSLA)
must meet the accreditation requirements
as stated by the Landscape Architectural Accreditation
Board (LAAB) to ensure RRSLA is meeting the expectations of the field.
The LAAB requires programs to provide digital copies
of student work as part of this process.
Students in this course will be expected
to comply with the following requirements
as 5\% of their course grade:
(1) Students must provide a course portfolio
with work samples specified by the instructor
before the end of the grading period.
(2) Each student's course portfolio must be saved as
a single, high resolution PDF file with multiple pages.
(3) Files must follow the naming convention
established by the school: departmentcoursenumber-semesteryear-username.pdf.
Example: LA2101-S2021-baharmon.pdf.\\

\noindent \textbf{Late Work Policy}
There is a one week grace period for late class work. 
After the grace period, 
the grade for the assignment will be lowered by
a letter grade (i.e.~$10$ points) per week late.\\

\noindent \textbf{Attendance Policy}
When students have valid reasons for absence, they are responsible for providing reasonable advance notification and appropriate documentation of the reason for the absence and for making up examinations, obtaining lecture notes, and otherwise compensating for what may have been missed. Valid reasons that must be documented include: illness; serious family emergency; special curricular requirements such as field trips; court-imposed legal obligations such as subpoenas or jury duty; military obligations; serious weather conditions; religious observances; official participation in varsity athletic competitions or university musical events. Absences without valid reasons are limited to three per term. Beyond these limits, each unexcused absence will lower the final course grade by one letter grade increment (i.e.~$3.\overline{3}$ points).\\

\noindent \textbf{Time Commitment Expectations}
LSU's general policy states that for each credit hour, you (the student) should plan to
spend at least two hours working on course related activities outside of class. Since this course is for three credit hours, you should expect to spend a minimum of six hours outside of class each week working on assignments for this course.\\  
%For more information see:
%\url{http://catalog.lsu.edu/content.php?catoid=12&navoid=822}.\\

\noindent \textbf{LSU Student Code of Conduct}
The LSU student code of conduct explains student rights, excused absences, and what is expected of student behavior. Students are expected to understand this code:  \url{http://students.lsu.edu/saa/students/code}. Any violations of the LSU student code will be duly reported to the Dean of Students.\\

\noindent \textbf{Disability Code}
The University is committed to making reasonable efforts to assist individuals with disabilities in
their efforts to avail themselves of services and programs offered by the University. To this end,
Louisiana State University will provide reasonable accommodations for persons with
documented qualifying disabilities. If you have a disability and feel you need accommodations in
this course, you must present a letter to me from Disability Services in 115 Johnston Hall,
indicating the existence of a disability and the suggested accommodations.\\

\noindent \textbf{Academic Integrity}
According to section 10.1 of the LSU Code of Student Conduct, ``A student may be charged with Academic Misconduct'' for a variety of offenses, including: unauthorized copying, collusion, or collaboration; ``falsifying'' data or citations; ``assisting someone in the commission or attempted commission of an offense''; and plagiarism, which is defined in section 10.1.H as a ``lack of appropriate citation, or the unacknowledged inclusion of someone else's words, structure, ideas, or data; failure to identify a source, or the submission of essentially the same work for two assignments without permission of the instructor.''\\

\noindent \textbf{Plagiarism and Citation Method}
Plagiarism is the ``lack of appropriate citation, or the unacknowledged inclusion of someone else's words, structure, ideas, or data; failure to identify a source, or the submission of essentially the same work for two assignments without permission of the instructor(s)'' (Sec. 10.1.H of the LSU Code of Student Conduct). As a student at LSU, it is your responsibility to refrain from plagiarizing the academic property of another and to utilize appropriate citation method for all coursework. In this class, it is recommended that you use Harvard Style author-date citations. Ignorance of the citation method is not an excuse for academic misconduct.

\end{document}
