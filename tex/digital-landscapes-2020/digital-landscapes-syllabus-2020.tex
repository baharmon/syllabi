%!TEX TS-program = xelatex
%!TEX encoding = UTF-8 Unicode

% Class
\documentclass[11pt,article,oneside]{memoir}

% Packages
\usepackage{../style/xelatex-preamble}

% Settings
\AtBeginBibliography{\small}

% Definitions
\def\myauthor{Author}
\def\mytitle{Title}
\def\mycopyright{\myauthor}
\def\mykeywords{}
\def\mybibliostyle{plain}
\def\mybibliocommand{}
\def\mysubtitle{}
\def\myaffiliation{Louisiana State University}
\def\myaddress{309 Design}
\def\myemail{baharmon@lsu.edu}
\def\myweb{https://baharmon.github.io/}
\def\myphone{919.622.8414}
\def\myversion{}
\def\myrevision{}
\def\myaffiliation{\ \\Louisiana State University}
\def\myauthor{Brendan Harmon}
\def\mykeywords{Landscape Architecture, Syllabus, Graduate, Undergraduate}
\def\mysubtitle{Syllabus}
\def\mytitle{ \includegraphics[width=6cm]{../logos/lsu_art_design_logo.pdf} \\[0.1cm] {\normalfont \normalsize LA 2101 \& 7012 |} \Large Digital Landscapes}

% Color
\makeatletter
\newcommand{\globalcolor}[1]{%
  \color{#1}\global\let\default@color\current@color
}
\makeatother

% Begin
\begin{document}

% Fonts
\defaultfontfeatures{}
\defaultfontfeatures{Scale=MatchLowercase}
\setmainfont{IBM Plex Sans}
\setmonofont[Scale=0.8]{IBM Plex Mono}

% Style
\setlength\bibitemsep{0.5em}
\def\ind{\hangindent=1 true cm\hangafter=1 \noindent}
\def\labelitemi{$\cdot$}
\chapterstyle{syllabus}

% Frontmatter
\title{\LARGE \mytitle}
\author{\Large\myauthor \newline \footnotesize\texttt{\noindent\myemail}}
\date{Spring 2020. Design 217.\newline Monday, Wednesday, \& Friday 10:00am--11:30am.}
\published{\,}

% -------------------------------- Cover page --------------------------------

\pagenumbering{gobble}
\globalcolor{black}
\vspace*{-10em}
\maketitle
\ThisCenterWallPaper{1}{../images/bioswale-1.jpg}
\clearpage


% -------------------------------- Description --------------------------------

\pagenumbering{arabic}
\globalcolor{black}

\vspace*{-10em}
\maketitle

\section{Course Description}
%
This course is an introduction to methods and styles
for the digital representation of landscape.
In this course you will develop a solid foundation in 3D modeling
by modeling and rendering streetscapes with green infrastructure.
You will learn a range of 3D modeling techniques including
3D drafting, solid modeling, surface modeling, and mesh modeling.
You will learn how to digitally fabricate physical models
using laser cutting.\\

% -------------------------------- Online --------------------------------

\section{Online}

Lectures will be recorded and posted 
on Youtube and Vimeo. 
Please watch before class.
During class we will have discussions 
hosted on our Discord server. 
Please check Notion
for news, announcements, and logistics.\\

\noindent
Notion | \url{https://www.notion.so/Classroom-Home-d06cbc3c1c5f4096884071af442aebb8}\\
Discord | \url{https://discord.gg/BA39uFt}\\
Youtube | \url{https://www.youtube.com/channel/UCmGEF6Bf1SO92oLQoGCPDTw}\\
Vimeo | \url{https://vimeo.com/baharmon}

% -------------------------------- Schedule --------------------------------
\section{Topics}

\begin{table}[H]
\begin{tabular}{l l @{\hskip 0.5cm} l l @{\hskip 0.5cm} l l}
\small
\textbf{1} & Virtual Reality I & \textbf{6} & Typography & \textbf{11} & Surface modeling II\\
\textbf{2} & 2D drafting I & \textbf{7} & Solid modeling & \textbf{12} & Surface modeling III\\
\textbf{3} & Photomontage & \textbf{8} & Surface modeling I & \textbf{13} & 3D rendering II\\
\textbf{4} & 2D drafting II & \textbf{9} & 3D rendering I & \textbf{14} & Drone photography\\
\textbf{5} & Laser cutting & \textbf{10} & 3D printing & \textbf{15} & Virtual reality II\\
\end{tabular}
\end{table}

% -------------------------------- Projects --------------------------------
\section{Projects}
As a class you will design and build a toolkit
for the participatory design of complete streets.
Individually you will design, model and render
streetscapes and green infrastructure in 3D
to compliment to the toolkit.
Upload your working files and finished work to the course drive.\\

\noindent \textbf{Participatory Design Toolkit Characters}
Collect photos of your classmates
during a greenscreen photoshoot. 
Use Photoshop to cut the people out of the photos 
for use in photomontages.
Use Illustrator to vectorize the cutouts 
for laser cutting.
Deliverables include raster and vector cutouts 
and laser cut models of people and trees.
\emph{Due: 02/14/2020}\\

\noindent \textbf{Participatory Design Toolkit}
As teams design and build a module for a toolkit of laser cut models
for participatory design workshops on complete streets.
Deliverables include a CAD or illustrator drawing
and a laser cut model of your module.
\emph{Due: 03/13/2020}\\

\noindent \textbf{Green Infrastructure}
3D model and 3D render a green infrastructure design
such as a bioswale, flow through planter, or infiltration bosque.
You model should include 3D plants, topography, hardscape, and water.
Layout a poster of your green infrastructure design
to go with the Participatory Design Toolkit.
Deliverables include 3D renderings and a poster.
\emph{Due: 04/08/2020}\\

\noindent \textbf{Participatory Design Toolkit Manual}
As teams layout, compose, and publish 
a manual for your toolkit. 
Your manual should include instructions or rules
for assembling and using the toolkit. 
Consider presenting your toolkit as a serious game
with rules, objectives, and scoring.
Deliverables include a pdf document.
\emph{Due: 04/29/2020}\\

\noindent \textbf{Portfolio}
Collect your work in a course portfolio
for the school's accreditation archive.
\emph{Due: 05/08/2020}\\

% -------------------------------- Grading --------------------------------

\section{Grading}

\begin{table}[H]
\begin{tabular}{l r @{\hskip 2cm} l @{\hskip 0.5cm} l}

Toolkit Characters & 25\% & Green Infrastructure & 25\% \\
Participatory Design Toolkit & 25\% & Portfolio & 5\% \\
Participatory Design Manual & 20\%\\
\end{tabular}
\end{table}

\clearpage

% -------------------------------- Software --------------------------------

\section{Software}
\begin{multicols}{2}
\raggedright
Autocad | \url{www.autodesk.com/‎}\\
Adobe CC | \url{https://www.adobe.com/}\\
Rhinoceros | \url{https://www.rhino3d.com/}\\
Thea Render | \url{www.thearender.com/}\\
Lumion | \url{https://lumion.com}\\
Sketchfab | \url{https://sketchfab.com/}\\
\end{multicols}

% --------------------------------Network --------------------------------

\section{Network drives}

\noindent
Windows for undergrads: \verb|\\desn-knox.lsu.edu\Landscape-Classes\LA2101-S2020| \\
Windows for grads: \verb|\\desn-knox.lsu.edu\Landscape-Classes\LA7102-S2020| \\

\noindent
Mac for undergrads: \verb|smb://desn-knox.lsu.edu/Landscape-Classes/LA2101-S2020| \\
Mac for grads: \verb|smb://desn-knox.lsu.edu/Landscape-Classes/LA7102-S2020| \\

% -------------------------------- Readings --------------------------------

\section{Readings}
\vspace*{0.5cm}
\nocite{*}
\setlength\bibitemsep{0.65\baselineskip}
\printbibliography[heading=none]

\clearpage

% -------------------------------- Policies --------------------------------

\section{Policies}

\noindent \textbf{Communication Intensive}
This is a certified Communication-Intensive (C-I) course,
which meets all of the requirements set
forth by LSU’s Communication across the Curriculum program, including
 instruction and assignments emphasizing
informal and formal visual and technological communication,
teaching of discipline-specific communication techniques,
use of feedback loops for learning,
40\% of the course grade rooted in communication-based work, and
practice of ethical and professional work standards.
Students interested in pursuing the LSU Distinguished Communicators 
certification may use this C-I course for credit. 
For more information about this student recognition program visit: 
\url{www.cxc.lsu.edu}.\\

\noindent \textbf{Accreditation Expectations}
As an accredited Landscape Architecture program
LSU's Robert Reich School of Landscape Architecture (RRSLA)
must meet the accreditation requirements
as stated by the Landscape Architectural Accreditation
Board (LAAB) to ensure RRSLA is meeting the expectations of the field.
The LAAB requires programs to provide digital copies
of student work as part of this process.
Students in this course will be expected
to comply with the following requirements
as 5\% of their course grade:
(1) Students must provide a course portfolio
with work samples specified by the instructor
before the end of the grading period.
(2) Each student's course portfolio must be saved as
a single, high resolution PDF file with multiple pages.
(3) Files must follow the naming convention
established by the school: department-coursenumber-semesteryear-username.pdf.
Example: LA4201-F2019-baharmon.pdf.\\

\noindent \textbf{Time Commitment Expectations}
LSU's general policy states that for each credit hour, you (the student) should plan to
spend at least two hours working on course related activities outside of class. Since this course is for three credit hours, you should expect to spend a minimum of six hours outside of class each week working on assignments for this course. For more information see:
\url{http://catalog.lsu.edu/content.php?catoid=12&navoid=822}.\\

\noindent \textbf{LSU Student Code of Conduct}
The LSU student code of conduct explains student rights, excused absences, and what is expected of student behavior. Students are expected to understand this code:  \url{http://students.lsu.edu/saa/students/code}.\\ %Any violations of the LSU student code will be duly reported to the Dean of Students.\\

\clearpage

\noindent \textbf{Disability Code}
The University is committed to making reasonable efforts to assist individuals with disabilities in
their efforts to avail themselves of services and programs offered by the University. To this end,
Louisiana State University will provide reasonable accommodations for persons with
documented qualifying disabilities. If you have a disability and feel you need accommodations in
this course, you must present a letter to me from Disability Services in 115 Johnston Hall,
indicating the existence of a disability and the suggested accommodations.\\

\noindent \textbf{Academic Integrity}
According to section 10.1 of the LSU Code of Student Conduct, ``A student may be charged with Academic Misconduct'' for a variety of offenses, including: unauthorized copying, collusion, or collaboration; ``falsifying'' data or citations; ``assisting someone in the commission or attempted commission of an offense''; and plagiarism, which is defined in section 10.1.H as a ``lack of appropriate citation, or the unacknowledged inclusion of someone else's words, structure, ideas, or data; failure to identify a source, or the submission of essentially the same work for two assignments without permission of the instructor.''\\

\noindent \textbf{Plagiarism and Citation Method}
Plagiarism is the ``lack of appropriate citation, or the unacknowledged inclusion of someone else's words, structure, ideas, or data; failure to identify a source, or the submission of essentially the same work for two assignments without permission of the instructor(s)'' (Sec. 10.1.H of the LSU Code of Student Conduct). As a student at LSU, it is your responsibility to refrain from plagiarizing the academic property of another and to utilize appropriate citation method for all coursework. In this class, it is recommended that you use Chicago Style author-date citations. Ignorance of the citation method is not an excuse for academic misconduct.

\end{document}
