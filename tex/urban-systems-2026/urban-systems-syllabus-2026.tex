%!TEX TS-program = xelatex
%!TEX encoding = UTF-8 Unicode

% Class
\documentclass[11pt,article,oneside]{memoir}

% Packages
\usepackage{../style/xelatex-preamble}

% Settings
\AtBeginBibliography{\small}

% Definitions
\def\myauthor{Author}
\def\mytitle{Title}
\def\mycopyright{\myauthor}
\def\mykeywords{}
\def\mybibliostyle{plain}
\def\mybibliocommand{}
\def\mysubtitle{}
\def\myaffiliation{Louisiana State University}
\def\myemail{baharmon@lsu.edu} 
\def\myweb{https://baharmon.github.io/}
\def\myphone{919.622.8414}
\def\myversion{}
\def\myrevision{}
\def\myaffiliation{\ \\Louisiana State University}
\def\myauthor{Brendan Harmon}
\def\mykeywords{Landscape Architecture, Syllabus, Graduate}
\def\mysubtitle{Syllabus}
\def\mytitle{ \includegraphics[width=6cm]{../logos/lsu_art_design_logo.pdf} \\
[0.1cm] {\normalfont \normalsize LA 7041 |} \Large Urban Systems} 

% Color
\makeatletter
\newcommand{\globalcolor}[1]{%
  \color{#1}\global\let\default@color\current@color
}
\makeatother

% Begin
\begin{document}

% Fonts
\defaultfontfeatures{}
\defaultfontfeatures{Scale=MatchLowercase}
\setmainfont{IBM Plex Sans}
\setmonofont[Scale=0.8]{IBM Plex Mono}

% Style
\setlength\bibitemsep{0.5em}
\def\ind{\hangindent=1 true cm\hangafter=1 \noindent}
\def\labelitemi{$\cdot$}
\chapterstyle{syllabus} 

% Frontmatter
\title{\LARGE \mytitle}
\author{\Large\myauthor \newline \footnotesize\texttt{\noindent\myemail}}
\date{Spring 2026 \newline Monday, Wednesday, \& Friday \newline 12:30--16:20 \newline Design 304}
\published{\,}

% -------------------------------- Cover page -------------------------------- 

%\pagenumbering{gobble}
%\globalcolor{black}
%\vspace*{-10em}
%\maketitle
%\ThisCenterWallPaper{1.1}{../images/.png}
%\clearpage

% -------------------------------- Description -------------------------------- 

\pagenumbering{arabic}
\globalcolor{black}

\vspace*{-10em}
\maketitle

\section{Course Description}

The City of Lake Charles 
in Calcasieu Parish 
in southwestern Louisiana
is threatened by coastal flooding, 
suffers from extreme heat in the summers,
and has more than 670 vacant and adjudicated properties. 
The regional masterplan -- 
\href{https://www.visitlakecharles.org/just-imagine-swla/}
{Just Imagine SWLA: 50-Year Resilience Master Plan for Calcasieu and Cameron Parishes}
-- calls for 
\href{https://www.visitlakecharles.org/just-imagine-swla/projects/underutilized-property/}
{strategies to address underutilized property}.
This studio explores how 
underutilized streets and properties
can be transformed into an urban forest
that shelters and cools the city, 
captures and filters stormwater,
supports biodiversity, 
and brings beauty into 
underinvested neighborhoods.
What species of trees should be planted where?
Where will all of these trees come from?
Can native species be sourced locally?
How can they be planted and managed for resilience to storms? 
What ecosystems services can they provide?
What other species can they host and feed?
How will the urban forest canopy evolve over time?
This studio is a collaboration 
with a graduate architecture studio
transforming underutilized properties
into mixed used developments. 

% -------------------------------- Schedule -------------------------------- 
\section{Schedule}

\begin{table}[H]
\begin{tabular}{@{} l @{\hskip 10em} l @{\hskip 10em} l}
\textbf{Planning} & \textbf{Streetscape} & \textbf{Site Design}\\
\end{tabular}
\end{table}

\vspace*{-1em}

\begin{table}[H]
\small
\begin{tabular}{@{} l l l l l l}
\small
\textbf{1} & Planning I \hspace{7em}  & \textbf{6} & Streetscape I \hspace{7em} & \textbf{11} & Site I\\
\textbf{2} & Planning II & \textbf{7} & Streetscape II & \textbf{12} & Site II\\
\textbf{3} & Planning III & \textbf{8} & Streetscape III & \textbf{13} & Site III\\
\textbf{4} & Planning IV & \textbf{9} & Streetscape IV & \textbf{14} & Site IV\\
\textbf{5} & Planning V & \textbf{10} & Streetscape V & \textbf{15} & Site V\\
 & \textbf{Review} & & \textbf{Review} & & \textbf{Review}\\
\end{tabular}
\end{table}

\clearpage

% -------------------------------- Projects -------------------------------- 

\section{Project}

\paragraph{Masterplan}
As a class, develop a masterplan 
for establishing an urban forest 
in the City of Lake Charles.
The masterplan should include strategies
for greening underutilized properties and streetscapes. 
Consider what trees should be planted,
which nurseries could supply them, 
and how densely they should spaced. 
Deliverables include maps, diagrams, 
a masterplan, a planting scheme,
and conceptual sections. 
\textbf{Due:} Week 5

\paragraph{Streetscape}
In small teams, design green, tree lined streetscapes 
that were identified as priorities in your masterplan. 
These streets should be lined with underutilized properties
that can be be connected into a forested network. 
The streetscape should enhance mobility, calm traffic, 
manage and filter stormwater, create habitat, 
and give its neighborhood a unique aesthetic. 
Design in detail -- consider paving patterns, curbs,
site furniture, tree pits, planters, and bioswales. 
Deliverables include diagrams, 
a plan, a planting plan, a paving plan,
sections, a set of details,
and a laser cut model. 
\textbf{Due:} Week 10

\paragraph{Site Design}
Design an intervention for an underutilized property
along your team's streetscape. 
Transform this lot into a patch of urban forest. 
Consider concepts such as food forests, tiny forests, 
Miyawaki forests, and phytoremediation. 
Deliverables include a plan, a planting plan, a paving plan, 
ssections, perspectives, and diagrams. 
\textbf{Due:} Week 15

% -------------------------------- Grading -------------------------------- 

\section{Grading}
%
\begin{table}[H]
\begin{tabular}{ @{} l @{\hskip 0.75em} l @{\hskip 3em} l @{\hskip 0.75em} l @{\hskip 3em} l @{\hskip 0.75em} l @{\hskip 3em} l @{\hskip 0.75em} l}
\textbf{Masterplan} & 32\% & \textbf{Streetscape} & 32\% & \textbf{Site Design} & 32\% & \textbf{Portfolio} & 4\%\\
\end{tabular}
\end{table}

% -------------------------------- Logistics -------------------------------- 

\section{Logistics}

We will meet on Mondays, Wednesdays, and Fridays 
from 12:30--16:20
in Design 304. 
Our Discord \href{https://discord.gg/S3W8npES5A}{server}
will be used for posting announcements, 
updates, projects, and troubleshooting. 
\\

\noindent
Server | \url{https://discord.gg/S3W8npES5A}\\
Videos | \url{https://www.youtube.com/@baharmon}

\clearpage

% -------------------------------- Software -------------------------------- 
\section{Software}

ArcGIS | \url{https://www.esri.com/}\\
Rhinoceros | \url{https://www.rhino3d.com}\\
Lumion | \url{https://lumion.com/}\\
Adobe Creative Cloud | \url{https://www.adobe.com}

% -------------------------------- Resources -------------------------------- 
\section{Resources}

SWLA Masterplan | \url{https://www.visitlakecharles.org/just-imagine-swla}\\ 
National Association of City Transportation Officials | \url{https://nacto.org}\\
Calcasieu Parish GIS | \url{https://gis.calcasieu.gov}\\
Calcasieu Parish Interactive Property Map | \url{https://cppj.totaland.com}\\
City of Lake Charles Interactive Property Map | \url{http://colc.totaland.com}\\
NOAA Digital Coast | \url{https://coast.noaa.gov/digitalcoast}

% -------------------------------- Readings -------------------------------- 
\section{Required Readings}
\vspace*{0.5cm}
\nocite{*}
\setlength\bibitemsep{0.65\baselineskip}
\printbibliography[keyword=required, heading=none]

% -------------------------------- Readings -------------------------------- 
\section{Recommended Readings}
\vspace*{0.5cm}
\printbibliography[notkeyword=required, heading=none]

% -------------------------------- Policies -------------------------------- 

\section{Policies}

\paragraph{Time Commitment Expectations}
LSU's general policy states that for each credit hour, 
you (the student) should plan to
spend at least two hours working 
on course related activities outside of class. 
Since this studio is for six credit hours, 
you should expect to spend a minimum of twelve hours 
outside of class each week 
working on assignments for this course. 
For more information see: 
\url{https://catalog.lsu.edu}.

\paragraph{LSU student code of conduct}
The LSU student code of conduct explains 
student rights, excused absences, 
and what is expected of student behavior. 
Students are expected to understand this code:  
\url{https://www.lsu.edu/saa/students/codeofconduct.php}.

\paragraph{Disability Code}
The University is committed to making reasonable efforts 
to assist individuals with disabilities in
their efforts to avail themselves of 
services and programs offered by the University. 
Louisiana State University will provide 
reasonable accommodations for persons
with documented qualifying disabilities.
If you have a disability and 
feel you need accommodations in this course, 
you must present a letter 
from Disability Services in 115 Johnston Hall,
indicating the existence of a disability 
and the suggested accommodations.

\paragraph{Academic Integrity}
According to section 10.1 
of the LSU Code of Student Conduct, 
``A student may be charged with Academic Misconduct'' 
for a variety of offenses, including the following: 
unauthorized copying, collusion, or collaboration,
``falsifying'' data or citations,
``assisting someone in the commission 
or attempted commission of an offense''; 
and plagiarism.

\paragraph{Plagiarism and Citation Method}
Plagiarism is the 
``lack of appropriate citation, or the unacknowledged inclusion 
of someone else's words, structure, ideas, or data; 
failure to identify a source, 
or the submission of essentially the same work 
for two assignments without permission of the instructor(s)''
(Sec. 10.1.H of the LSU Code of Student Conduct). 
As a student at LSU, 
it is your responsibility to refrain from 
plagiarizing the academic property of another 
and to utilize the appropriate citation method for all coursework. 
In this class, it is recommended that you use 
Chicago Style author-date citations. 
Ignorance of the citation method
is not an excuse for academic misconduct.

\paragraph{Generative Artificial Intelligence}
In this course the use of generative artificial intelligence (AI) 
is permitted for the purposes of enhancing your understanding 
of course materials, encouraging creative exploration, 
and supporting academic growth. 
The use of generative AI, however, 
is not permitted for writing assignments. 
These programs should not be used to produce work 
that misrepresents your abilities 
or deceives as to the conditions 
under which the work was completed. 
If you use AI to generate content 
you must clearly acknowledge the use of AI generated material. 
Proper attribution of AI program use
should include an explanation of how the program 
contributed to the assignment and your academic growth. 
Failing to give proper attribution 
to the use of AI programs in academic work
 will be reported to Student Advocacy \& Accountability 
 for review under the Code of Student Conduct 
 and may result in impacts to your assignment and course grades.

\paragraph{Accreditation}
LSU's Robert Reich School of Landscape Architecture (RRSLA) 
must meet the accreditation requirements 
as stated by the Landscape Architectural Accreditation
Board (LAAB) to ensure the school
 is meeting the expectations of the field. 
The LAAB requires programs to provide digital copies 
of student work as part of this process.
Students in this course will be expected 
to comply with the following requirements
as 4\% of their course grade: 
(1) Students must provide a course portfolio
with work samples specified by the instructor 
before the end of the grading period. 
(2) Each student's course portfolio must be saved as 
a single, high resolution PDF file with multiple pages. 
(3) Files must follow the naming convention
established by the school: 
department-coursenumber-semesteryear-username.pdf.
Example: LA7504-S2026-baharmon.pdf.

\end{document}
