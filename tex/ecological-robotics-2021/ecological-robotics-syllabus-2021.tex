%!TEX TS-program = xelatex
%!TEX encoding = UTF-8 Unicode

%%%  Syllabus template for use with style files at http://kjhealy.github.com/latex-custom-kjh
%%%  Kieran Healy

\documentclass[11pt,article,oneside]{memoir}

% packages
\usepackage{org-preamble-xelatex}
\usepackage{wallpaper}
\usepackage{xcolor}
\usepackage{multicol}
\usepackage{enumitem}
\setlist[itemize]{leftmargin=*}

\AtBeginBibliography{\small}

% Definitions
\def\myauthor{Author}
\def\mytitle{Title}
\def\mycopyright{\myauthor}
\def\mykeywords{}
\def\mybibliostyle{plain}
\def\mybibliocommand{}
\def\mysubtitle{}
\def\myaffiliation{Louisiana State University}
\def\myaddress{Design 304}
\def\myemail{baharmon@lsu.edu}
\def\myweb{https://baharmon.github.io/}
\def\myphone{919.622.8414}
\def\myversion{}
\def\myrevision{}
\def\myaffiliation{\ \\Louisiana State University}
\def\myauthor{\normalsize Brendan Harmon}
\def\mykeywords{Landscape Architecture, Syllabus, Graduate}
\def\mysubtitle{Syllabus}
\def\mytitle{
\includegraphics[height=1cm]{../logos/lsu_art_design_logo.pdf}
\vspace*{0.5cm}
\\[0.1cm] {\normalfont \normalsize LA 4008 \& LA 7051 |} \Large Ecological Robotics
} 

% color
\makeatletter
\newcommand{\globalcolor}[1]{%
  \color{#1}\global\let\default@color\current@color
}
\makeatother

% begin
\begin{document}

\setlength\bibitemsep{0.5em}

% fonts
\defaultfontfeatures{}
\defaultfontfeatures{Scale=MatchLowercase}         
\setmainfont{Lato Regular}
\setmonofont[Scale=0.8]{IBM Plex Mono}

\def\ind{\hangindent=1 true cm\hangafter=1 \noindent}
\def\labelitemi{$\cdot$}
\chapterstyle{article-4-sans}  

\title{\LARGE \mytitle}
\author{\Large\myauthor \newline \footnotesize\texttt{\noindent\myemail}}
\date{ \footnotesize Design 304 
\newline Monday, Wednesday, \& Friday
\newline 1:30pm--5:30pm
\newline Fall 2021}
\published{\,}

% -------------------------------- COVER PAGE -------------------------------- 

\pagenumbering{gobble}
\globalcolor{black}
\vspace*{-10em}
\maketitle
\vspace*{1cm}
\ThisCenterWallPaper{1}{../images/robotic-garden-01.png}
\clearpage


% -------------------------------- DESCRIPTION -------------------------------- 

\pagenumbering{arabic}
\globalcolor{black}

\vspace*{-10em}
\maketitle

\section{Course Description}
This studio will explore ecological applications for robots. In this studio you will learn how to program industrial robots, design your own robotic tools, and robotically plant a landscape. Through a series of projects you will design a robotic process for algorithmic planting and test it both in the studio and in the field. The field site will be a set of plots at Burden where you will conduct a controlled planting experiment. As an introduction to robotics for designers, this studio will cover topics such as visual programming, generative design, robotic operations, 3D printing, 3D scanning, and drone photogrammetry. \\

% -------------------------------- SCHEDULE -------------------------------- 
\section{Topics}
%
\begin{table}[H]
\begin{tabular}{l @{\hskip 1.8cm} l @{\hskip 2.3cm} l}
\textbf{Robotic Sandbox} & \textbf{Robotic Planting} & \textbf{Robotic Landscape}\\
\end{tabular}
\end{table}
%
\vspace*{-1em}
%
\begin{table}[H]
\begin{tabular}{l l l l l l}
\small
\textbf{1} & Robotic Operations  & \textbf{6} & 3D Printing Seeds & \textbf{11} & Site Preparation\\
\textbf{2} & Robotic Programming & \textbf{7} & Laboratory Planting & \textbf{12} & Field Planting\\
\textbf{3} &  Robotic Tools & \textbf{8} & Algorithmic Seeding & \textbf{13} & Drone Mapping\\
\textbf{4} & Visual Programming & \textbf{9} & Algorithmic Landforms & \textbf{14} & Documentation\\
\textbf{5} & Robotic Sandbox & \textbf{10} & 3D Scanning & \textbf{15} & Exhibition\\
\end{tabular}
\end{table}

\clearpage

% -------------------------------- Projects -------------------------------- 
\section{Projects}

Post your project work on Discord: 
\url{https://discord.gg/sn92G8sJSb}\\

\noindent \textbf{Robotic Sandbox}
Design and fabricate a customized robotic tool
for sculpting sand. 
Use a collaborative, industrial robotic arm
to sculpt algorithmically generated terrain
in a sandbox. 
Record your models with 3D scans.
\\

\noindent \textbf{Robotic planting}
Design, fabricate, and test a robotic process for planting.
In the studio use a collaborative, industrial robotic arm
to plant a set of trays with seeds. 
Experiment with different algorithmic planting patterns. 
Record your plants as they grow with a series of 3D scans.
\\

\noindent \textbf{Robotic Landscapes}
Plant a set of square meter plots at Burden
as a controlled field experiment
to compare hand sown seeds with robotic seeding.
Use drone photogrammetry
to record the start of the experiment.
\\

% -------------------------------- Hardware -------------------------------- 
\section{Hardware}
UR10e Robot | \url{https://www.universal-robots.com/products/ur10-robot}\\
UR5e Robot | \url{https://www.universal-robots.com/products/ur5-robot}\\
Faro Focus 3D | \url{https://www.faro.com/en/Products/Hardware/Focus-Laser-Scanners}\\
DJI Matrice 600 Pro | \url{https://www.dji.com/matrice600-pro}\\
FireFly 6 Pro | \url{https://www.birdseyeview.aero/products/firefly6}\\
3D PotterBot | \url{https://3dpotter.com/}\\

% -------------------------------- Software -------------------------------- 
\section{Software}
Rhinoceros | \url{https://www.rhino3d.com/}\\
Machina | \url{https://github.com/RobotExMachina}\\
Metashape | \url{https://www.agisoft.com/}\\
Scene | \url{https://www.faro.com/}\\

% -------------------------------- Grading -------------------------------- 
\section{Grading}
%
\begin{table}[H]
%\small
\begin{tabular}{l r @{\hskip 2cm} l @{\hskip 0.5cm} l}
%\begin{tabular}{l l}
%
Robotic Sandbox & 30\% & Robotic Landscape & 30\% \\
Robotic Planting & 35\%  & Course Portfolio & 5\% \\
%
\end{tabular}
\end{table}

%\clearpage

% -------------------------------- Resources -------------------------------- 
\section{Resources}
Grasshopper Primer | \url{http://grasshopperprimer.com}\\
Machina Tutorials | \url{https://www.youtube.com/c/garciadelcastillo}

% -------------------------------- Readings -------------------------------- 
\section{Readings}
\vspace*{0.5cm}
\nocite{*}
\setlength\bibitemsep{0.65\baselineskip}
\printbibliography[heading=none]
\clearpage

% -------------------------------- Policies -------------------------------- 
\section{Policies}

\noindent \textbf{Time Commitment Expectations}
LSU's general policy states that for each credit hour, you (the student) should plan to
spend at least two hours working on course related activities outside of class. Since this course is for three credit hours, you should expect to spend a minimum of six hours outside of class each week working on assignments for this course. For more information see: 
\url{http://catalog.lsu.edu/content.php?catoid=12&navoid=822}.\\

\noindent \textbf{LSU student code of conduct}
The LSU student code of conduct explains student rights, excused absences, and what is expected of student behavior. Students are expected to understand this code:  \url{http://students.lsu.edu/saa/students/code}.\\ %Any violations of the LSU student code will be duly reported to the Dean of Students.\\

\noindent \textbf{Disability Code}
The University is committed to making reasonable efforts to assist individuals with disabilities in
their efforts to avail themselves of services and programs offered by the University. To this end,
Louisiana State University will provide reasonable accommodations for persons with
documented qualifying disabilities. If you have a disability and feel you need accommodations in
this course, you must present a letter to me from Disability Services in 115 Johnston Hall,
indicating the existence of a disability and the suggested accommodations.\\

\noindent \textbf{Academic Integrity}
According to section 10.1 of the LSU Code of Student Conduct, ``A student may be charged with Academic Misconduct'' for a variety of offenses, including the following: unauthorized copying, collusion, or collaboration; ``falsifying'' data or citations; ``assisting someone in the commission or attempted commission of an offense''; and plagiarism, which is defined in section 10.1.H as a ``lack of appropriate citation, or the unacknowledged inclusion of someone else's words, structure, ideas, or data; failure to identify a source, or the submission of essentially the same work for two assignments without permission of the instructor(s).''\\

\noindent \textbf{Plagiarism and Citation Method}
Plagiarism is the ``lack of appropriate citation, or the unacknowledged inclusion of someone else's words, structure, ideas, or data; failure to identify a source, or the submission of essentially the same work for two assignments without permission of the instructor(s)'' (Sec. 10.1.H of the LSU Code of Student Conduct). As a student at LSU, it is your responsibility to refrain from plagiarizing the academic property of another and to utilize appropriate citation method for all coursework. In this class, it is recommended that you use Chicago Style author-date citations. Ignorance of the citation method is not an excuse for academic misconduct.

\end{document}
