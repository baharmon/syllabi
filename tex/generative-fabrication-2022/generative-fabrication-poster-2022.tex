%!TEX TS-program = xelatex
%!TEX encoding = UTF-8 Unicode

\documentclass[11pt,article,oneside]{memoir}

% packages
\usepackage{org-preamble-xelatex}
\usepackage{wallpaper}
\usepackage{xcolor}
\usepackage{multicol}
\usepackage{enumitem}
\setlist[itemize]{leftmargin=*}
\usepackage{tikz}
\usepackage{tikzpagenodes} 

\AtBeginBibliography{\small}

% Definitions
\def\myauthor{Author}
\def\mytitle{Title}
\def\mycopyright{\myauthor}
\def\mykeywords{}
\def\mybibliostyle{plain}
\def\mybibliocommand{}
\def\mysubtitle{}
\def\myaffiliation{Louisiana State University}
\def\myaddress{309 Design}
\def\myemail{baharmon@lsu.edu} 
\def\myweb{https://baharmon.github.io/}
\def\myphone{919.622.8414}
\def\myversion{}
\def\myrevision{}
\def\myaffiliation{\ \\Louisiana State University}
\def\myauthor{Brendan Harmon}
\def\mykeywords{Landscape Architecture, Syllabus, Graduate, Undergraduate}
\def\mysubtitle{Syllabus}
\def\mytitle{\includegraphics[width=6cm]{../logos/lsu_art_design_logo.pdf}\\
[0.1cm]{\Large Generative Fabrication} \\ 
[-0.2cm]{\normalfont \normalsize LA 4008 \& LA 7051}} 

% color
\makeatletter
\newcommand{\globalcolor}[1]{%
  \color{#1}\global\let\default@color\current@color
}
\makeatother

% begin
\begin{document}

\setlength\bibitemsep{0.5em}

% fonts
\defaultfontfeatures{}
\defaultfontfeatures{Scale=MatchLowercase} 
\setmainfont{Lato Regular}
\setmonofont[Scale=0.8]{IBM Plex Mono}

\def\ind{\hangindent=1 true cm\hangafter=1 \noindent}
\def\labelitemi{$\cdot$}
\chapterstyle{article-4-sans}  

\title{\LARGE \mytitle}
\author{\Large\myauthor \newline \footnotesize\texttt{\noindent\myemail}}
\date{Fall 2022 Design 304.\newline Monday, Wednesday, \& Friday\newline 1:30pm--5:20pm.}
\published{\,}


% -------------------------------- DESCRIPTION -------------------------------- 

\pagenumbering{gobble}
\globalcolor{black}
\vspace*{-10em}
\maketitle

\section{Course Description}

In this studio, you will design and build
a living wall for LSU's Hill Farm.
The Hill Farm --
an organic farm with orchards and greenhouses
for teaching horticulture --
has commissioned us to design and build a green wall
with generous funding by the Student Sustainability Fund.
This studio will explore
ecological design principles,
computational design strategies,
digital fabrication techniques, and
the use of ceramics as a building material.
In this studio you will learn how to use Grasshopper
to design complex structures
composed of non-standard modules.
You will learn how to fabricate unique components
using computer numerical control machining
and clay 3D printing.
As a design-build studio with a tight schedule,
this will be a team project entailing group work,
the design process will be streamlined,
and much of the course will be spent
fabricating modules of the green wall
and preparing construction documents.
For those who wish to see the project through
and earn a salary as well as academic credit,
an elective will be offered in the spring semester
to install and plant the wall on site at the Hill Farm.
\\

\begin{center}
\resizebox{\textwidth}{!}{%
\includegraphics[height=3cm]{../images/robotic-inform-1.jpg}%
\enspace
\includegraphics[height=3cm]{../images/robotic-inform-2.jpg}}
\end{center}

\end{document}
