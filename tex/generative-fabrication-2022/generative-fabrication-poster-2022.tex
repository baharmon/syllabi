%!TEX TS-program = xelatex
%!TEX encoding = UTF-8 Unicode

%%%  Syllabus template for use with style files at http://kjhealy.github.com/latex-custom-kjh
%%%  Kieran Healy

\documentclass[11pt,article,oneside]{memoir}

% packages
\usepackage{org-preamble-xelatex}
\usepackage{wallpaper}
\usepackage{xcolor}
\usepackage{multicol}
\usepackage{enumitem}
\setlist[itemize]{leftmargin=*}

\AtBeginBibliography{\small}

% Definitions
\def\myauthor{Author}
\def\mytitle{Title}
\def\mycopyright{\myauthor}
\def\mykeywords{}
\def\mybibliostyle{plain}
\def\mybibliocommand{}
\def\mysubtitle{}
\def\myaffiliation{Louisiana State University}
\def\myaddress{309 Design}
\def\myemail{baharmon@lsu.edu} 
\def\myweb{https://baharmon.github.io/}
\def\myphone{919.622.8414}
\def\myversion{}
\def\myrevision{}
\def\myaffiliation{\ \\Louisiana State University}
\def\myauthor{Brendan Harmon}
\def\mykeywords{Landscape Architecture, Syllabus, Graduate, Undergraduate}
\def\mysubtitle{Syllabus}
\def\mytitle{ \includegraphics[width=6cm]{../../logos/lsu_art_design_logo.pdf} \\[0.1cm] {\normalfont \normalsize LA 4008 + 7051 |} \Large Living Structure Studio}

% color
\makeatletter
\newcommand{\globalcolor}[1]{%
  \color{#1}\global\let\default@color\current@color
}
\makeatother

% begin
\begin{document}

\setlength\bibitemsep{0.5em}

% fonts
\defaultfontfeatures{}
\defaultfontfeatures{Scale=MatchLowercase}         
\setmainfont[Scale=1, Path = ../../fonts/lato/,BoldItalicFont=Lato-RegIta,BoldFont=Lato-Reg,ItalicFont=Lato-LigIta]{Lato-Lig}
\setsansfont[Scale=1, Path = ../../fonts/lato/,BoldItalicFont=Lato-RegIta,BoldFont=Lato-Reg,ItalicFont=Lato-LigIta]{Lato-Lig}
\setmonofont[Mapping=tex-text,Scale=0.8,Path = ../../fonts/inconsolata/]{i}
\newfontfamily\icon[Scale=1, Path = ../../fonts/fontawesome/]{fontawesome-regular-400}

\def\ind{\hangindent=1 true cm\hangafter=1 \noindent}
\def\labelitemi{$\cdot$}
\chapterstyle{article-4-sans}  

\title{\LARGE \mytitle}
\author{\Large\myauthor \newline \footnotesize\texttt{\noindent\myemail}}
\date{Fall 2020 Design 324.\newline Monday, Wednesday, \& Friday\newline 1:30pm--5:20pm.}
\published{\,}

% -------------------------------- DESCRIPTION -------------------------------- 

\pagenumbering{arabic}
\globalcolor{black}
\vspace*{-10em}
\maketitle

\section{Course Description}

Digitally design and fabricate a prototype of a living green wall. 
This interdisciplinary studio will explore 
non standard construction 
in architecture and landscape architecture
using generative processes 
for the design and fabrication 
of complex living structures. 
In the first section of this studio you will 
learn how to program 3D printers and industrial robots 
and then design, render, and fabricate 
a small ceramic form.  
In the second section you will design 
a ceramic structure for a free standing living green wall.
Use a generative design process to model 
and analyze a family of variations on your design.
Fabricate scale models of your designs.  
In the third section of the course 
you will make a detailed design, renderings,  
and construction documentation
for your living wall. 
Fabricate and plant a module of your living wall.
This studio will be conducted online 
with either remote or in person access 
to digital fabrication tools such as 
fused deposition modeling (FDM) ,
stereolithography (SLA), and
ceramic 3D printers
and industrial robots. \\

\begin{figure}[h!]
\center
\includegraphics[width=0.332\textwidth]{../images/robotic-inform-1.jpg}
\includegraphics[width=0.59\textwidth]{../images/robotic-inform-2.jpg}
\end{figure}


\end{document}
