%!TEX TS-program = xelatex
%!TEX encoding = UTF-8 Unicode

\documentclass[11pt,article,oneside]{memoir}

% packages
\usepackage{org-preamble-xelatex}
\usepackage{wallpaper}
\usepackage{xcolor}
\usepackage{multicol}
\usepackage{enumitem}
\setlist[itemize]{leftmargin=*}
\usepackage{tikz}
\usepackage{tikzpagenodes} 

\AtBeginBibliography{\small}

% Definitions
\def\myauthor{Author}
\def\mytitle{Title}
\def\mycopyright{\myauthor}
\def\mykeywords{}
\def\mybibliostyle{plain}
\def\mybibliocommand{}
\def\mysubtitle{}
\def\myaffiliation{Louisiana State University}
\def\myaddress{309 Design}
\def\myemail{baharmon@lsu.edu} 
\def\myweb{https://baharmon.github.io/}
\def\myphone{919.622.8414}
\def\myversion{}
\def\myrevision{}
\def\myaffiliation{\ \\Louisiana State University}
\def\myauthor{Brendan Harmon}
\def\mykeywords{Landscape Architecture, Syllabus, Graduate, Undergraduate}
\def\mysubtitle{Syllabus}
\def\mytitle{\includegraphics[width=6cm]{../logos/lsu_art_design_logo.pdf}\\
[0.1cm]{\Large Generative Fabrication} \\ 
[-0.2cm]{\normalfont \normalsize LA 4008 \& LA 7051}} 

% color
\makeatletter
\newcommand{\globalcolor}[1]{%
  \color{#1}\global\let\default@color\current@color
}
\makeatother

% begin
\begin{document}

\setlength\bibitemsep{0.5em}

% fonts
\defaultfontfeatures{}
\defaultfontfeatures{Scale=MatchLowercase}         
\setmainfont[Scale=1, Path = ../fonts/lato/,BoldItalicFont=Lato-BolIta,BoldFont=Lato-Bol,ItalicFont=Lato-RegIta]{Lato-Reg}
\setmainfont[Scale=1, Path = ../fonts/lato/,BoldItalicFont=Lato-BolIta,BoldFont=Lato-Bol,ItalicFont=Lato-RegIta]{Lato-Reg}
\setmonofont[Mapping=tex-text,Scale=0.8,Path = ../fonts/inconsolata/]{i}
\newfontfamily\icon[Scale=1, Path = ../fonts/fontawesome/]{fontawesome-regular-400}

\def\ind{\hangindent=1 true cm\hangafter=1 \noindent}
\def\labelitemi{$\cdot$}
\chapterstyle{article-4-sans}  

\title{\LARGE \mytitle}
\author{\Large\myauthor \newline \footnotesize\texttt{\noindent\myemail}}
\date{Fall 2022 Design 304.\newline Monday, Wednesday, \& Friday\newline 1:30pm--5:20pm.}
\published{\,}

% -------------------------------- COVER PAGE -------------------------------- 

\pagenumbering{gobble}
\globalcolor{black}
\begin{tikzpicture}[remember picture,overlay]
\fill[white] ([xshift=-4.5cm,yshift=4cm]current page text area.west) rectangle (7,3);
\end{tikzpicture}
\vspace*{-10em}
\maketitle
\ThisCenterWallPaper{1.1}{../images/robotic-inform-1.jpg}
\clearpage

% -------------------------------- DESCRIPTION -------------------------------- 

\pagenumbering{arabic}
\globalcolor{black}
\vspace*{-10em}
\maketitle

\section{Course Description}

This studio will 
the design and construction of a living wall
for the LSU Hill Farm.

%Student Sustainability Fund
...
\\

%This studio will explore non standard construction 
%in architecture and landscape architecture.
%You will use generative processes 
%for the design and fabrication 
%of complex architectural forms. 
%In the first section of this studio you will 
%learn how to program 3D printers and industrial robots 
%and then design, render, and fabricate 
%a small ceramic vessel.  
%In the second section you will design 
%a ceramic structure for a free standing green wall.
%Use a generative design process to model 
%and analyze a family of variations on your design.
%Fabricate scale models of your designs.  
%In the third section of the course 
%you will make a detailed design, renderings,  
%and construction documentation
%for your living wall. 
%\\

% -------------------------------- SCHEDULE ----------------------------- 
\section{Topics}
%
\begin{table}[H]
\begin{tabular}{l @{\hskip 3.5cm} l @{\hskip 2.2cm} l}
\textbf{Prototyping} & \textbf{Fabrication} & \textbf{Documentation}\\
\end{tabular}
\end{table}
%
\vspace*{-1em}
%
\begin{table}[H]
\begin{tabular}{l l l l l l}
\small
\textbf{1} & Visual Programming  & \textbf{6} & Module Design & \textbf{11} & Firing \& Glazing\\
\textbf{2} & 3D Printed Ceramics & \textbf{7} & Mold Design & \textbf{12} & 3D Scanning\\
\textbf{3} & Parametric Wall Design & \textbf{8} & Mold Fabrication & \textbf{13} & Documentation I\\
\textbf{4} & Subtractive Manufacturing & \textbf{9} & Casting & \textbf{14} & Documentation II\\
\textbf{5} & Slip Casting & \textbf{10} & Assembly & \textbf{15} & Exhibition\\
\end{tabular}
\end{table}

\clearpage

% -------------------------------- Projects ------------------------------ 

\section{Projects}
%Design and build a modular living wall.
%\\

\noindent \textbf{Module Prototype}
Build a full-scale prototype of a wall module
and 3D print a scale model of the wall.
Assemble the prototypes, 
install plants, 
and document their growth.
\\

\noindent \textbf{Module Fabrication}
Design and fabricate ceramic modules for the wall assembly
using CNC machining and slip casting.
\\

\noindent \textbf{Documentation}
Prepare construction documents,
posters, and an exhibition for the living wall.
Documentation should include 
construction drawings, 3D scans, 3D renderings,
plant lists, materials, schedules, and cost estimates.
\\

\noindent \textbf{Course Portfolio}
Collect your work in a course portfolio 
for the school's accreditation archive.
%\emph{Due: 12/11/2022}
\\

\noindent \textbf{Construction}
The living wall will be built at the LSU Hill Farm in Spring 2023.
To participate and earn a salary, enroll in our Design \& Build elective.

% -------------------------------- Grading -------------------------------- 
\section{Grading}
\vspace*{-0.4cm}
\begin{table}[H]
\begin{tabular}{@{}l r @{\hskip 2cm} l @{\hskip 0.5cm} l}
Module Prototype & 30\% & Documentation & 35\% \\
Module Fabrication & 30\% & Course Portfolio & 5\% \\
\end{tabular}
\end{table}

% -------------------------------- Community ----------------------------- 
\section{Community}
Discord | \url{https://discord.gg/eYKrsahQtW}

% -------------------------------- Software -------------------------------- 
\section{Software}
%\vspace*{-0.4cm}
\begin{table}[H]
\begin{tabular}{@{}l l}
Rhinoceros | \url{https://www.rhino3d.com/} &
RhinoCAM  | \url{https://mecsoft.com/}\\
Thea Render | \url{https://www.thearender.com/} &
Scene | \url{https://www.faro.com/} \\

\end{tabular}
\end{table}
%Rhinoceros | \url{https://www.rhino3d.com/}\\
%Thea Render for Rhino | \url{https://www.thearender.com/}\\
%RhinoCAM | \url{https://mecsoft.com/rhinocam/}\\
%Scene | \url{https://www.faro.com/}

% -------------------------------- Equipment ------------------------------ 
\section{Equipment}
\textbf{FabLab} | 3D PotterBot \& CNC Router\\
\textbf{ART 334} | 3D PotterBot Micro 9 \& Kiln

% -------------------------------- Readings -------------------------------- 
\section{Readings}
\vspace*{0.5cm}
\nocite{*}
\setlength\bibitemsep{0.65\baselineskip}
\printbibliography[heading=none]


% -------------------------------- Policies -------------------------------- 
\section{Policies}

\noindent \textbf{Accreditation Expectations}
As an accredited Landscape Architecture program
LSU's Robert Reich School of Landscape Architecture (RRSLA) 
must meet the accreditation requirements 
as stated by the Landscape Architectural Accreditation
Board (LAAB) to ensure RRSLA is meeting the expectations of the field. 
The LAAB requires programs to provide digital copies 
of student work as part of this process.
Students in this course will be expected 
to comply with the following requirements
as 5\% of their course grade: 
(1) Students must provide a course portfolio
with work samples specified by the instructor 
before the end of the grading period. 
(2) Each student's course portfolio must be saved as 
a single, high resolution PDF file with multiple pages. 
(3) Files must follow the naming convention
established by the school: department-coursenumber-semesteryear-username.pdf.
Example: LA7051-F2022-baharmon.pdf.
\\

\clearpage

\noindent \textbf{Time Commitment Expectations}
LSU's general policy states that for each credit hour, you -- the student -- should plan to
spend at least two hours working on course related activities outside of class. Since this course is for six credit hours, you should expect to spend a minimum of twelve hours outside of class each week working on assignments for this course. For more information see: 
\url{http://catalog.lsu.edu/content.php?catoid=12&navoid=822}.
\\

\noindent \textbf{LSU student code of conduct}
The LSU student code of conduct explains student rights, excused absences, and what is expected of student behavior. Students are expected to understand this code:  \url{http://students.lsu.edu/saa/students/code}.
\\ %Any violations of the LSU student code will be duly reported to the Dean of Students.\\

\noindent \textbf{Disability Code}
The University is committed to making reasonable efforts to assist individuals with disabilities in
their efforts to avail themselves of services and programs offered by the University. To this end,
Louisiana State University will provide reasonable accommodations for persons with
documented qualifying disabilities. If you have a disability and feel you need accommodations in
this course, you must present a letter to me from Disability Services in 115 Johnston Hall,
indicating the existence of a disability and the suggested accommodations.
\\

\noindent \textbf{Academic Integrity}
According to section 10.1 of the LSU Code of Student Conduct, ``A student may be charged with Academic Misconduct'' for a variety of offenses, including the following: unauthorized copying, collusion, or collaboration; ``falsifying'' data or citations; ``assisting someone in the commission or attempted commission of an offense''; and plagiarism, which is defined in section 10.1.H as a ``lack of appropriate citation, or the unacknowledged inclusion of someone else's words, structure, ideas, or data; failure to identify a source, or the submission of essentially the same work for two assignments without permission of the instructor(s).''
\\

\noindent \textbf{Plagiarism and Citation Method}
Plagiarism is the ``lack of appropriate citation, or the unacknowledged inclusion of someone else's words, structure, ideas, or data; failure to identify a source, or the submission of essentially the same work for two assignments without permission of the instructor(s)'' (Sec. 10.1.H of the LSU Code of Student Conduct). As a student at LSU, it is your responsibility to refrain from plagiarizing the academic property of another and to utilize appropriate citation method for all coursework. In this class, it is recommended that you use Chicago Style author-date citations. Ignorance of the citation method is not an excuse for academic misconduct.
%\\

\end{document}
