%!TEX TS-program = xelatex
%!TEX encoding = UTF-8 Unicode

\documentclass[11pt,article,oneside]{memoir}

% packages
\usepackage{org-preamble-xelatex}
\usepackage{wallpaper}
\usepackage{xcolor}
\usepackage{multicol}
\usepackage{enumitem}
\setlist[itemize]{leftmargin=*}
\usepackage{tikz}
\usepackage{tikzpagenodes} 

\AtBeginBibliography{\small}

% Definitions
\def\myauthor{Author}
\def\mytitle{Title}
\def\mycopyright{\myauthor}
\def\mykeywords{}
\def\mybibliostyle{plain}
\def\mybibliocommand{}
\def\mysubtitle{}
\def\myaffiliation{Louisiana State University}
\def\myaddress{309 Design}
\def\myemail{baharmon@lsu.edu} 
\def\myweb{https://baharmon.github.io/}
\def\myphone{919.622.8414}
\def\myversion{}
\def\myrevision{}
\def\myaffiliation{\ \\Louisiana State University}
\def\myauthor{Brendan Harmon}
\def\mykeywords{Landscape Architecture, Syllabus, Graduate}
\def\mysubtitle{Syllabus}
\def\mytitle{ \includegraphics[width=6cm]{../logos/lsu_art_design_logo.pdf} \\[0.1cm] {\normalfont \normalsize LA 7052 |} \Large Thesis Preparation} 

% color
\makeatletter
\newcommand{\globalcolor}[1]{%
  \color{#1}\global\let\default@color\current@color
}
\makeatother

% begin
\begin{document}

\setlength\bibitemsep{0.5em}

% fonts
\defaultfontfeatures{}
\defaultfontfeatures{Scale=MatchLowercase}
\setmainfont{IBM Plex Sans}
\setmonofont[Scale=0.8]{IBM Plex Mono}

\def\ind{\hangindent=1 true cm\hangafter=1 \noindent}
\def\labelitemi{$\cdot$}
\chapterstyle{article-4-sans}  

\title{\LARGE \mytitle}
\author{\Large\myauthor \newline \footnotesize\texttt{\noindent\myemail}}
\date{Spring 2024 \newline Tuesday \& Thursday \newline 16:30--18:00 \newline Design 308}
\published{\,}

% -------------------------------- COVER PAGE -------------------------------- 

%\pagenumbering{gobble}
%\globalcolor{black}
%\begin{tikzpicture}[remember picture,overlay]
%\fill[white] ([xshift=-1cm,yshift=3.5cm]current page text area.west) rectangle (7.25,2);
%\end{tikzpicture}
%\vspace*{-10em}
%\maketitle
%\ThisCenterWallPaper{1.1}{../images/}
%\clearpage


% -------------------------------- DESCRIPTION -------------------------------- 

\pagenumbering{arabic}
\globalcolor{black}

\vspace*{-10em}
\maketitle

\section{Course Description}
This course is preparation for your thesis. 
On Tuesdays you will learn about writing, theory, and methods. 
On Thursdays you will work on your thesis -- 
formulating your research questions,
conducting your literature review, 
developing your outline, 
planning your methods, 
and writing your introduction. 

% -------------------------------- SCHEDULE -------------------------------- 
\section{Schedule}

\begin{table}[H]
\begin{tabular}{l l l l}
\textbf{1} & \textbf{Writing:} Style \& References \hspace{4em}  & \textbf{8} & \textbf{Theory:} Object Oriented Ontology \\
\textbf{2} & \textbf{Theory:} Landscape Infrastructure & \textbf{9} & \textbf{Methods:}  Case Studies \\
\textbf{3} & \textbf{Methods:} Cartography & \textbf{10} & \textbf{Writing:} Typesetting \\
\textbf{4} & \textbf{Writing:} Typography & \textbf{11} & \textbf{Theory:} Landscape Topology \\
\textbf{5} & \textbf{Theory:} Vibrant Matter & \textbf{12} & \textbf{Methods:} Open Science \\
\textbf{6} & \textbf{Methods:} Hauntology & \textbf{13} & \textbf{Writing:} Publishing \\
\textbf{7} & \textbf{Writing:} Thesis Structure & \textbf{14} & \textbf{Writing:} Figures \\
\end{tabular}
\end{table}

% -------------------------------- Projects -------------------------------- 
\section{Projects}
%
\begin{table}[H]
\begin{tabular}{l @{\hskip 2.25cm} l @{\hskip 2.25cm} l}
\textbf{5} \enspace Literature Review & \textbf{10}  \enspace Introduction & \textbf{15} \enspace Methodology
\end{tabular}
\end{table}

% -------------------------------- Grading -------------------------------- 
\section{Assignments}
%
\begin{table}[H]
\begin{tabular}{l @{\hskip 0.8cm} l @{\hskip 0.8cm} l @{\hskip 0.8cm} l}
\textbf{25\%} \enspace Participation & \textbf{25\%} \enspace Lit.~Review & \textbf{25\%} \enspace Introduction & \textbf{25\%} \enspace Methods \\
\end{tabular}
\end{table}

% -------------------------------- Readings -------------------------------- 

\section{1 Style \& References}
\nocite{*}
\setlength\bibitemsep{0.65\baselineskip}
\printbibliography[keyword=style, heading=none]

\section{2 Landscape Infrastructure}
\nocite{*}
\setlength\bibitemsep{0.65\baselineskip}
\printbibliography[keyword=infrastructure, heading=none]

%\section{3 Cartography}
%\nocite{*}
%\setlength\bibitemsep{0.65\baselineskip}
%\printbibliography[keyword=cartography, heading=none]

\section{5 Vibrant Matter}
\nocite{*}
\setlength\bibitemsep{0.65\baselineskip}
\printbibliography[keyword=matter, heading=none]

\section{8 Object Oriented Ontology}
\nocite{*}
\setlength\bibitemsep{0.65\baselineskip}
\printbibliography[keyword=ooo, heading=none]

\section{9 Case Studies}
\nocite{*}
\setlength\bibitemsep{0.65\baselineskip}
\printbibliography[keyword=case-studies, heading=none]


% -------------------------------- Software -------------------------------- 

\section{Software}
\vspace*{1em}
\begin{minipage}{0.48\textwidth}
Zotero | \url{https://www.zotero.org/}\\
\end{minipage}
\hfill
\begin{minipage}{0.48\textwidth}
Overleaf | \url{https://www.overleaf.com/}\\
\end{minipage}

% -------------------------------- Policies -------------------------------- 

\section{Policies}

\paragraph{Time Commitment Expectations}
LSU's general policy states that for each credit hour, you (the student) should plan to
spend at least two hours working on course related activities outside of class. Since this course is for three credit hours, you should expect to spend a minimum of six hours outside of class each week working on assignments for this course. For more information see: 
\url{http://catalog.lsu.edu/content.php?catoid=12&navoid=822}.

\paragraph{LSU student code of conduct}
The LSU student code of conduct explains student rights, excused absences, and what is expected of student behavior. Students are expected to understand this code:  \url{http://students.lsu.edu/saa/students/code}.
%Any violations of the LSU student code will be duly reported to the Dean of Students.

\paragraph{Disability Code}
The University is committed to making reasonable efforts to assist individuals with disabilities in
their efforts to avail themselves of services and programs offered by the University. To this end,
Louisiana State University will provide reasonable accommodations for persons with
documented qualifying disabilities. If you have a disability and feel you need accommodations in
this course, you must present a letter to me from Disability Services in 115 Johnston Hall,
indicating the existence of a disability and the suggested accommodations.

\paragraph{Academic Integrity}
According to section 10.1 of the LSU Code of Student Conduct, ``A student may be charged with Academic Misconduct'' for a variety of offenses, including the following: unauthorized copying, collusion, or collaboration; ``falsifying'' data or citations; ``assisting someone in the commission or attempted commission of an offense''; and plagiarism, which is defined in section 10.1.H as a ``lack of appropriate citation, or the unacknowledged inclusion of someone else's words, structure, ideas, or data; failure to identify a source, or the submission of essentially the same work for two assignments without permission of the instructor(s).''

\paragraph{Plagiarism and Citation Method}
Plagiarism is the ``lack of appropriate citation, or the unacknowledged inclusion of someone else's words, structure, ideas, or data; failure to identify a source, or the submission of essentially the same work for two assignments without permission of the instructor(s)'' (Sec. 10.1.H of the LSU Code of Student Conduct). As a student at LSU, it is your responsibility to refrain from plagiarizing the academic property of another and to utilize appropriate citation method for all coursework. In this class, it is recommended that you use Chicago Style author-date citations. Ignorance of the citation method is not an excuse for academic misconduct.

\paragraph{Expectations}
As an accredited Landscape Architecture program
LSU's Robert Reich School of Landscape Architecture (RRSLA) 
must meet the accreditation requirements 
as stated by the Landscape Architectural Accreditation
Board (LAAB) to ensure RRSLA is meeting the expectations of the field. 
The LAAB requires programs to provide digital copies 
of student work as part of this process.
Students in this course will be expected 
to comply with the following requirements
as 5\% of their course grade: 
(1) Students must provide a course portfolio
with work samples specified by the instructor 
before the end of the grading period. 
(2) Each student's course portfolio must be saved as 
a single, high resolution PDF file with multiple pages. 
(3) Files must follow the naming convention
established by the school: department-coursenumber-semesteryear-username.pdf.
Example: LA7201-F2023-baharmon.pdf.

\end{document}
